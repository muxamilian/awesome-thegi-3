\documentclass[a4paper,10pt]{article}
\usepackage[utf8]{inputenc}
\usepackage{amsmath}
\usepackage{amsfonts}
\usepackage{amssymb}
\usepackage[german]{babel}
\setlength{\parindent}{0cm}
\usepackage{setspace}
\usepackage{mathpazo}
\usepackage{graphicx}
\usepackage{wasysym} 
\usepackage{booktabs}
\usepackage{verbatim}
\usepackage{enumerate}
\usepackage{hyperref}
\usepackage{ulem}
\usepackage{stmaryrd }
\usepackage[a4paper,
left=1.8cm, right=1.8cm,
top=2.0cm, bottom=2.0cm]{geometry}
\usepackage{tabularx}
\begin{document}

\begin{center}
\Large{Theoretische Grundlagen der Informatik 3: Hausaufgabenabgabe 6} \\
\large{Tutorium: Sebastian , Mi 14.00 - 16.00 Uhr}
\end{center}
\begin{tabbing}
Tom Nick \hspace{2cm}\= - 340528\\
Maximillian Bachl \> - 341455 \\
Marius Liwotto\> -  341051
\end{tabbing}
\subsection*{Aufgabe 1}
\begin{enumerate}
	\item[(i)]
			\( h: V(G) \rightarrow V(H)\\
			v_1 \mapsto w_1 \\
			v_2 \mapsto w_2 \\
			v_3 \mapsto w_1 \\
			v_4 \mapsto w_3 \\
			v_5 \mapsto w_2 \)

	\item[(ii)]
			Zu zeigen ist: \\
			(i) Graph G ist 3-färbbar $\Rightarrow$ es existiert ein Homomorph. von G nach H\\
			(ii) es existiert ein Homomorph. von G nach H $\Rightarrow$ Graph G ist 3-färbbar\\
			\\
			H ist 3-färbbar mit folgender Farbbelegung \\
			c: V(H) $\rightarrow \{r,g,b\}$ \\
			\( c(w_1) \mapsto r \\
			c(w_2) \mapsto g \\
			c(w_3) \mapsto b \)
\begin{enumerate}
	\item[(i)]
		Da G 3-färbbar ist, gilt: \\
		\( \exists c': V(G) \rightarrow $\{r,g,b\}$.~\text{c' ist eine 3-Färbung von G} \\
		\\
		\text{Daraus kann man nun folgenden Homomorphismus h bilden:} \\
		\\
		h: V(G) \rightarrow V(H) \\
		h(v) \mapsto 
		\begin{cases}
		w_1, & c'(v) = r \\
		w_2, & c'(v) = g \\
		w_3, & c'(v) = b \\
		\end{cases} \\
		\\
		\text{Beweis der Richtigkeit des gebildeten h:} \\
		\text{Es muss gelten:} \\
		\\
		\forall \{u,v\} \in V(G).~\{h(u),h(v)\} \in V(H)\\
		\\
		\text{Da nach Annahme alle Knoten einer Kante aus G verschiedenfarbig sind, gilt:}\\
		\\
		(1)~\forall \{u,v\} \in V(G).~h(u) \neq h(v)\\
		\\
		\text{Des Weiteren gilt:}\\
		\\
		(2)~\forall~ u \in \{w_1,w_2,w_3\}.~\forall~ v \in \{w_1,w_2,w_3\} \setminus{ \{u\} }.~\{u,v\} \in E(H) \\
		\\
		\text{Aus (1) und (2) folgt:} \\
		\\
		\forall \{u,v\} \in V(G).~\{h(u),h(v)\} \in V(H) \)
\item[(ii)]
Da ein Homomorphismus h von G nach H existiert, gilt: \\
\\
\((*)~ \forall \{u,v\} \in V(G).~\{h(u),h(v)\} \in V(H)\\
\\
\\
\\
\\
\\
\\
\\
\text{Daraus kann man nun folgende 3-Färbung c' ableiten:} \\
\\
c': V(G) \rightarrow $\{r,g,b\}$ \\
\\
c'(v) \mapsto c(h(v)) \\
\\
\text{Beweis der Richtigkeit der 3-Färbung c':}\\
\\
\text{Es gilt:} \\
\\
(**)~\forall~ \{u,v\} \in V(G).~h(u) \neq h(v) \\
\\
\text{Würde dies nicht gelten, würde das bedeuten, dass $\{h(u),h(u)\} \in E(H)$ sein würde, da aber} \\
\text{H irreflexiv ist, wäre das ein Widerspruch. }\\
\text{Aus (*) und (**) folgt:}\\
\\
\forall~ \{u,v\} \in V(G).~c(h(u)) \neq c(h(v)) 
\Leftrightarrow
\forall~ \{u,v\} \in V(G).~c'(u) \neq c'(v) \\
\\
\Rightarrow \text{c' ist eine passende 3-Färbung für G}
\Rightarrow \text{G ist 3-färbbar}
\)
\end{enumerate}

\end{enumerate}
\subsection*{Aufgabe 4}
Wir definieren $n = |V(G)|$. Es werden $n$ Variablen eingeführt, die jeweils $n$ Indizes haben mit der Namensgebung $X_{i,j}$. Die Idee ist, dass diese $n \times n$ Matrix aus Variablen einen Pfad definiert, indem man jeweils genau eine Variable der $n$ Variablen mit 1 belegt, alle anderen mit 0. Nun würden die Variablen $X_{i,j}$ und $X_{i+1,k}$ für die Kante $\{j,k\}$ stehen. $\varphi$ muss folgendes leisten:
\begin{enumerate}
	\item 	Jede Variable definiert genau einen Knoten.
		\[\varphi_1 = \bigwedge_{0 \le i < n } ~\bigvee_{j \in {V(G)}} \left( X_{i,j} \land \left( \bigwedge_{k \in V(G)\setminus\{ j\}} ~ \lnot X_{i,k} \right) \right) \]
		Jede Klausel hat die Grösse $n$, es werden $n$ mal $n$ Klauseln gebildet $\Rightarrow \varphi_1 \in \mathcal{O}(n^3)$
	\item 	Jeder Knoten kommt genau einmal vor:
		\[\varphi_2 = \bigwedge_{j \in {V(G)}} ~\bigvee_{0 \le i < n } \left( X_{i,j} \land \left( \bigwedge_{i' \in \{0,...,n-1 \}\setminus\{ i\}} ~ \lnot X_{i',j} \right) \right)\]
		Jede Klausel hat die Grösse $n$, es werden $n$ mal $n$ Klauseln gebildet $\Rightarrow \varphi_2 \in \mathcal{O}(n^3)$
	\item  	Die Kanten existieren und bilden einen geschlossenen Pfad:
		\[\varphi_3 = X_{n-1,0}\bigwedge_{0 \le i <n}~ \bigvee_{\{k,l\} \in E(G)} X_{i,k} \land X_{i+1,l}  \]
		Jede Klausel hat die Grösse $2$, es werden $n$ mal maximal $\binom{n}{2}$ Klauseln gebildet $\Rightarrow \varphi_3 \in \mathcal{O}(n^3)$
\end{enumerate}
$\varphi$ wird nun als Konjunktion des Ganzen definiert: \[\varphi(G) = \varphi_1 \land \varphi_2 \land \varphi_3 \] 
$\varphi(G)$ genau erfüllbar wenn ein Hamilton Kreis in $G$ existiert. Da für $\varphi_{1-3}$ gezeigt wurde, dass es polynomiellen Aufwand hat, ist der Aufwand $\varphi$ zu bilden auch polynomiell, da $n^2$ Variablen eingeführt wurden, wäre der Aufwand um eine Belegung zu finden $\mathcal{O}(n^2)$
\end{document}


