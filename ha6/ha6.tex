\documentclass[a4paper,10pt]{article}
\usepackage[utf8]{inputenc}
\usepackage{amsmath}
\usepackage{amsfonts}
\usepackage{amssymb}
\usepackage[german]{babel}
\setlength{\parindent}{0cm}
\usepackage{setspace}
\usepackage{mathpazo}
\usepackage{graphicx}
\usepackage{wasysym} 
\usepackage{booktabs}
\usepackage{verbatim}
\usepackage{enumerate}
\usepackage{hyperref}
\usepackage{ulem}
\usepackage{stmaryrd }
\usepackage[a4paper,
left=1.8cm, right=1.8cm,
top=2.0cm, bottom=2.0cm]{geometry}
\usepackage{tabularx}

\newcommand{\tabspace}{15cm}

\begin{document}
\begin{center}
\Large{Theoretische Grundlagen der Informatik 3: Hausaufgabenabgabe 6} \\
\large{Tutorium: Sebastian , Mi 14.00 - 16.00 Uhr}
\end{center}
\begin{tabbing}
Tom Nick \hspace{2cm}\= - 340528\\
Maximillian Bachl \> - 341455 \\
Marius Liwotto\> -  341051
\end{tabbing}
\section*{Aufgabe 1}
\begin{enumerate}
	\item[(i)]
	\( h: V(G) \rightarrow V(H)\\
	v_1 \mapsto w_1 \\
	v_2 \mapsto w_2 \\
	v_3 \mapsto w_1 \\
	v_4 \mapsto w_3 \\
	v_5 \mapsto w_2 \)

	\item[(ii)]
			Zu zeigen ist: \\
			(i) Graph G ist 3-färbbar $\Rightarrow$ es existiert ein Homomorph. von G nach H \\
			(ii) es existiert ein Homomorph. von G nach H $\Rightarrow$ Graph G ist 3-färbbar\\
			\\
			H ist 3-färbbar mit folgender Farbbelegung \\
			c: V(H) $\rightarrow \{r,g,b\}$ \\
			\( c(w_1) \mapsto r \\
			c(w_2) \mapsto g \\
			c(w_3) \mapsto b \)
\begin{enumerate}
	\item[(i)]
		Da G 3-färbbar ist, gilt: \\
		\( \exists c': V(G) \rightarrow $\{r,g,b\}$.~\text{c' ist eine 3-Färbung von G} \\
		\\
		\text{Daraus kann man nun folgenden Homomorphismus h bilden:} \\
		\\
		h: V(G) \rightarrow V(H) \\
		h(v) \mapsto 
		\begin{cases}
		w_1, & c'(v) = r \\
		w_2, & c'(v) = g \\
		w_3, & c'(v) = b \\
		\end{cases} \\
		\\
		\text{Beweis der Richtigkeit des gebildeten h:} \\
		\text{Es muss gelten:} \\
		\\
		\forall \{u,v\} \in E(G).~\{h(u),h(v)\} \in E(H)\\
		\\
		\text{Da nach Annahme alle Knoten einer Kante aus G verschiedenfarbig sind, gilt:}\\
		\\
		(1)~\forall \{u,v\} \in E(G).~h(u) \neq h(v)\\
		\\
		\text{Des Weiteren gilt:}\\
		\\
		(2)~\forall~ u \in \{w_1,w_2,w_3\}.~\forall~ v \in \{w_1,w_2,w_3\} \setminus{ \{u\} }.~\{u,v\} \in E(H) \\
		\\
		\text{Aus (1) und (2) folgt:} \\
		\\
		\forall \{u,v\} \in E(G).~\{h(u),h(v)\} \in E(H) \)
\item[(ii)]
Da ein Homomorphismus h von G nach H existiert, gilt: \\
\\
\((*)~ \forall \{u,v\} \in E(G).~\{h(u),h(v)\} \in E(H)\\
\\
\\
\\
\\
\\
\\
\\
\text{Daraus kann man nun folgende 3-Färbung c' ableiten:} \\
\\
c': V(G) \rightarrow $\{r,g,b\}$ \\
\\
v \mapsto c(h(v)) \\
\\
\text{Beweis der Richtigkeit der 3-Färbung c':}\\
\\
\text{Es gilt:} \\
\\
(**)~\forall~ \{u,v\} \in E(G).~h(u) \neq h(v) \\
\\
\begin{tabular}{@{} p{\tabspace}}
Würde dies nicht gelten, würde das bedeuten, dass $\{h(u),h(u)\} \in E(H)$ sein würde, da aber
H irreflexiv ist, wäre das ein Widerspruch. 
Aus (*) und (**) folgt:
\end{tabular}\\
\\
\forall~ \{u,v\} \in E(G).~c(h(u)) \neq c(h(v)) 
\Leftrightarrow
\forall~ \{u,v\} \in E(G).~c'(u) \neq c'(v) \\
\\
\Rightarrow \text{c' ist eine passende 3-Färbung für G}
\Rightarrow \text{G ist 3-färbbar}
\)
\end{enumerate}
\end{enumerate}

\section*{Aufgabe 2}
\begin{tabular}{@{} p{\tabspace}}
Zu zeigen ist:\\
(i) Existieren für alle endlichen Teilgraphen G' von G Homomorphismen von G' nach H, so 
existiert auch ein Homomorphismus von G nach H.\\
\\
(ii) Existiert für G ein Homomorphismus von G nach H, so existieren für alle
endlichen Teilgraphen G' von G Homomorphismen von G' nach H.
\end{tabular}

\begin{enumerate}
	\item[(i)]
	Der Graph G = (V,E) wird folgendermaßen in einer Aussagenlogische Formel übersetzt:\\
	Für jede Kante $\{u,v\} \in$ E(G) führen wir eine Variable $X_{u,v}$ ein.\\
	\\
	Sei h: V(G) $\rightarrow$ V(H) \\
	\\
	\( \Phi = \{\{X_{u,v} \rightarrow X_{h(u),h(v)}\} ~|~ \{u,v\} \in E(G) \} \\
	\\
	\text{Sei $\Phi_0 \subseteq \Phi$} \\
	\\
	E' = \{\{u,v\} ~|~ \{X_{u,v} \rightarrow X_{h(u),h(v)}\} \in \Phi_0 \} \\
	V' = \{v ~|~ \{u,v\} \in E'\} \\
	\\
	\begin{tabular}{@{} p{\tabspace}}	
	Da $\Phi_0$ eine endliche Teilmenge von $\Phi$ ist, muss auch E' und damit V' endlich sein.
	Für den von V' induzierten endlichen Untergraphen G' von G existiert
	nach Annahme ein Homomorphismus h': V(G') $\rightarrow$ V(H).
	Daraus folgt, dass folgendes gilt:
	\end{tabular} \\
	\\
	\forall~ \{u,v\} \in E'.~\{h'(u),h'(v)\} \in E(H) \\
	\\
	\text{Eine passende Belegung $\beta$ wäre somit Folgende:}\\
	\\
	\beta(X_{u,v}) = 1 \text{, wenn $\{u,v\} \in$ E' und } 
	\beta(X_{h'(u),h'(v)}) = 1 \text{, wenn $\{h'(u),h'(v)\} \in$ E(H)} \)\\
	\\
	\begin{tabular}{@{} p{\tabspace}}
	Aus der Erfüllbarkeit von jedem beliebigen $\Phi_0$ folgt nach dem Kompaktheitssatz, dass $\Phi$ auch erfüllbar ist.
	Daraus folgt, dass ein Homomorphismus h: V(G) $\rightarrow$ V(H) existiert, wenn für alle Teilgraphen G'
	ein Homomorphismus existiert.
	\end{tabular}
	
	\item[(ii)]
	Ist trivial.
\end{enumerate}

\section*{Aufgabe 3}
\begin{tabular}{@{} p{\tabspace}}
In der Hausaufgabe 4 wurde bewiesen, dass die P-Resolution korrekt ist, d.h. falls es eine Menge von Klausels eine P-Resolutionswiderlegung hat, so ist die Klauselmenge unerfüllbar. Damit die P-Resolution vollständig ist, muss gezeigt werden, dass für jede unerfüllbare Klauselmenge eine P-Resolutionswiderlegung existiert. Dazu wird zuerst folgende Behauptung aufgestellt: \\
\end{tabular}
\ \\
\textbf{Behauptung: } Sei $n \in \mathbb{N}$ und sei $\mathcal{C}$ eine unerfüllbare Klauselmenge in den Variablen $\{V_1,...,V_{n-1} \}$. Dann hat $\mathcal{C}$ eine Resolutionswiderlegung. 
\begin{itemize}
	\item[\textbf{IA:}] 
		\begin{tabular}{@{} p{\tabspace}}
		$n=0$. \\
		In diesem Fall ist $\mathcal{C}$ unerfüllbar und enthält keine Variablen. Also $\mathcal{C} := \{ \square \}$ und somit existiert eine Resolutionswiderlegung, bzw. das ganze ist schon eine Resolutionswiderlegung.
		\end{tabular}
	\item[\textbf{IS:}] $n \rightarrow n+1$ \\
	
		Sei $\mathcal{C}$ eine unerfüllbare Klauselmenge in den Variablen $\{V_1,...,V_n \}$. Wir definieren
		\begin{align*}
			\mathcal{C}^+ &:= \{C \setminus \{ \lnot V_n  \} \mid C \in \mathcal{C} \land V_n \not \in C \} \\
			\mathcal{C}^- &:= \{C \setminus \{V_n  \} \mid C \in \mathcal{C} \land \lnot V_n \not \in C \}
		\end{align*}
		
		\begin{tabular}{@{} p{\tabspace}}
		$\mathcal{C}^+$ und $\mathcal{C}^-$ sind beide unerfüllbar. Denn wäre z.B. $\mathcal{C}^+$ erfüllbar, z.B. durch $\beta \vDash \mathcal{C}^+$, dann würde $\beta' := \beta \cup \{V_n \rightarrow 1 \}$ die Menge $\mathcal{C}$ erfüllen. \\
		Nach Induktionsvorraussetzung gibt es Resolutionsableitungen $(C_1,...,C_s)$ und $(D_1,...,D_t)$ der leeren Klausel $C_s = D_t = \square$ aus $\mathcal{C}^+$ bzw. $\mathcal{C}^-$. \\
		 \ \\
		Falls  $(D_1,...,D_t)$  schon eine Ableitung von $\square$ aus $\mathcal{C}$ ist, sind wir fertig. Andernfalls werden Klauseln $D_i$ benutzt, die aus $\mathcal{C}$ durch Entfernen von $V_n$ entstanden sind, d.h. $D_i \cup \{V_n \} \in \mathcal{C}$. \\
		Fügen wir zu diesen Klauseln und allen Resolventen wieder $V_n$ hinzu, so erhalten wir eine Ableitung $(D_1',...,D_t')$ von $V_n$ aus $\mathcal{C}$. \\
		 \ \\
		 Falls $(C_1,..., C_s)$ schon eine Ableitung von $\square$ aus $\mathcal{C}$ ist, sind wir fertig. Andernfalls werden Klauseln $C_i$ benutzt, die aus $\mathcal{C}$ durch Entfernen von $\lnot V_n$ entstanden sind, d.h. $C_i \cup \{\lnot V_n \} \in \mathcal{C}$. \\
		Fügen wir zu diesen Klausel $\lnot V_n$ hinzu, gibt es in der normalen Resolution eine Ableitung $(C_1',...,C_s')$ von $\lnot V_n$ aus $\mathcal{C}$. 
		\ \\Doch es gibt in der P-Resolution das Problem, dass immer eine positive Klausel resolviert werden muss, und da wir negative Variablen zu den Klauseln hinzufügten, ist $(C_1',...C_s')$ unter Umständen keine gültige P-Resolution. \\
		\ \\
		Doch haben wir schon zuvor $\{V_n \}$ resolviert, wodurch alle Klauseln der Form $C_i \cup \{\lnot V_n \} \in \mathcal{C}$ wieder zu $C_i$ resolviert werden können. Diese Resolutionskette bezeichne als $(C'D'_1,...,C'D'_k)$. Nun sind die Resolutionsableitungen $(C_1,..., C_s)$ möglich, da alle Klauseln vorliegen. \\
		\ \\
		Damit wäre $(D_1',...,D'_t,C'D'_1,...,C'D'_k,C_1,...,C_s,\square)$ eine Resolutionswiderlegung von $\mathcal{C}$ 
	\end{tabular}
\end{itemize}

\begin{tabular}{@{} p{\tabspace}}
Nun muss das ganze noch für jede Klauselmenge $\mathcal{C}$ bewiesen werden. Sei dazu $\mathcal{C}$ eine unerfüllbare Klauselmenge
\begin{enumerate}
	\item Ist $\mathcal{C}$ endlich, dann enthält sie nur endlich viele Variablen und der Beweis folgt sofort aus der Behauptung. 
	\item Ist $\mathcal{C}$ unendlich, dann folgt aus dem Kompaktheitssatz, dass bereits eine endliche Teilmenge $\mathcal{C}'\subseteq \mathcal{C} $unerfüllbar ist. Also hat $\mathcal{C}'$ eine Resolutionswiderlegung. Diese ist aber auch eine Resolutionswiderlegung von $\mathcal{C}$.	
\end{enumerate}
\end{tabular}

\textbf{Hinweis: Große Teile des Beweises kommen aus den Vorlesungsfolien der VL TheGI3 von S. Kreutzer. }
\section*{Aufgabe 4}
\begin{tabular}{@{} p{\tabspace}}
Wir definieren $n = |V(G)|$. Es werden $n$ Variablen eingeführt, die Indizes haben mit der Namensgebung $X_{i,j}$. Die Idee ist, dass diese $n \times n$ Matrix aus Variablen einen Pfad definiert, indem man jeweils genau eine Variable der $n$ Variablen mit 1 belegt, alle anderen mit 0. Nun würden die Variablen $X_{i,j}$ und $X_{i+1,k}$ für die Kante $\{j,k\}$ stehen. $\varphi$ muss folgendes leisten:
\end{tabular}
\begin{enumerate}
	\item 	Jede Variable definiert genau einen Knoten.
		\[\varphi_1 = \bigwedge_{0 \le i < n } ~\bigvee_{j \in {V(G)}} \left( X_{i,j} \land \left( \bigwedge_{k \in V(G)\setminus\{ j\}} ~ \lnot X_{i,k} \right) \right) \]
		Jede Klausel hat die Größe $n$, es werden $n$ mal $n$ Klauseln gebildet $\Rightarrow \varphi_1 \in \mathcal{O}(n^3)$
	\item 	Jeder Knoten kommt genau einmal vor:
		\[\varphi_2 = \bigwedge_{j \in {V(G)}} ~\bigvee_{0 \le i < n } \left( X_{i,j} \land \left( \bigwedge_{i' \in \{0,...,n-1 \}\setminus\{ i\}} ~ \lnot X_{i',j} \right) \right)\]
		Jede Klausel hat die Größe $n$, es werden $n$ mal $n$ Klauseln gebildet $\Rightarrow \varphi_2 \in \mathcal{O}(n^3)$
	\item  	Die Kanten existieren und bilden von jeglichem Startpunkt einen geschlossenen Pfad:

		\[\varphi_3 = \bigwedge_{0 \le m < n}\bigwedge_{0 \le i <n}~ \left(\bigvee_{\{k,l\} \in E(G)} X_{(i+m)\text{mod}~n,k} \land X_{((i+m)\text{mod}~n)+1,l} \right) \]
		Jede Klausel hat die Größe $2$, es werden $n^2$ mal maximal $\binom{n}{2}$ Klauseln gebildet $\Rightarrow \varphi_3 \in \mathcal{O}(n^4)$

\end{enumerate}
\begin{tabular}{@{} p{\tabspace}}
$\varphi$ wird nun als Konjunktion des Ganzen definiert: \[\varphi(G) = \varphi_1 \land \varphi_2 \land \varphi_3 \] 
$\varphi(G)$ ist genau erfüllbar, wenn ein Hamilton Kreis in $G$ existiert. Da für $\varphi_{1-3}$ gezeigt wurde, dass es polynomiellen Aufwand hat, ist der Aufwand von $\varphi$ ebenfalls polynomiell, nämlich\\
$\mathcal{O}(n^3) + \mathcal{O}(n^3) + \mathcal{O}(n^4) = \mathcal{O}(n^4)$.

\end{tabular}
%Zuerst führen wir Variablen $V_i$ für jeden Knoten ein. Außerdem führen wir die Variable $X_{i,j}$ für eine Kante $\{i,j\}$ ein. Für jedes $X_{i,j}$ führen wir $X_{j,i}$ ein. Alle diese Variablen werden in ein $2|E|$-Tupel namens T gesteckt, wobei die Reihenfolge egal ist.
%
%\begin{verbatim}
%len = |E|
%for i in range(0,len-1):
%    var = T[i]
%    var_after_var = T[i+1]
%    second_index = var[1] # gets the second index of the variable
%    for j in range(j,2*len):
%        looked_for_var = T[j]
%        first_index = looked_for_var[0]
%        if(second_index == first_index):
%            swap(var_after_var, looked_for_var)
%            break
%T = T[:len-1] # take the first len-1 elements (because the one that links
%              # the last and the first must be added separately
%\end{verbatim}
%Nun stellen wir die folgende Formel $\varphi$ auf mit dem sortierten T, das der Algorithmus zurückgegeben hat:
%$$ \varphi = \left(\bigwedge_{v \in T} v\right) \land \left(\bigwedge_{1\le i < j \le |E|} X_{i,j} \leftrightarrow X_{j,i}\right) \land X_{T[n-1],T[0]} \land \left(\bigwedge_{\right)$$
\end{document}


