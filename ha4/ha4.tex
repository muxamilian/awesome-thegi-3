\documentclass[a4paper,10pt]{article}
\usepackage[utf8]{inputenc}
\usepackage{amsmath}
\usepackage{amsfonts}
\usepackage{amssymb}
\usepackage[german]{babel}
\setlength{\parindent}{0cm}
\usepackage{setspace}
\usepackage{mathpazo}
\usepackage{graphicx}
\usepackage{wasysym} 
\usepackage{booktabs}
\usepackage{verbatim}
\usepackage{enumerate}
\usepackage{ulem}
\usepackage{stmaryrd }
\usepackage[a4paper,
left=1.8cm, right=1.8cm,
top=2.0cm, bottom=2.0cm]{geometry}
\usepackage{tabularx}

\begin{document}

\begin{center}
\Large{Theoretische Grundlagen der Informatik 3: Hausaufgabenabgabe 4} \\
\large{Tutorium: Sebastian , Mi 14.00 - 16.00 Uhr}
\end{center}
\begin{tabbing}
Tom Nick \hspace{2cm}\= - 340528\\
Maximillian Bachl \> - 341455 \\
Marius Liwotto\> -  341051
\end{tabbing}
	\subsection*{Aufgabe 1}
	\begin{enumerate}
	\item[(i)]
		Der Aufbau von \(\varphi\) besitzt 3 Muster für Klauseln, welche auf dem Aufgabenblatt auch untereinander dargestellt sind. Man kann \(\varphi \) auch schreiben als:
		\[ \bigvee_{X \in \{A,B,C\}} (X_0 \lor X_1) \land \bigvee_{Y \in \{A,B,C \}\setminus\{X\} }(\lnot X_0 \lor \lnot Y_0) \land (\lnot X_1 \lor \lnot Y_1) \]
		Diese Form lässt es offensichtlich machen, warum
		\begin{enumerate}[1.]
			\item $\varphi$ nicht erfüllbar ist: \\
				In der zweiten grossen Konjunktion, kommt in den Klauseln jeweils die negierte Version der Variablen der Klausel aus der ersten grossen Konjunktion vor $( (X_0 \lor X_1) \land (\lnot X_0 \lor ..) \land (\lnot X_1 \lor ..))$. Da dies jedoch für alle Variablen aus $\varphi$ gleichermassen gilt, kann $\varphi$ nicht erfüllbar sein.
			\item $\varphi$ erfüllbar ist, wenn eine beliebieg Klausel entfernt wird:
				\begin{enumerate}[\text{Fall} 1:]
				\item Eine Klausel des Typs $(X_0 \lor X_1)$  mit $ X \in \{ A,B,C\}$ wird weggelassen: \\
				$\Rightarrow X_0,X_1$ kann mit 0 belegt werden. \\
				$\Rightarrow $ Klauseln der Form $(\lnot X_0 \lor \lnot Y_0)$ und $(\lnot X_1 \lor \lnot Y_1)$ mit $ \in \{A,B,C \}\setminus\{X\}$  sind erfüllt\\
				$\Rightarrow$ Belege $Y_0,Z_1$ mit 1 und $Y_1,Z_0$ mit 0, wobei $Z \in \{A,B,C\}\setminus\{X,Y\}$ \\
				$\Rightarrow$ Die Klauseln $(Y_0 \lor Y_1),(Z_0 \lor Z_1),(\lnot Y_0 \lor \lnot Z_0), (\lnot Y_1 \lor Z_1)$ sind erfüllt.
				\item Eine Klausel des Typs $(\lnot X_0 \lor \lnot Y_0)$ mit $X \in \{A,B,C\}, Y \in \{A,B,C \}\setminus\{X\}$ wird weggelassen.\\
				Belege $(X_0, Y_0)$ mit 1 \\
				$\Rightarrow$ $(X_1, Y_1)$ können mit 0 belegt werden. \\
				$\Rightarrow$ Klauseln der Form $(\lnot X_1 \lor \lnot Z_1)$ und $(\lnot Y_1 \lor \lnot Z_1)$ mit $Z \in \{A,B,C\}\setminus\{X,Y\}$ sind erfüllt\\
				$\Rightarrow Z_1$ kann mit 1 belegt werden.\\
				$\Rightarrow$ $(Z_0 \lor Z_1)$ ist erfüllt\\
				$\Rightarrow$ $Z_0 $ kann mit 0 belegt werden \\
				$\Rightarrow$ Klauseln der Form $(\lnot X_0 \lor \lnot Z_0)$ und $(\lnot Y_0 \lor \lnot Z_0)$ sind erfüllt \\
				\item Eine Klausel des Typs $(\lnot X_1 \lor \lnot Y_1)$ mit $X \in \{A,B,C\}, Y \in \{A,B,C \}\setminus\{X\}$ wird weggelassen.\\
				analog zu Fall 2.

				\end{enumerate}
		\end{enumerate}
	\item[(ii)]
	\begin{align*}
		\varphi \overset{res(\{ A_0,A_1\},\{ \neg A_0,\neg C_0\})}{\equiv} &\varphi \cup \{ A_1,\neg C_0 \} \\
		\overset{res(\{ A_1,\neg C_0 \},\{ \neg A_1,\neg B_1\})}{\equiv} &\varphi \cup \{ A_1,\neg C_0 \} \cup \{ \neg C_0, \neg B_1\}\\
		\overset{res(\{ B_0, B_1 \},\{ \neg B_1,\neg C_1\})}{\equiv} &\varphi \cup \{ A_1,\neg C_0 \} \cup \{ \neg C_0, \neg B_1\} \cup \{ B_0, \neg C_1\}\\
		\overset{res(\{ B_0, B_1 \},\{ \neg A_1,\neg B_1\})}{\equiv} &\varphi \cup \{ A_1,\neg C_0 \} \cup \{ \neg C_0, \neg B_1\} \cup \{ B_0, \neg C_1\} \cup \{\neg A_1, B_0\}\\
		\overset{res(\{ A_1,\neg C_0 \},\{\neg A_1, B_0\})}{\equiv} &\varphi \cup \{ A_1,\neg C_0 \} \cup \{ \neg C_0, \neg B_1\} \cup \{ B_0, \neg C_1\} \cup \{\neg A_1, B_0\} \cup \{\neg C_0, B_0 \}\\
		\overset{res(\{ \neg B_0,\neg C_0 \},\{\neg C_0, B_0\})}{\equiv} &\varphi \cup \{ A_1,\neg C_0 \} \cup \{ \neg C_0, \neg B_1\} \cup \{ B_0, \neg C_1\} \cup \{\neg A_1, B_0\} \cup \{\neg C_0, B_0 \} \cup \{\neg C_0\}\\
		\overset{res(\{ C_0, C_1 \},\{\neg C_0\})}{\equiv} &\varphi \cup \{ A_1,\neg C_0 \} \cup \{ \neg C_0, \neg B_1\} \cup \{ B_0, \neg C_1\} \cup \{\neg A_1, B_0\} \cup \{\neg C_0, B_0 \} \cup \{\neg C_0\} \cup \{C_1\}\\
		\overset{res(\{ \neg A_1, \neg C_1 \},\{C_1\})}{\equiv} &\varphi \cup \{ A_1,\neg C_0 \} \cup \{ \neg C_0, \neg B_1\} \cup \{ B_0, \neg C_1\} \cup \{\neg A_1, B_0\} \cup \{\neg C_0, B_0 \} \cup \{\neg C_0\} \cup \{C_1\} \\
		&\cup \{ \neg A_1 \}\\
		\overset{res(\{ B_0, \neg C_1 \},\{C_1\})}{\equiv} &\varphi \cup \{ A_1,\neg C_0 \} \cup \{ \neg C_0, \neg B_1\} \cup \{ B_0, \neg C_1\} \cup \{\neg A_1, B_0\} \cup \{\neg C_0, B_0 \} \cup \{\neg C_0\} \cup \{C_1\} \\
		&\cup \{ \neg A_1 \} \cup \{ B_0 \}\\
		\overset{...}{\equiv} &\varphi \cup \{ A_1,\neg C_0 \} \cup \{ \neg C_0, \neg B_1\} \cup \{ B_0, \neg C_1\} \cup \{\neg A_1, B_0\} \cup \{\neg C_0, B_0 \} \cup \{\neg C_0\} \cup \{C_1\} \\
		&\cup \{ \neg A_1 \} \cup \{ B_0 \} \cup \{\neg A_0\} \cup \{A_1\} \\
		\overset{res(\{\neg A_1\},\{ A_1 \})}{\equiv} &\varphi \cup \{ A_1,\neg C_0 \} \cup \{ \neg C_0, \neg B_1\} \cup \{ B_0, \neg C_1\} \cup \{\neg A_1, B_0\} \cup \{\neg C_0, B_0 \} \cup \{C_1\} \\
		&\cup \{ \neg A_1 \} \cup \{ B_0 \} \cup \{\neg A_0\} \cup \{A_1\} \cup \square
	\end{align*}
	Somit ist $\varphi$ unerfüllbar, da sich die leere Klausel resolvieren lässt.

	\end{enumerate}
	\subsection*{Aufgabe 2}
	\begin{enumerate}
	\item[(i)]
	Damit eine Klauselmenge nicht erfüllbar ist, muss es eine Resolutionswiderlegung geben, d.h. man muss aus der Klauselmenge die leere Klausel herleiten können.Die leere Klausel lässt sich wiederum  nur herleiten, wenn 2 Klauseln der Form $\{ V \} \{ \neg V \}$ aus der
	Klauselmenge hergeleitet werden können, d.h., dass es möglich sein muss eine pos. Klausel herzuleiten.
	
	Ich werde nun beweisen, dass es nicht möglich ist eine positive Klausel herzuleiten, wenn man eine Klauselmenge gegeben hat, 
	die keine positiven Klauseln enthält.
	
	\textbf{Annahme: }$\mathcal{C}$ ist eine Klauselmenge ohne positive Klauseln.
	\begin{enumerate}[\text{Fall}: 1]
	\item Es können keine Resolventen gebildet werden \\
	Hier ist nichts zu beweisen.
	
	\item Es können Resolventen gebildet werden \\
	Es existieren 2 Klauseln $\mathcal{C}_1$, $\mathcal{C}_2 \in \mathcal{C}$ 
	mit $\mathcal{C}_1 = \{ X_1,...,X_n \},\mathcal{C}_2 = \{ \neg X_1,...,X_m \} $, sodass:\\
	
	\[R := \mathcal{C}_1 \setminus{ \{ X_1 \} } \cup \mathcal{C}_2 \setminus{ \{ \neg X_1 \} } \]
	\[ \mathcal{C} \overset{res(\mathcal{C}_1,\mathcal{C}_2)}{\equiv} \mathcal{C} \cup \{ R \} \]
	
	Es gilt wegen $\mathcal{C}$ enthält keine pos. Klauseln: 
	\[\exists i \in [1,n].(\neg X_i \in \mathcal{C}_1) \Rightarrow \neg X_i \in R \Rightarrow \text{R ist keine pos. Klausel}\]
	$\Rightarrow \text{ es können keine pos. Klauseln durch das Resolvieren der Klauseln aus $\mathcal{C}$ entstehen}$ \\
	$\Rightarrow \text{ es nicht möglich eine pos. Klausel aus $\mathcal{C}$ herzuleiten}$\\
	\end{enumerate}
	
	Da es notwendig ist eine positive Klausel herzuleiten, damit die leere Klausel hergeleitet werden kann,
	folgt, dass es keine Resolutionswiderlegung für eine Klauselmengel ohne pos. Klauseln gibt. 
	Daraus folgt, wegen dem Resolutionskalkül, dass jede Klauselmengel ohne pos. Klauseln erfüllbar ist.
	
	

	
	 
	\item[(ii)]
	\( \{ \{ \neg Z, Y \},\{ V,Y,Z \},\{ \neg X,V \},\{ \neg V, Y \},\{ \neg Y \} \} 
	\overset{res(\{ \neg Z, Y \},\{ V,Y,Z \})}{\equiv} \mathcal{C} \cup \{\{ Y,V \}\}
	\overset{res(\{ \neg V, Y \},\{ V,Y,Z \})}{\equiv} \\ 
	\mathcal{C} \cup \{\{ Y,V \}\} \cup \{\{ Y,Z \}\}
	\overset{res(\{ \neg Z, Y \},\{ Y,Z \})}{\equiv} \mathcal{C} \cup \{\{ Y,V \}\} \cup \{\{ Y,Z \}\} \cup \{\{ Y \}\}
	\overset{res(\{ \neg Y \},\{ Y \})}{\equiv} 
	\mathcal{C} \cup \{\{ Y,V \}\} \cup \{\{ Y,Z \}\} \cup \{\{ Y \}\} \cup \{\emptyset\} \)
	
	\ \\Aus dem Resolutionskalkül folgt, dass $\mathcal{C}$ nicht erfüllbar ist, da es die leere Klausel enthält.
	
	\item[(iii)]
	Die P-Resolution muss korrekt sein, da alle von der P-Resolution gebildeten Resolventen auch mit der normalen
	Resolution gebildet werden können.
	Wenn also die leere Klausel mit der P-Resolution gebildet werden kann, dann kann man mit der normalen Resolution
	ebenfalls die leere Klausel herleiten.
	Aus der VL wissen wir, dass wenn man mit der Resolution die leere Klausel herleiten kann, so ist die 
	Klauselmenge unerfüllbar.
	Daraus folgt unweigerlich, dass die Herleitung der leeren Klausel mit der P-Resolution, bedeutet, dass die
	Klauselmenge unerfüllbar ist.	
	\end{enumerate}
	
	\subsection*{Aufgabe 3}
	
	\textbf{Rückrichtung:} \\
	Den Kanten $\{ u,v \} \in E$ werden die Variablen $X_{u,v,col}$ und $col \in \{0,...,3\}$ zugeordnet.
	\[ \textsf{proper-colour} = 
	 \left\{ \bigvee\limits_{col \in \{0,...,3\}} (X_{u,v,col} \wedge \bigwedge\limits_{\overline{col} \in \{0,...,3\}\setminus{ \{ col \} } } 
	\neg X_{u,v,\overline{col}}) \mid u,v \in V \right\} \] 	
	
	Die Funktion \textsf{proper-colour} stellt sicher, dass jeder Kante nur eine Farbe zugewiesen wird.
	
	\[\textsf{different-colour} = 
	 \{ X_{u,v,col} \wedge \neg X_{u',v',col}\mid
	\{ u,v \} \cap \{ u',v' \} \neq \emptyset \wedge 
	\{ u,v \} \neq \{ u',v' \} \wedge 
	c(\{ u,v \}) = col \wedge 
	\{u,v\},\{u',v'\} \in E \} \] 
	\[\Phi = \textsf{proper-colour } \cup \textsf{ different-colour} \]
	Sei $\Phi_0 \subset \Phi$ endlich. 
	\[E' = \{ \{u,v\} \in E~|~\text{es existiert ein col } \in \{0,...,3\}, \text{sodass } X_{u,v,col} \in var(\Phi_0) \}\] 
	Der von $E'$ induzierte Untergraph ist endlich, da var($\Phi_0$) endlich ist. Daraus folgt nach Annahme, dass der Untergraph 4-kantenfärbbar ist.\\
	Sei $f$ definiert als:
	\[f: E' \rightarrow \{0,...,3\} \text{ eine 4-Kantenfärbung von } G' |_{E'}\]
	Und $\beta$ wie folgt definiert: \\
	\[\beta(X_{u,v,f(\{ u,v \})}) = 1,~\beta(X_{u,v,col}) = 0 \text{ mit } f(\{ u,v \}) \neq col \] 

	Somit erfüllt $\beta~~\Phi_0$.\\
	Nach dem Kompaktheitssatz gilt nun, dass $\Phi$ ebenfalls erfüllbar ist, da alle endlichen Teilmengen $\Phi_0$ erfüllbar sind. Daraus folgt auch, dass G 4-kantenfärbbar ist, wenn es alle endlichen Untergraphen sind.
	
	\ \\ \textbf{Hinrichtung:} \\
	Die Hinrichtung ist trivial.
	
	\subsection*{Aufgabe 4}
	\begin{enumerate}
	\item[(i)]
		\begin{enumerate}
		\item[a)]
		\(\varphi = \bigwedge\limits_{0 \leq i < j \leq \infty} X_{i,j} \)
		\\
		\\
		\\
		\underline{Diese Formel wird wahr, wenn G ein vollständiger Graph ist:}\\
		\\
		Für einen vollständigen Graphen gilt, dass alle $\{i,j\}$ mit i,j $\in \mathbb{N}$ in E enthalten sein müssen, \\
		da es sonst kein vollständiger Graph wäre. \\
		Aus der Definition des $\beta_G$ folgt dann sofort, dass alle $X_{i,j}$ wahr sein müssen und damit diese Formel erfüllen. \\
		\\
		\underline{G ist ein vollständiger Graph, wenn diese Formel wahr ist:}\\
		\\
		Wenn diese Formel wahr ist gilt, dass $\beta_G (X_{i,j})$ = 1 für alle i,j $\in \mathbb{N}$ woraus,\\
		nach Definition von $\beta_G$, folgt, dass auch alle i,j $\{i,j\} \in E$ gilt. \\
		Es besteht also zwischen jedem Knoten i und j eine Verbindung, sodass G ein vollständiger Graph ist.
		
		\item[b)]
		\( \psi = \bigwedge\limits_{0 \leq i < j < k \leq \infty}(X_{i,j} \wedge X_{j,k} \Rightarrow X_{i,k}) \)
		\\
		\\
		\\
		\underline{Diese Formel wird wahr, wenn G transitiv ist:}\\
		\\
		Aus der Transitivität von G folgt, dass für alle Kanten $\{i,j\}, \{j,k\} \in E$ gilt, \\
		dass $\{i,k\} \in E$ mit i,j,k $\in \mathbb{N}$. \\
		Es gilt also, dass für alle $X_{i,j}, X_{j,k}$ auch ein $X_{i,k}$ existiert, sodass $\beta_G (X_{i,k}) = 1$. \\
		Daraus folgt, dass $X_{i,j} \wedge X_{j,k} \Rightarrow X_{i,k}$ immer erfüllt wird. \\
		Ist $\{i,j\}$oder $\{j,k\}$ nicht aus E, dann wird die Implikation auch erfüllt.
		\\
		\\
		\underline{G ist transitiv, wenn diese Formel wahr ist:}\\
		\\
		Fall 1: $\{i,j\},\{j,k\} \in E$:
		\\
		Da die Formel wahr ist, muss gelten, dass $\beta_G(X_{i,k}) = 1$ gilt.\\
		Somit existiert die Kante $\{i,k\}$ auch im Graphen G und die Transitivität ist erfüllt.\\
		\\
		Sonst:
		\\
		Hier ist nichts zu zeigen.
		
		
		\end{enumerate}
	\item[(ii)]
	Wir konstruieren folgende Menge: 
	$$ \Phi = \{X_1,X_2...\}$$
	$$\psi = \bigwedge_{i=1}^{\infty} X_i $$
	Nach dem 2. Punkt des Kompaktheitssatzes muss gelten: $\Phi \models \psi \Leftrightarrow \Phi_0 \models \psi$, wobei $\Phi_0$ eine endliche Teilmenge von $\Phi$ ist.

	Hier erfüllt $\Phi$ offensichtlich $\psi$, wobei es keine endliche Teilmenge $\Phi_0$ von $\Phi$ gibt, die $\psi$ erfüllt. 

	Das ist ein Widerspruch zum Kompaktheitssatz.
 
	\end{enumerate}
\end{document}