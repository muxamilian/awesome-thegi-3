\documentclass[a4paper,10pt]{article}
\usepackage[utf8]{inputenc}
\usepackage{amsmath}
\usepackage{amsfonts}
\usepackage{amssymb}
\usepackage{fullpage}
\usepackage[german]{babel}
\setlength{\parindent}{0cm}
\usepackage{setspace}
\usepackage{mathpazo}
\usepackage{graphicx}
\usepackage{wasysym} 
\usepackage{booktabs}
\usepackage{verbatim}
\usepackage{pst-all}
\usepackage{enumerate}
\usepackage{pstricks}
\usepackage{ulem}
\usepackage{ stmaryrd }
\usepackage[a4paper,
left=1.8cm, right=1.8cm,
top=2.0cm, bottom=2.0cm]{geometry}
\newcommand{\N}{\mathbb{N}}
\newcommand{\A}{\mathcal{A}}
\newcommand{\ts}{\textsf}

\usepackage{tabularx}

\newcolumntype{L}[1]{>{\raggedright\arraybackslash}p{#1}}
\newcolumntype{C}[1]{>{\centering\arraybackslash}p{#1}}
\newcolumntype{R}[1]{>{\raggedleft\arraybackslash}p{#1}}


\begin{document}

\begin{center}
\Large{Theoretische Grundlagen der Informatik 3: Hausaufgabenabgabe 4} \\
\large{Tutorium: Sebastian , Mi 14.00 - 16.00 Uhr}
\end{center}
\begin{tabbing}
Tom Nick \hspace{2cm}\= - 340528\\
Maximillian Bachl \> - 341455 \\
Marius Liwotto\> -  341051
\end{tabbing}
	\subsection*{Aufgabe 1}
	\subsection*{Aufgabe 2}
	\begin{enumerate}
	\item[(i)]
	Damit eine Klauselmenge nicht erfüllbar ist, muss eine Resolutionswiderlegung geben,\\
	d.h. man muss aus der Klauselmenge die leere Klausel herleiten können.\\
	Die leere Klausel lässt sich wiederum  nur herleiten, wenn 2 Klauseln der Form $\{ V \} \{ \neg V \}$ aus der\\
	Klauselmenge hergeleitet werden können, d.h., dass es möglich sein muss eine pos. Klausel herzuleiten.
	
	\ \\Ich werde nun beweisen, dass es nicht möglich ist eine positive Klausel herzuleiten, wenn man eine Klauselmenge gegeben hat, 
	die keine positiven Klauseln enthält.
	
	\ \\$\mathcal{C}$ ist eine Klauselmenge ohne positive Klauseln.
	
	\ \\Fall 1: Es können keine Resolventen gebildet werden \\
	Hier ist nichts zu beweisen.\\
	\\
	Fall 2: Es können Resolventen gebildet werden \\
	Es existieren 2 Klauseln $\mathcal{C}_1$, $\mathcal{C}_2 \in \mathcal{C}$ 
	mit $\mathcal{C}_1 = \{ X_1,...,X_n \},\mathcal{C}_2 = \{ \neg X_1,...,X_m \} $, sodass:\\
	\\
	\(R := \mathcal{C}_1 \setminus{ \{ X_1 \} } \cup \mathcal{C}_2 \setminus{ \{ \neg X_1 \} } \) \\
	\\
	\( \mathcal{C} \overset{res(\mathcal{C}_1,\mathcal{C}_2)}{\equiv} \mathcal{C} \cup \{ R \} \\
	\\
	\text{Es gilt wegen $\mathcal{C}$ enthält keine pos. Klauseln: }\\
	(\exists i \in [1,n]. \neg X_i \in \mathcal{C}_1) \Rightarrow \neg X_i \in R \Rightarrow \text{R ist keine pos. Klausel}\\
	\Rightarrow \text{ es können keine pos. Klauseln durch das Resolvieren der Klauseln aus $\mathcal{C}$ entstehen} \\
	\Rightarrow \text{ es nicht möglich eine pos. Klausel aus $\mathcal{C}$ herzuleiten} \)\\
	\\
	Da es notwendig ist eine positive Klausel herzuleiten, damit die leere Klausel hergeleitet werden kann, \\
	folgt, dass es keine Resolutionswiderlegung für eine Klauselmengel mit nur pos. Klauseln gibt. \\
	Daraus folgt, wegen dem Resolutionskalkül, dass jede Klauselmengel mit nur pos. Klauseln erfüllbar ist.
	
	

	
	 
	\item[(ii)]
	\( \{ \{ \neg Z, Y \},\{ V,Y,Z \},\{ \neg X,V \},\{ \neg V, Y \},\{ \neg Y \} \} 
	\overset{res(\{ \neg Z, Y \},\{ V,Y,Z \})}{\equiv} \mathcal{C} \cup \{\{ Y,V \}\}
	\overset{res(\{ \neg V, Y \},\{ V,Y,Z \})}{\equiv} \\ 
	\mathcal{C} \cup \{\{ Y,V \}\} \cup \{\{ Y,Z \}\}
	\overset{res(\{ \neg Z, Y \},\{ Y,Z \})}{\equiv} \mathcal{C} \cup \{\{ Y,V \}\} \cup \{\{ Y,Z \}\} \cup \{\{ Y \}\}
	\overset{res(\{ \neg Y \},\{ Y \})}{\equiv} 
	\mathcal{C} \cup \{\{ Y,V \}\} \cup \{\{ Y,Z \}\} \cup \{\{ Y \}\} \cup \{\emptyset\} \)
	
	\ \\Aus dem Resolutionskalkül folgt, dass $\mathcal{C}$ nicht erfüllbar ist, da es die leere Klausel enthält.
	
	\item[(iii)]
	Die P-Resolution muss korrekt sein, da alle von der P-Resolution gebildeten Resolventen auch mit der normalen
	Resolution gebildet werden können.\\
	Wenn also die leere Klausel mit der P-Resolution gebildet werden kann, dann kann man mit der normalen Resolution
	ebenfalls die leere Klausel herleiten.\\
	Aus der VL wissen wir, dass wenn man mit der Resolution die leere Klausel herleiten kann, so ist die \\
	Klauselmenge unerfüllbar.\\
	Daraus folgt unweigerlich, dass die Herleitung der leeren Klausel mit der P-Resolution, bedeutet, dass die\\
	Klauselmenge unerfüllbar ist.	
	\end{enumerate}
	


\end{document}