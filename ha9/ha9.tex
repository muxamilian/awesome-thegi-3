\documentclass[a4paper,10pt]{article}
\usepackage[utf8]{inputenc}
\usepackage{amsmath}
\usepackage{amsfonts}
\usepackage{amssymb}
\usepackage[german]{babel}
\setlength{\parindent}{0cm}
\usepackage{setspace}
\usepackage{mathpazo}
\usepackage{graphicx}
\usepackage{wasysym} 
\usepackage{booktabs}
\usepackage{verbatim}
\usepackage{enumerate}
\usepackage{hyperref}
\usepackage{ulem}
\usepackage{stmaryrd }
\usepackage[a4paper,
left=1.8cm, right=1.8cm,
top=2.0cm, bottom=2.0cm]{geometry}
\usepackage{tabularx}

\newcommand{\rf}{\right\rfloor}
\newcommand{\lf}{\left\lfloor}
\newcommand{\tabspace}{15cm}
\newcommand{\N}{\mathbb{N}}
\newcommand{\Z}{\mathbb{Z}}

\begin{document}
\begin{center}
\Large{Theoretische Grundlagen der Informatik 3: Hausaufgabenabgabe 9} \\
\large{Tutorium: Sebastian , Mi 14.00 - 16.00 Uhr}
\end{center}
\begin{tabbing}
Tom Nick \hspace{2cm}\= - 340528\\
Maximillian Bachl \> - 341455 \\
Marius Liwotto\> -  341051
\end{tabbing}
\section*{Aufgabe 1}
\begin{enumerate}[(i)]
	\item 
		\begin{align*}
			\varphi_1 &:= \lnot (\exists x \exists y E(x,y) \land \lnot \exists x \forall y \exists z (\lnot E(x,z) \lor f(x,y) = z)) \rightarrow \exists x E(x,f(y,x)) \\
			&\equiv  (\exists x \exists y E(x,y) \land \lnot \exists x \forall y \exists z (\lnot E(x,z) \lor f(x,y) = z)) \lor \exists x E(x,f(y,x)) \\
			&\equiv  (\exists x \exists y E(x,y) \land \forall x \exists y \forall z \lnot(\lnot E(x,z) \lor f(x,y) = z))) \lor \exists x E(x,f(y,x)) \\
			&\equiv  (\exists x \exists y E(x,y) \land \forall x \exists y \forall z (E(x,z) \land \lnot(f(x,y) = z)))) \lor \exists x E(x,f(y,x)) \\
			&\equiv  (\exists x_1 \exists y_2 E(x_1,y_2) \land \forall x_2 \exists y_2 \forall z_1 (E(x_2,z_1) \land \lnot(f(x_2,y_2) = z_1)))) \lor \exists x_3 E(x_3,f(y,x_3)) \\
			&\equiv  \exists x_1 \exists y_2\forall x_2 \exists y_2 \forall z_1\exists x_3((E(x_1,y_2) \land   (E(x_2,z_1) \land \lnot(f(x_2,y_2) = z_1)))) \lor E(x_3,f(y,x_3)))  \\
		\end{align*}
	\item 
		\begin{align*}
			\varphi_2 &:= \exists y \forall z (E(x,z) \land (E(y,z) \rightarrow \forall x(E(f(x,y),z)\land \lnot \forall  y R(x,y) ))) \\
			&\equiv \exists y \forall z (E(x,z) \land (\lnot E(y,z) \lor\forall x(E(f(x,y),z)\land \lnot \forall  y R(x,y) ))) \\
			&\equiv \exists y \forall z (E(x,z) \land (\lnot E(y,z) \lor\forall x(E(f(x,y),z)\land \exists y \lnot R(x,y) ))) \\
			&\equiv \exists y_1 \forall z_1 (E(x_1,z_1) \land (\lnot E(y_1,z_1) \lor \forall x_2 (E(f(x_2,y_1),z_1)\land \exists y_1 \lnot R(x_2,y_1) )) \\
			&\equiv \exists y_1 \forall z_1 \forall x_2 (E(x_1,z_1) \land (\lnot E(y_1,z_1) \lor  (E(f(x_2,y_1),z_1)\land \lnot R(x_2,y_1) ))) \\
		\end{align*}
\end{enumerate}
\section*{Aufgabe 2}
\begin{align*}
\phi_1(\mathcal{N}) &:= \exists~ x~ (y = x + x) \\
 &\phi_1(x) := \exists y(y \cdot y = x) \\
\phi_2(\mathcal{N}) &:= \forall b \forall c \exists d (b \cdot c = a \rightarrow (c = \phi_1(d) + \phi_1(d) \lor b = \phi_1(d) + \phi_1(d))) \\
\phi_3(\mathcal{R}) &:= x = y \cdot y \\
\phi_4(\mathcal{R}) &:= \exists~ m~ \forall~ n~ (m \cdot n = m \wedge m = x + y) \\
\phi_5(\mathcal{R}) &:= \exists~ m~ \exists~ n~ (n \cdot n = m \wedge y = x + m) \\
\phi_6(\mathcal{R}) &:= (u'' = u \cdot u' - v \cdot v') \wedge (v'' = u' \cdot v + u \cdot v') \\
\end{align*}

\section*{Aufgabe 3}
\begin{enumerate}[(i)]
\item
$\overline{x} := (x_1,x_2,...,x_k)$

$\varphi(\mathcal{B}) = \pi(\varphi(\mathcal{A}))$
$\Leftrightarrow \forall \overline{x}~ (\overline{x} \in \varphi(\mathcal{B})$
$\Leftrightarrow \overline{x} \in \pi(\varphi(\mathcal{A}))$
\\
\\
Da $\pi$ ein Isomorphismus von $\mathcal{A}$ nach $\mathcal{B}$ ist, gilt für alle Relationen R aus $\sigma$: 

(*) $\overline{a} \in R^{\mathcal{A}} \Leftrightarrow \pi(\overline{a}) \in R^{\mathcal{B}}$
\\
\\
\begin{align*}
&\big(\forall \overline{x}~ \big( \overline{x} \in \varphi(\mathcal{B})  \Leftrightarrow \varphi(\overline{x}) =1\big)\big) \\
&\Leftrightarrow \big(\forall \overline{x}~ \big( \overline{x} \in \varphi(\mathcal{B})  \overset{(*)}{\Leftrightarrow} \varphi(\pi^{-1}(x_1),...,\pi^{-1}(x_k)) = 1\big)\big) \\
&\Leftrightarrow \big(\forall \overline{x}~ \big( \overline{x} \in \varphi(\mathcal{B})  \Leftrightarrow (\pi^{-1}(x_1),...,\pi^{-1}(x_k)) \in \varphi(\mathcal{A}))\big) \\ 
&\Leftrightarrow \big(\forall \overline{x}~ \big( \overline{x} \in \varphi(\mathcal{B})  \Leftrightarrow \overline{x} \in \pi(\varphi(\mathcal{A}) \big)\big)\big) \\
&\Leftrightarrow \varphi(\mathcal{B}) = \pi(\varphi(\mathcal{A}))
\end{align*}


\item
Die gegebene Struktur enthält nur die Relation $<$ aber keine Funktionssymbole. 
$<$ ist über $\mathbb{Z}$ eine Relation ohne Maximum oder Minimum. 

Damit es ein $\varphi$ gibt sodass $\varphi(\mathcal{Z}) = \lbrace 0 \rbrace$, muss es möglich sein, die $0$ von allen anderen Zahlen zu unterscheiden. Durch die Unendlichkeit von $\mathbb{Z}$ ist es nicht möglich durch Quantifikation bestimmte Zahlen zu erkennen:
\begin{itemize}
\item
$\exists x \exists y (x < y)$
\item
$\exists y (x < y)$
\item
$\exists x (x < y)$
\item
$\forall x \forall y (x < y)$
\item
$\forall y (x < y)$
\item
$\forall x (x < y)$
\item
$x < y$
\item
$\exists x \forall y (x < y)$
\item
$\forall x \exists y (x < y)$

In jedem der Fälle gibt es unendlich viele Variablen, die die Gleichung erfüllen. Die mehrfache Verwendung von $<$ hilft nicht weiter, da die Menge an erfüllenden Werten immer unendlich groß bleibt. 

Durch die Abwesenheit von Funktionssymbolen ist es somit nicht möglich, eine Formel $\varphi$ aufzustellen, die die gegebenen Vorraussetzungen erfüllt.
\end{itemize}
\end{enumerate}

\section*{Aufgabe 4}
Es gibt folgende Fälle:
\begin{enumerate}
\item{$q = 0$}\\
Dann gilt die Aussage schon nach dem Hinweis des Aufgabenblattes.
\item{$q \le 1$}\\
Sei $\varphi$ die Formel ohne freie Variablen mit maximal $q$ Quantoren. Dann gibt es eine zu $\varphi$ nach Theorem 4.34 der Folien eine äquivalente Formel $\varphi'$ in Pränexnormalform. Also gilt $\varphi \equiv \varphi'$. Es reicht also zu zeigen, dass es nur endlich viele Formeln in Pränexnormalform gibt.
\\\\
Diese Formel $\varphi'$ hat dann die Form $Q_1x_1...Q_px_p\ \psi$, mit $1 \le p \le q$ wobei $\psi$ eine Formel der Aussagenlogik ist, in der keine freien Variablen vorhanden sind. 
\\\\
Somit kann nach der Annahme des Aufgabenblattes $\psi$ nur eine von endlich vielen Formeln sein.  Die Reihenfolge der Quantoren $Q_1x_1...Q_px_p\ \psi$ spielt für den Wahrheitswert von $\varphi$ keine Rolle. 
\\\
Somit gibt es bis auf logische Äquivalenz nur endlich viele Formeln mit weniger als $q$ Quantoren.
\end{enumerate}
\end{document}



