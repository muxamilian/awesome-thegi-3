\documentclass[a4paper,10pt]{article}
\usepackage[utf8]{inputenc}
\usepackage{amsmath}
\usepackage{amsfonts}
\usepackage{amssymb}
\usepackage[german]{babel}
\setlength{\parindent}{0cm}
\usepackage{setspace}
\usepackage{mathpazo}
\usepackage{graphicx}
\usepackage{wasysym} 
\usepackage{booktabs}
\usepackage{verbatim}
\usepackage{enumerate}
\usepackage{hyperref}
\usepackage{ulem}
\usepackage{stmaryrd }
\usepackage[a4paper,
left=1.8cm, right=1.8cm,
top=2.0cm, bottom=2.0cm]{geometry}
\usepackage{tabularx}

\newcommand{\rf}{\right\rfloor}
\newcommand{\lf}{\left\lfloor}
\newcommand{\tabspace}{15cm}
\newcommand{\N}{\mathbb{N}}
\newcommand{\Z}{\mathbb{Z}}

\begin{document}
\begin{center}
\Large{Theoretische Grundlagen der Informatik 3: Hausaufgabenabgabe 9} \\
\large{Tutorium: Sebastian , Mi 14.00 - 16.00 Uhr}
\end{center}
\begin{tabbing}
Tom Nick \hspace{2cm}\= - 340528\\
Maximillian Bachl \> - 341455 \\
Marius Liwotto\> -  341051
\end{tabbing}
\subsection*{Aufgabe 1}
\begin{enumerate}[(i)]
	\item 
		\begin{align*}
			\varphi_1 &:= \lnot (\exists x \exists y E(x,y) \land \lnot \exists x \forall y \exists z (\lnot E(x,z) \lor f(x,y) = z)) \rightarrow \exists x E(x,f(y,x)) \\
			&\equiv  (\exists x \exists y E(x,y) \land \lnot \exists x \forall y \exists z (\lnot E(x,z) \lor f(x,y) = z)) \lor \exists x E(x,f(y,x)) \\
			&\equiv  (\exists x \exists y E(x,y) \land \forall x \exists y \forall z \lnot(\lnot E(x,z) \lor f(x,y) = z))) \lor \exists x E(x,f(y,x)) \\
			&\equiv  (\exists x \exists y E(x,y) \land \forall x \exists y \forall z (E(x,z) \land \lnot(f(x,y) = z)))) \lor \exists x E(x,f(y,x)) \text{ in NNF} \\
			&\equiv  (\exists x_1 \exists y_2 E(x_1,y_2) \land \forall x_2 \exists y_2 \forall z_1 (E(x_2,z_1) \land \lnot(f(x_2,y_2) = z_1)))) \lor \exists x_3 E(x_3,f(y,x_3)) \\
			&\equiv  \exists x_1 \exists y_2\forall x_2 \exists y_2 \forall z_1\exists x_3((E(x_1,y_2) \land   (E(x_2,z_1) \land \lnot(f(x_2,y_2) = z_1)))) \lor E(x_3,f(y,x_3))) \text{ in PN} \\
		\end{align*}
	\item \[ \]
\end{enumerate}
\subsection*{Aufgabe 2}
\begin{align*}
\phi_1(\mathcal{N}) &:= \exists~ x~ (y = x + x) \\
\phi_2(\mathcal{N}) &:= \exists~ x~ (y = x \cdot x) \\
\phi_3(\mathcal{R}) &:= x = y \cdot y \\
\phi_4(\mathcal{R}) &:= \exists~ m~ \forall~ n~ (m \cdot n = m \wedge m = x + y) \\
\phi_5(\mathcal{R}) &:= \exists~ m~ \exists~ n~ (n \cdot n = m \wedge y = x + m) \\
\phi_6(\mathcal{R}) &:= (u'' = u \cdot u' - v \cdot v') \wedge (v'' = u' \cdot v + u \cdot v') \\
\end{align*}
\end{document}