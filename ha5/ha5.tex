\documentclass[a4paper,10pt]{article}
\usepackage[utf8]{inputenc}
\usepackage{amsmath}
\usepackage{amsfonts}
\usepackage{amssymb}
\usepackage[german]{babel}
\setlength{\parindent}{0cm}
\usepackage{setspace}
\usepackage{mathpazo}
\usepackage{graphicx}
\usepackage{wasysym} 
\usepackage{booktabs}
\usepackage{verbatim}
\usepackage{enumerate}
\usepackage{ulem}
\usepackage{stmaryrd }
\usepackage[a4paper,
left=1.8cm, right=1.8cm,
top=2.0cm, bottom=2.0cm]{geometry}
\usepackage{tabularx}

\begin{document}

\begin{center}
\Large{Theoretische Grundlagen der Informatik 3: Hausaufgabenabgabe 5} \\
\large{Tutorium: Sebastian , Mi 14.00 - 16.00 Uhr}
\end{center}
\begin{tabbing}
Tom Nick \hspace{2cm}\= - 340528\\
Maximillian Bachl \> - 341455 \\
Marius Liwotto\> -  341051
\end{tabbing}
\subsection*{Aufgabe 1}
\begin{enumerate}[(i)]
\item 	Die wohl einfachste Methode um eine Formel in eine KNF umzuformen ist per Wahrheitstabelle:
	\[
	\left.\begin{array}{cccc|c}
		\toprule 
		A & B & C & D & \varphi\\
		\midrule
		0 & 0 & 0 & 0 & 1\\
		0 & 0 & 0 & 1 & 1\\
		0 & 0 & 1 & 0 & 1\\
		0 & 0 & 1 & 1 & 1\\
		0 & 1 & 0 & 0 & 1\\
		0 & 1 & 0 & 1 & 1\\
		0 & 1 & 1 & 0 & 1\\
		0 & 1 & 1 & 1 & 1\\
		1 & 0 & 0 & 0 & 1\\
		1 & 0 & 0 & 1 & 1\\
		1 & 0 & 1 & 0 & 0\\
		1 & 0 & 1 & 1 & 1\\
		1 & 1 & 0 & 0 & 0\\
		1 & 1 & 0 & 1 & 1\\
		1 & 1 & 1 & 0 & 1\\
		1 & 1 & 1 & 1 & 1\\
		\bottomrule
		\end{array} \quad \right\} ( \lnot A \lor  \lnot B \lor C \lor D) \land (\lnot A \lor B \lor \lnot C \lor D) = \varphi'
	 \]
\item  	\begin{enumerate}[$\quad$ 1.]
		\item  	Wegen Assoziativgesetz gilt: \[\varphi' \equiv  ( (\lnot A \lor  \lnot B) \lor (C \lor D)) \land ((\lnot A \lor B) \lor (\lnot C \lor D)) \]
		\item  	Füge für jede Klausel der Form $(X \lor Y)$ eine Variable $Z_i$ ein und bilde Klauseln der Form:
			\[((X \lor Y) \leftrightarrow Z_i) \] 
			Falls man diese Klauseln verundet und zusätzlich $Z_0$ mit dem Term verundet, entsteht ein Erfüllbarkeits\-äquivalenter Term (Siehe Schöning S. 160), also gilt:
			\begin{align*} \varphi' \equiv   &Z_0 \land (Z_0 \leftrightarrow (Z_1 \lor Z_2)) 
			\land (Z_1 \leftrightarrow (Z_3 \lor Z_4)) \\
			&\land (Z_2 \leftrightarrow (Z_5 \lor Z_6)) 
			\land (Z_3 \leftrightarrow (\lnot A \lor  \lnot B)) \\
			&\land (Z_4 \leftrightarrow (C \lor D))) 
			\land (Z_5 \leftrightarrow(\lnot A \lor B)) 
			\land (Z_6\leftrightarrow(\lnot C \lor D))
			\end{align*}
		\item 	Es gilt:
			\[Z \leftrightarrow (X \lor Y) \equiv (Z \lor \lnot X) \land (\lnot Z \lor X \lor Y) \land (Z \lor \lnot Y)\]
			Also gilt:
			\begin{align*} \varphi' \equiv   &Z_0 \land ((\lnot Z_0 \lor Z_1 \lor Z_2) \land (Z_0 \lor \lnot Z_1) \land (Z_0 \lor \lnot Z_2)) 
			\land ((\lnot Z_1 \lor Z_3 \lor Z_4) \land (Z_1 \lor \lnot Z_3) \land (Z_1 \lor \lnot Z_4)) \\
			&\land ((\lnot Z_2 \lor Z_5 \lor Z_6) \land (Z_2 \lor \lnot Z_5) \land (Z_2 \lor \lnot Z_6)) 
			\land ((\lnot Z_3 \lor\lnot A \lor  \lnot B) \land (Z_3 \lor A) \land (Z_3 \lor B)) \\
			&\land (( \lnot Z_4 \lor C \lor D) \land (Z_4 \lor \lnot C) (Z_4 \lor \lnot D)) 
			\land ((\lnot Z_5 \lor \lnot A \lor B) \land (Z_5 \lor A) \land (Z_5 \lor \lnot B)) \\
			&\land ((\lnot Z_6\lor \lnot C \lor D) \land (Z_6 \lor C) \land (Z_6 \lor \lnot D))
			\end{align*}
		\item 	Wegen Assosiativgesetz gilt:
			\begin{align*} \varphi''    &Z_0 \land (\lnot Z_0 \lor Z_1 \lor Z_2) \land (Z_0 \lor \lnot Z_1) \land (Z_0 \lor \lnot Z_2) 
			\land (\lnot Z_1 \lor Z_3 \lor Z_4) \land (Z_1 \lor \lnot Z_3) \land (Z_1 \lor \lnot Z_4) \\
			&\land (\lnot Z_2 \lor Z_5 \lor Z_6) \land (Z_2 \lor \lnot Z_5) \land (Z_2 \lor \lnot Z_6))
			\land (\lnot Z_3 \lor\lnot A \lor  \lnot B) \land (Z_3 \lor A) \land (Z_3 \lor B) \\
			&\land ( \lnot Z_4 \lor C \lor D) \land (Z_4 \lor \lnot C) (Z_4 \lor \lnot D)
			\land (\lnot Z_5 \lor \lnot A \lor B) \land (Z_5 \lor A) \land (Z_5 \lor \lnot B) \\
			&\land (\lnot Z_6\lor \lnot C \lor D) \land (Z_6 \lor C) \land (Z_6 \lor \lnot D)
			\end{align*}
	\end{enumerate}
 \end{enumerate}
\end{document}