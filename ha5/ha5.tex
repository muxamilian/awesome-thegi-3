\documentclass[a4paper,10pt]{article}
\usepackage[utf8]{inputenc}
\usepackage{amsmath}
\usepackage{amsfonts}
\usepackage{amssymb}
\usepackage[german]{babel}
\setlength{\parindent}{0cm}
\usepackage{setspace}
\usepackage{mathpazo}
\usepackage{graphicx}
\usepackage{wasysym} 
\usepackage{booktabs}
\usepackage{verbatim}
\usepackage{enumerate}
\usepackage{hyperref}
\usepackage{ulem}
\usepackage{stmaryrd }
\usepackage[a4paper,
left=1.8cm, right=1.8cm,
top=2.0cm, bottom=2.0cm]{geometry}
\usepackage{tabularx}
\begin{document}

\begin{center}
\Large{Theoretische Grundlagen der Informatik 3: Hausaufgabenabgabe 5} \\
\large{Tutorium: Sebastian , Mi 14.00 - 16.00 Uhr}
\end{center}
\begin{tabbing}
Tom Nick \hspace{2cm}\= - 340528\\
Maximillian Bachl \> - 341455 \\
Marius Liwotto\> -  341051
\end{tabbing}
\subsection*{Aufgabe 1}
\begin{enumerate}[(i)]
\item 	Die wohl einfachste Methode um eine Formel in eine KNF umzuformen ist per Wahrheitstabelle:
	\[
	\left.\begin{array}{cccc|c}
		\toprule 
		A & B & C & D & \varphi\\
		\midrule
		0 & 0 & 0 & 0 & 1\\
		0 & 0 & 0 & 1 & 1\\
		0 & 0 & 1 & 0 & 1\\
		0 & 0 & 1 & 1 & 1\\
		0 & 1 & 0 & 0 & 1\\
		0 & 1 & 0 & 1 & 1\\
		0 & 1 & 1 & 0 & 1\\
		0 & 1 & 1 & 1 & 1\\
		1 & 0 & 0 & 0 & 1\\
		1 & 0 & 0 & 1 & 1\\
		1 & 0 & 1 & 0 & 0\\
		1 & 0 & 1 & 1 & 1\\
		1 & 1 & 0 & 0 & 0\\
		1 & 1 & 0 & 1 & 1\\
		1 & 1 & 1 & 0 & 1\\
		1 & 1 & 1 & 1 & 1\\
		\bottomrule
		\end{array} \quad \right\} ( \lnot A \lor  \lnot B \lor C \lor D) \land (\lnot A \lor B \lor \lnot C \lor D) = \varphi'
	 \]
\item
	Wir wählen $\varphi'' = A$. $\varphi''$ ist erfüllbarkeitsäquivalent zu $\varphi'$, da beide Formeln erfüllbar sind. Für die Belegung $\gamma(A) = 1$ gilt $\llbracket \varphi'' \rrbracket^{\gamma} = \top$. 
 \end{enumerate}
\section*{Aufgabe 2}
	Man könnte das ganze auch als rekursive Formel angeben, wenn mir langweilig ist, 		mache ich das vllt (Dadurch koennte man einen schoenen Induktionsbeweis fuehren). 
	Ansonsten erstmal eine Rechenvorschrift.

	Sei $\varphi$ eine beliebige aussagenlogische Formel, um sie zu einer erf\"ullbarkeits
	\"aquivalenten 3KNF umzuformen, befolge folgende Rechenanleitung:
	\begin{enumerate}
		\item 	Mit Hilfe von deMorgan's Regel werden wir Formeln der Form 
			$\lnot(\xi)$ mit $\xi \in AL\setminus VAR$ umformen, sodass alle 
			Negationszeichen an den Variablen stehen. 
			Dieser Vorgang dauert so lange wie die Formel lang ist, kann also mit
			linear abgesch\"atzt werden.
		\item   F\"ur jeden Junktor ausser der Negation f\"uhren wir eine 
			Variable $Y_x$ ein. Wobei der Wurzelknoten der Formel, also falls 
			$\varphi = (\xi' \bullet \xi'') \text{ mit } \bullet \in \{\lor, \land\}$, 
			würde $\bullet$ die Variable 
			$Y_0$ erhalten. Mithilfe der Varibale wird jeder solcher Teilformel 
			eine neue Teilformel zugeordnet: 
			$Y_x \leftrightarrow (\xi' \bullet \xi'')$
			Alle diese Teilformeln + der Variablen des Wurzelknotens ($Y_0$) werden mit einem $\land$ verkn\"upft. Diese 
			Formel definieren wir als $\Psi$. 
			$\Psi$ ist erfüllbarkeits\"aquivalent zu 
			$\varphi$. Von einer erfüllenden Belegung f\"ur $\varphi$ erh\"alt 
			man eine erfüllende Belegung f\"ur $\Psi$ indem man die Belegung 
			\"ubernimmt und die $Y$-Variablen mit den an den jeweiligen Junktoren 
			resultierenden Wahrheitswerten belegt. Eine erf\"ullende Belegung von 
			$\Psi$ ist auch trivialerweise auch eine f\"ur $\varphi$.
		\item	Es gilt: $(Y_x \leftrightarrow (\xi' \bullet \xi'')) 
			\leftrightarrow (Y_x \lor \lnot \xi') \land (Y_x \lor \lnot \xi'') 				\land (\lnot Y_x \lor \xi' \lor \xi'')$
Die Rechenanleitung erzeugt verundete Disjunktionsklauseln der gr\"osse h\"ochstes 3, was einer 3KNF entspricht. Alle Umformungsschritte k\"onnen in polynomiellen Aufwand durchgeführt werden.
			
	\end{enumerate}
\section*{Aufgabe 3}
Es ist zu zeigen, dass \((i) \leftrightarrow (ii) \) gilt. D.h es muss gezeigt werden, dass
 	\begin{enumerate}
 		\item 	\( (i) \rightarrow (ii) \)
 		\item 	\( (ii) \rightarrow (i)\)
 	\end{enumerate}
 	Sei $\Phi$ eine erfüllbare Formelmenge.
 	\begin{enumerate}
 		\item 	\( (i) \rightarrow (ii)\) \\
 			\textbf{Annahme: } Für alle Formeln \(\varphi\) mit \( \text{var}(\varphi)  \subseteq \text{var}(\Phi)\) gilt \(\Phi \vDash \varphi  \) oder \(\Phi \vDash \lnot \varphi\) \\
 			Sei $\varphi$ beliebig. Fallunterscheidung über die Anzahl Belegungen von $\Phi$:
 			\begin{enumerate}[$\quad$\text{Fall } 1:]
 				\item 	Es gibt genau eine Belegung $\beta$, sodass $\beta \vDash \Phi$ \\
 					Falls $\beta \vDash \Phi \land \beta \vDash \varphi$ folgt direkt \(\Phi \vDash \varphi  \), da keine andere Belegung zu überprüfen ist. \\
 					Falls $\beta \not \vDash \varphi \land \beta \vDash \Phi$ ist $\varphi$ für die einzige Belegung bei der $\Phi$ erfüllt ist, nicht erfüllt. Also ist $\varphi$ nicht erfüllt wenn $\Phi$ erfüllt, formal: \(\Phi \vDash \lnot \varphi  \)
 				\item 	Es gibt mehr als eine Belegung $\beta$, sodass $\beta \vDash \Phi$ \\
 					Dann gibt es den Fall, dass für ein $\beta$ gilt $\beta \vDash \Phi \land \beta \vDash \varphi$, aber es kann auch noch ein anderes $\beta'$ geben mit $\beta' \vDash \Phi \land \beta' \not \vDash \varphi$. Damit gilt weder $\Phi \vDash \varphi$ noch $\Phi \vDash \varphi$
 			\end{enumerate}
 			Da die Aussage nur gilt, wenn $\Phi$ genau eine erfüllbare Belegung hat, gilt: $(i) \rightarrow (ii)$
 		\item 	\( (ii) \rightarrow (i) \)	\\
 			\textbf{Annahme: } Es existiert genau eine Belegung \(\beta \) von var\((\Phi)\) mit \( \beta \vDash \Phi\)\\
 			Sei $\varphi$ beliebig. Fallunterscheidung über die Anzahl Belegungen von $\Phi$:
 			\begin{enumerate}[$\quad$\text{Fall }1:]
 				\item 	$\beta \vDash \varphi $ \\
 					Folgt direkt \(\Phi \vDash \varphi  \), da keine andere Belegung zu überprüfen ist. 
 				\item  	$\beta \not \vDash \varphi$ \\
 					$\varphi$ ist für die einzige Belegung bei der $\Phi$ erfüllt ist, nicht erfüllt. Also ist $\varphi$ nicht erfüllt wenn $\Phi$ erfüllt, formal: \(\Phi \vDash \lnot\varphi  \)
 			\end{enumerate}
 			Daraus folgt, dass $\Phi \vDash \varphi$ oder $\Phi \vDash \lnot \varphi$, also $(ii) \rightarrow (i)$		
 	\end{enumerate}
 	Aus 1. und 2. folgt: $(i) \leftrightarrow (ii)$

\section*{Aufgabe 4}
\begin{enumerate}[(i)]
\item
$\phi_1$ ist in H-Form.\\
$\phi_2$ ist nicht in H-Form (keine KNF).\\
$\phi_3$ ist nicht in H-Form (mehr als ein positives Literal).\\
\item
% Nach Definition ist $(\beta \sqcap \beta')(X) = 1$ wenn $\beta$ und $\beta'$ von X ebenfalls 1 sind.

% Nach Annahme gilt: 
% $$\beta \models \phi \land \beta' \models \phi$$
Sei $\phi$ eine beliebige Formel in H-Form. Damit $\phi$ erfüllt ist, muss jede Klausel erfüllt werden. 

Annahmen: $\beta \models \phi$ und $\beta' \models \phi$

Es gibt folgende Fälle für den Aufbau einer Klausel $k$:
\begin{enumerate}[(a)]
\item $k = X$\\
	Um k zu erfüllen muss $\beta(X) = 1$ und $\beta'(X) = 1$. Somit ist per Definiton auch $(\beta \sqcap \beta')(X) = 1$ und $(\beta \sqcap \beta')$ erfüllt diese Klausel.
\item $k = \lnot X$\\
	Um k zu erfüllen muss $\beta(X) = 0$ und $\beta'(X) = 0$. Somit ist per Definiton auch $(\beta \sqcap \beta')(X) = 0$ und $(\beta \sqcap \beta')$ erfüllt diese Klausel.
\item $k = \lnot X \lor \lnot Y \lor \lnot Z \lor ...$\\
	Um k zu erfüllen muss es ein $X$ und $Y$ geben, sodass $\beta(X) = 0$ und $\beta(Y) = 0$. $X$ und $Y$ müssen nicht die gleiche Variable sein. Somit ist per Definiton auch $(\beta \sqcap \beta')(X) = 0$ und $(\beta \sqcap \beta')(Y) = 0$ und $(\beta \sqcap \beta')$ erfüllt diese Klausel.
\item $k = X \lor \lnot Y \lor \lnot Z \lor ...$\\
	Um diese Formel zu erfüllen gibt es wiederum 4 Fälle:
	\begin{enumerate}[(1)]
	\item
	$\beta(X) = 1$ und $\beta'(X) = 1$\\
	Dann gilt $(\beta \sqcap \beta')(X) = 1$ und $\beta \sqcap \beta'$ erfüllt die Formel.
	\item
	$\beta(X) = 1$ und $\beta'(Y) = 0$\\
	Dann gilt $(\beta \sqcap \beta')(Y) = 0$ und $\beta \sqcap \beta'$ erfüllt die Formel.
	\item
	$\beta(Y) = 0$ und $\beta'(X) = 1$\\
	Dann gilt $(\beta \sqcap \beta')(Y) = 0$ und $\beta \sqcap \beta'$ erfüllt die Formel.
	\item
	$\beta(Y) = 0$ und $\beta'(Z) = 0$\\
	Dann gilt $(\beta \sqcap \beta')(Y) = 0$ und $\beta \sqcap \beta'$ erfüllt die Formel.
	\end{enumerate}
	$X$ und $Y$ stehen hier für beliebige, in der Klausel negiert vorkommende Variablen.
\end{enumerate}
\item 
Wir wählen $\phi = A \lor B$ als Formel in KNF.

% Die Wahrheitstabelle sieht dann folgendermaßen aus:
% \[
% \begin{array}{ccc}
% 	\toprule 
% 	A & \lor & B\\
% 	\midrule
% 	0 & 0 & 0\\
% 	0 & 1 & 1\\
% 	1 & 1 & 0\\
% 	1 & 1 & 1\\
% 	\bottomrule
% \end{array}
% \]
Formte man $\phi$ in H-Form um, so würde diese ebenfalls nur aus den beiden Variablen $A$ und $B$ bestehen und hätte höchstens 3 verschiedene Klauseln, nämlich $\lnot A \lor \lnot B$, $\lnot A \lor B$, $A \lor \lnot B$, da $A \lor B$ nicht der H-Form genügt. \\
Klauseln, die äquivalent zu $\top$ oder $\bot$ sind, werden ebenfalls nicht betrachtet, da sie trivialerweise nicht äquivalent zu $\phi$ sein können.

Alle Formeln, die sich aus den 3 Klauseln bilden lassen, sind nicht äquivalent zu $\phi$.

\begin{enumerate}[(a)]
\item
$(\lnot A \lor \lnot B) \land (\lnot A \lor B) \not \equiv \phi$
\item
$(\lnot A \lor \lnot B) \land (A \lor \lnot B) \not \equiv \phi$
\item
$(\lnot A \lor B) \land (A \lor \lnot B) \not \equiv \phi$
\end{enumerate}
Die Fälle, die 3 Klauseln jeweils mit sich selbst zu verunden, sind trivialerweise nicht äquivalent zu $\phi$.
Somit gibt es keine äquivalente Formel in H-Form zu $\phi$
\end{enumerate}
\end{document}


