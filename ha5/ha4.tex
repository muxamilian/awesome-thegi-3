\documentclass[a4paper,10pt]{article}
\usepackage[utf8]{inputenc}
\usepackage{amsmath}
\usepackage{amsfonts}
\usepackage{amssymb}
\usepackage[german]{babel}
\setlength{\parindent}{0cm}
\usepackage{setspace}
\usepackage{mathpazo}
\usepackage{graphicx}
\usepackage{wasysym} 
\usepackage{booktabs}
\usepackage{verbatim}
\usepackage{enumerate}
\usepackage{ulem}
\usepackage{stmaryrd }
\usepackage[a4paper,
left=1.8cm, right=1.8cm,
top=2.0cm, bottom=2.0cm]{geometry}
\usepackage{tabularx}

\begin{document}

\begin{center}
\Large{Theoretische Grundlagen der Informatik 3: Hausaufgabenabgabe 5} \\
\large{Tutorium: Sebastian , Mi 14.00 - 16.00 Uhr}
\end{center}
\begin{tabbing}
Tom Nick \hspace{2cm}\= - 340528\\
Maximillian Bachl \> - 341455 \\
Marius Liwotto\> -  341051
\end{tabbing}
\subsection*{Aufgabe 1}
\begin{enumerate}[(i)]
\item Jede aussagenlogische Formel kann in polynomieller Zeit in eine KNF umgeformt werden. Hier in der Aufgabe ist das ganze an $\varphi$ verlangt.
	\begin{enumerate}
	\item Forme die Formel zur NNF um
		\begin{enumerate}
			\item Elimimiere Implikationen
			\item Ziehe Negationen in die Klauseln mithilfe von DeMorgan's. Weiterhin verchnichte dopplte Negationen.
		\end{enumerate}
	\item Standardisiere Variablen

	\end{enumerate}
\end{enumerate}

\end{document}