\documentclass[a4paper,10pt]{article}
\usepackage[utf8]{inputenc}
\usepackage{amsmath}
\usepackage{amsfonts}
\usepackage{amssymb}
\usepackage[german]{babel}
\setlength{\parindent}{0cm}
\usepackage{setspace}
\usepackage{mathpazo}
\usepackage{graphicx}
\usepackage{wasysym} 
\usepackage{booktabs}
\usepackage{verbatim}
\usepackage{enumerate}
\usepackage{hyperref}
\usepackage{ulem}
\usepackage{stmaryrd }
\usepackage[a4paper,
left=1.8cm, right=1.8cm,
top=2.0cm, bottom=2.0cm]{geometry}
\usepackage{tabularx}

\newcommand{\rf}{\right\rfloor}
\newcommand{\lf}{\left\lfloor}
\newcommand{\tabspace}{15cm}
\newcommand{\N}{\mathbb{N}}
\newcommand{\Z}{\mathbb{Z}}

\begin{document}
\begin{center}
\Large{Theoretische Grundlagen der Informatik 3: Hausaufgabenabgabe 11} \\
\large{Tutorium: Sebastian , Mi 14.00 - 16.00 Uhr}
\end{center}
\begin{tabbing}
Tom Nick \hspace{2cm}\= - 340528\\
Maximillian Bachl \> - 341455 \\
Marius Liwotto\> -  341051
\end{tabbing}
\section*{Aufgabe 1}
	\begin{enumerate}[(i)]
		\item 
			\begin{align*}
				\varphi_1 &= &&\forall x~ \neg E(x,x) \wedge \\
				&&&\forall x \forall y~ 
				(x \neq y \Rightarrow E(x,y) \vee E(y,x)) \wedge  \\
				&&&\forall x \forall y \forall z (E(x,y) \wedge E(y,z) \Rightarrow E(x,z))
			\end{align*}
			
		\item
			Widerspruchsannahme:
			
			Es existiert ein $\varphi_2$ mit Quantorenrang m, sodass für jede endliche lineare Ordnung $\mathcal{A} = (A,E^{\mathcal{A}})$ 
			\begin{align*}
				\mathcal{A} \vDash \varphi_2 \text{ genau dann, wenn $|A|$ ungerade ist.}
			\end{align*}
			gilt.
			
			Nehmen wir nun die lineare endliche Ordnung $\mathcal{B}_1 = (B_1,E^{\mathcal{B}_1})$ mit $|B_1| = 2^{m+1} > 2^m$ und $\mathcal{B}_1 			= (B_2,E^{\mathcal{B}_1})$ mit $|B_2| = 2^m + 1 > 2^m$. Es gilt nach dem Satz der Vorlesung, dass die Duplikatorin das EF-Spiel 					gewinnen würde, woraus folgt, dass $\varphi_2$ die beiden Ordnungen nicht unterscheiden könnte. Da aber $|B_1|$ gerade ist und $|					B_2|$ ungerade, ist das ein Widerspruch zur Annahme, woraus folgt, dass die Annahme falsch sein muss. Somit kann 									kein solcher $FO[\sigma]$-Satz $\varphi_2$ existieren. 
		\item
Die Formel lautet folgendermaßen:
\begin{align*}
\varphi_3(x,y) = \lnot \exists z~ 
\end{align*}
		\item
			Widerspruchsannahme: Es existiert ein solcher $FO[\sigma]$-Satz $\varphi_4$. \\
			
			Nach der Teilaufgabe (iii) existiert eine Formel $\varphi_3$, sodass für jeden 				endlichen Graph G = (A,E) gilt:
			\begin{align*}
				\text{Der Graph } (A, \varphi_3(\mathcal{A})) \text{ ist zusammenhängend genau dann, wenn $|A|$ ungerade ist.}
			\end{align*}
			
			Nun gilt nach $\varphi_4$ auch:
			\begin{align*}
				&G' = (A,\varphi_3(\mathcal{A})) \vDash \varphi_4 \text{ genau dann, wenn $|A|$ ungerade ist.} \\
				\equiv%\Leftrightarrow
				&G' = (A,\varphi_3(\mathcal{A})) \nvDash \varphi_4 \text{ genau dann, wenn $|A|$ gerade ist.} 
			\end{align*}
			
			Das ist ein Widerspruch zur Teilaufgabe (ii), woraus folgt, dass die Annahme falsch sein muss, sodass es keinen solchen 							$FO[\sigma]$-Satz $\varphi_4$ geben kann.
	\end{enumerate}
\section*{Aufgabe 2}
Der gesuchte partielle Isomorphismus hat nach Definition folgende Abbildungsvorschrift:
\begin{align*}
a_1 = 1 \mapsto & ~ b_1 = 1 \\
a_2 = 2 \mapsto &  ~ b_2 = 2 \\
a_3 \mapsto & ~ b_3 = x \\
a_4 \mapsto & ~ b_4 = 1 + x \\
a_5 \mapsto & ~ b_5
\end{align*}
Es gilt $\left(b_1,b_3,b_4\right) \in P^{\mathcal B}$, was $1 + x = 1 + x$ entspricht.
Es muss also auch gelten $\left(a_1 = 1,a_3,a_4\right) \in P^{\mathcal A}$. Daraus folgt, dass entweder $a_3$ oder $a_4$ ungerade sind.\\

Fallunterscheidung:
\begin{enumerate}[(a)]
\item $a_3$ ist die gerade Zahl.

Wir wählen als Herausforderer im 3. Zug $a_5 = \frac{a_3}{2}$.
Nun gilt offensichtlich $\left(a_2 = 2,a_5 = \frac{a_3}{2},a_3\right) \in M^{\mathcal A}$.

$\left(b_2 = 2,b_5,b_3 = x\right) \in M^{\mathcal B}$ gilt aber keinesfalls, da es trivialerweise kein Element in $B$ gibt, die mit $2$ multipliziert $x$ ergibt.
\item $a_4$ ist die gerade Zahl.

Wir wählen als Herausforderer im 3. Zug $a_5 = \frac{a_4}{2}$.
Nun gilt offensichtlich $\left(a_2 = 2,a_5 = \frac{a_4}{2},a_4\right) \in M^{\mathcal A}$.

$\left(b_2 = 2,b_5,b_4 = x\right) \in M^{\mathcal B}$ gilt aber keinesfalls, da es trivialerweise kein Element in $B$ gibt, die mit $2$ multipliziert $x + 1$ ergibt.
\end{enumerate}
\section*{Aufgabe 3}
	\begin{enumerate}[(i)]
		\item
			Der Herausforderer spielt in der Struktur $\mathcal{B}$ und wählt $\infty$. Gibt die Duplikatorin das Element $a$ als Antwort, dann 					gilt, dass ein Element $i$ in $\mathcal{A}$ existiert, sodass $a < i$. Daraus folgt aber, dass es ebenfalls ein $P_i^{\mathcal{A}}$ 					gibt, sodass $a \notin P_i^{\mathcal{A}}$, was jedoch ein Widerspruch ist, da $\infty$ in allen einstelligen Relationen aus 						$\mathcal{B}$ vorkommt.
		\item
			Sei $\varphi \in FO[\sigma]$ beliebig.
			
			Für jedes $P_i(x)$ mit $x \in \mathbb{N}$ gilt (*): \\
			\begin{itemize}
				\item[i $>$ 0:] ~\\
					1. wurde x durch einen Existenzquantor quantifiziert, ist $P_i(x) = 1$
				
					Für jedes $P_i$ existiert eine Zahl x, sodass x $>$ i und damit $P_i(x) = 1$. \\
					
					2. wenn x durch einen Allquantor quantifiziert wurde, ist $P_i(x) = 0$
					
					Für jedes $P_i$ existiert eine Zahl x, sodass x $<$ i und damit $P_i(x) = 0$. \\
						
				\item[i = 0:] ~\\
					1. wurde x durch einen Existenzquantor quantifiziert, ist $P_0(x) = 1$
				
					Für $P_0$ existiert eine Zahl x, sodass x $>$ 0 und damit $P_0(x) = 1$. \\
					
					2. wenn x durch einen Allquantor quantifiziert wurde, ist $P_0(x) = 1$
					
					Für $P_0$ existiert keine Zahl x, sodass x $<$ 0 und damit $P_0(x) = 1$. \\
			\end{itemize}
			
			Die Aussagen (*) gelten in beiden Strukturen, woraus folgt, dass alle Relationen gleich auswerten und somit auch der Satz $\varphi$ 				in beiden Strukturen immer gleich auswertet. Da keine Einschränkung bei $\varphi$ getroffen wurde, folgt, dass es keinen 							$FO[\sigma]$-Satz gibt, der beide Strukturen unterscheidet, sodass sie elementar äquivalent sind.

\item
Der Widerspruch rührt daher, dass der Satz von Ehrenfeucht nur für endliche relationale Signaturen und Strukturen gilt. \\

Mit einer endlichen Menge an Relationssymbolen tritt der oben aufgezeigte Widerspruch nicht auf:\\
\textbf{Annahme:} Es gibt nicht unendlich viele $P_i$, sondern $i \le n$, wobei $n \in \mathbb{N}$.\\
Wählt nun der Herausforderer $\infty$, so wählt die Duplikatorin eine Zahl $j$ mit $j > m + n$, wobei $n$ die gleiche Zahl wie in der obigen Zeile ist.
Der Rest des Spiels ist trivial.

Somit sind die beider Strukturen nun elementar äquivalent und die Duplikatorin gewinnt das Ehrenfeucht-Fraïsse-Spiel.


	\end{enumerate}
\end{document}