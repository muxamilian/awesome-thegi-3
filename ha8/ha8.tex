\documentclass[a4paper,10pt]{article}
\usepackage[utf8]{inputenc}
\usepackage{amsmath}
\usepackage{amsfonts}
\usepackage{amssymb}
\usepackage[german]{babel}
\setlength{\parindent}{0cm}
\usepackage{setspace}
\usepackage{mathpazo}
\usepackage{graphicx}
\usepackage{wasysym} 
\usepackage{booktabs}
\usepackage{verbatim}
\usepackage{enumerate}
\usepackage{hyperref}
\usepackage{ulem}
\usepackage{stmaryrd }
\usepackage[a4paper,
left=1.8cm, right=1.8cm,
top=2.0cm, bottom=2.0cm]{geometry}
\usepackage{tabularx}

\newcommand{\rf}{\right\rfloor}
\newcommand{\lf}{\left\lfloor}
\newcommand{\tabspace}{15cm}
\newcommand{\N}{\mathbb{N}}
\newcommand{\Z}{\mathbb{Z}}

\begin{document}
\begin{center}
\Large{Theoretische Grundlagen der Informatik 3: Hausaufgabenabgabe 8} \\
\large{Tutorium: Sebastian , Mi 14.00 - 16.00 Uhr}
\end{center}
\begin{tabbing}
Tom Nick \hspace{2cm}\= - 340528\\
Maximillian Bachl \> - 341455 \\
Marius Liwotto\> -  341051
\end{tabbing}
\subsection*{Aufgabe 1}
\begin{enumerate}[(i)]
	\item 
	\begin{itemize}
		\item 	Gilt für alle Graphen mit einer Kante: $A_1 \vDash \varphi_1$, $A_2 \vDash \varphi_1$ , $A_3 \vDash \varphi_1$
		\item 	Gilt für alle Graphen bei dem jeder Knoten zwei Kanten hat und ein dritter Knoten existiert, der keine Kante zu diesem Knoten hat.
			$A_1 \vDash \varphi_2$, $A_2 \vDash \varphi_2$ , $A_3 \not\vDash \varphi_2$
		\item  	Gilt für alle Graphen, bei denen alle unterschiedlichen Knoten mit allen unterschiedlichen Knoten verbunden sind.
			$A_1 \vDash \varphi_3$, $A_2 \vDash \not\varphi_3$ , $A_3 \not\vDash \varphi_3$
	\end{itemize}
	\item 
	\begin{align*}
		\varphi_1 &= \lnot(\forall x \exists y. \cdot( x, y ) = 1) \land \lnot \exists x. \cdot(x,x) = -1 &\\
		\varphi_2 &= \forall x \lnot \exists y. \cdot( x, y ) = 1 \land \lnot \exists x. \cdot(x,x) = -1& \\
		\varphi_3 &= \exists x. \cdot(x,x) = -1&
	\end{align*}
	Es gibt eigentlich nur zwei Bedingugen: Jede Zahl hat eine Zahl, um mit dieser Zahl multipliziert 1 als Ergebnis zu haben. Bei rationalen und komplexen kann man immer den Kehrwert einer Zahl bilden und hat genau das erreicht. Bei natürlichen gibt es den Kehrwert nicht. Es gibt eine Zahl mit sich selber multipliziert, die -1 ergibt. Dies gilt nur in den komplexen Zahles ($i^2 = -1$). Damit hat man genug Bedingungen um für alle 3 Strukturen eine Formel festzulegen, die nur bei ihnen gilt: Bei den natürlichen Zahlen gibt es für nicht jede Zahl ein Inverses Element der Multiplikation (falls 1 das neutrale Element ist). Die rationalen Zahlen habe für jede Zahl ein Inverses Element der Multiplikation (Falls 1 das neutrale Element ist), aber besitzen keine Zahl die mit sich selber multipliziert $-1$ ergibt. Die komplexen erfüllen erste und zweite Bedingung.
	
\end{enumerate}

\subsection*{Aufgabe 2}
\begin{align*}
	var(\phi) &= \{x_1,x_2,...,x_n\} \\
	\beta(x_i) &= 
	\begin{cases}
	\top, & x_i = 1 \\
	\bot, & x_i = 0
	\end{cases} \\
	\varphi_e &= \exists x_1 \exists x_2 ...\exists x_n \neg Z^{\mathcal{A}}( \llbracket \phi \rrbracket^{\beta}) \\
	\varphi_t &= \forall x_1 \forall x_2 ...\forall x_n \neg Z^{\mathcal{A}}( \llbracket \phi \rrbracket^{\beta})
\end{align*}

\subsection*{Aufgabe 3}
\begin{align*}
	\varphi_k &= \exists x_1...\exists x_k(\forall y( \bigvee_{1 \le i \le k} x_i = y \lor E(y,x_i))
\end{align*}

\end{document}