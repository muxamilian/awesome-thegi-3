\documentclass[a4paper,10pt]{article}
\usepackage[utf8]{inputenc}
\usepackage{amsmath}
\usepackage{amsfonts}
\usepackage{amssymb}
\usepackage[german]{babel}
\setlength{\parindent}{0cm}
\usepackage{setspace}
\usepackage{mathpazo}
\usepackage{graphicx}
\usepackage{wasysym} 
\usepackage{booktabs}
\usepackage{verbatim}
\usepackage{enumerate}
\usepackage{hyperref}
\usepackage{ulem}
\usepackage{stmaryrd }
\usepackage[a4paper,
left=1.8cm, right=1.8cm,
top=2.0cm, bottom=2.0cm]{geometry}
\usepackage{tabularx}

\newcommand{\rf}{\right\rfloor}
\newcommand{\lf}{\left\lfloor}
\newcommand{\tabspace}{15cm}
\newcommand{\N}{\mathbb{N}}
\newcommand{\Z}{\mathbb{Z}}

\begin{document}
\begin{center}
\Large{Theoretische Grundlagen der Informatik 3: Hausaufgabenabgabe 8} \\
\large{Tutorium: Sebastian , Mi 14.00 - 16.00 Uhr}
\end{center}
\begin{tabbing}
Tom Nick \hspace{2cm}\= - 340528\\
Maximillian Bachl \> - 341455 \\
Marius Liwotto\> -  341051
\end{tabbing}
\subsection{Aufgabe 1}
\begin{enumerate}[(i)]
	\item 
	\begin{itemize}
		\item 	Gilt für alle Graphen mit einer Kante: $A_1 \vDash \varphi_1$, $A_2 \vDash \varphi_1$ , $A_3 \vDash \varphi_1$
		\item 	Gilt für alle Graphen bei dem jeder Knoten zwei Kanten hat und ein dritter Knoten existiert, der keine Kante zu diesem Knoten hat.
			$A_1 \vDash \varphi_2$, $A_2 \vDash \varphi_2$ , $A_3 \not\vDash \varphi_2$
		\item  	Gilt für alle Graphen, bei denen alle unterschiedlichen Knoten mit allen unterschiedlichen Knoten verbunden sind.
			$A_1 \vDash \varphi_3$, $A_2 \vDash \not\varphi_3$ , $A_3 \not\vDash \varphi_3$
	\end{itemize}
	\item 
	\begin{align*}
		\varphi_1 &= \lnot(\forall x \exists y. \cdot( x, y ) = 1) \land \lnot \exists x. \cdot(x,x) = -1 &\\
		\varphi_2 &= \forall x \lnot \exists y. \cdot( x, y ) = 1 \land \lnot \exists x. \cdot(x,x) = -1& \\
		\varphi_3 &= \exists x. \cdot(x,x) = -1&
	\end{align*}
	
\end{enumerate}
\end{document}