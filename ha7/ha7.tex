\documentclass[a4paper,10pt]{article}
\usepackage[utf8]{inputenc}
\usepackage{amsmath}
\usepackage{amsfonts}
\usepackage{amssymb}
\usepackage[german]{babel}
\setlength{\parindent}{0cm}
\usepackage{setspace}
\usepackage{mathpazo}
\usepackage{graphicx}
\usepackage{wasysym} 
\usepackage{booktabs}
\usepackage{verbatim}
\usepackage{enumerate}
\usepackage{hyperref}
\usepackage{ulem}
\usepackage{stmaryrd }
\usepackage[a4paper,
left=1.8cm, right=1.8cm,
top=2.0cm, bottom=2.0cm]{geometry}
\usepackage{tabularx}

\newcommand{\tabspace}{15cm}
\newcommand{\N}{\mathbb{N}}
\newcommand{\Z}{\mathbb{Z}}

\begin{document}
\begin{center}
\Large{Theoretische Grundlagen der Informatik 3: Hausaufgabenabgabe 7} \\
\large{Tutorium: Sebastian , Mi 14.00 - 16.00 Uhr}
\end{center}
\begin{tabbing}
Tom Nick \hspace{2cm}\= - 340528\\
Maximillian Bachl \> - 341455 \\
Marius Liwotto\> -  341051
\end{tabbing}
\subsection*{Aufgabe 1}
trivial ;)

\subsection*{Aufgabe 2}
\begin{enumerate}
\item[(i)]
\textbf{Homomorphismus:} \\
\\
\( h: \N \rightarrow \Z \\
n \mapsto n \) \\
\\
\textbf{Beweis für Richtigkeit des Homomorphismus:} \\
\\
Konstanten:\\
\\
\( h(0^{\mathcal{N}}) = 0^{\mathcal{N}} = 0 \\
h(1^{\mathcal{N}}) = 1^{\mathcal{N}} = 1 \) \\
\\
Operatoren: \\
\\
Sei $n_1,n_2 \in \N$:\\
\\
\( h(+^{\mathcal{N}} (n_1,n_2)) = h(n_1 + n_2) = n_1 + n_2 
= h(n_1) + h(n_2) = +^{\mathcal{Z}} (h(n_1),h(n_2)) \\
\\
 h(\cdot^{\mathcal{N}} (n_1,n_2)) = h(n_1 \cdot n_2) = n_1 \cdot n_2 
= h(n_1) \cdot h(n_2) = \cdot^{\mathcal{Z}} (h(n_1),h(n_2)) \) \\
\\
Somit ist h ein gültiger Homomorphismus von $\N$ nach $\Z$.

\item[(i)]
Angenommen es gäbe besagten Homomorphismus. Nach der Definition des Homomorphismus, müssen die Konstantensymbole wieder auf sich abgebildet werden.
Somit gilt:

$h(0^{\mathcal{Z}}) = 0^{\mathcal{N}} = 0 $ \\
$h(1^{\mathcal{Z}}) = 1^{\mathcal{N}} = 1 $ \\

Außerdem gilt: 

\begin{align*}
&h(\cdot^{\mathcal{Z}}(-1,-1)) = h(1) \overset{Def. Hom.}{=} 1 = \cdot^{\mathcal{N}}(h(-1),h(-1))\\
&\Rightarrow -1 \mapsto 1\\
\\
&h(+^{\mathcal{Z}}(-1,1)) = h(0) = 0 \neq 2 = +^{\mathcal{N}}(1,1) = +^{\mathcal{N}}(h(-1),h(1))
\end{align*}

Somit entsteht unweigerlich ein Widerspruch, wenn es einen Homomorphismus von $\mathbb{Z}$ nach $\mathbb{N}$ gäbe.

\end{enumerate}

\subsection*{Aufgabe 3}
Damit $\mathfrak{B}_{h(A)} \subseteq \mathfrak{B}$ gilt, müssen folgende Bedingungen erfüllt sein: \\
\\
\begin{enumerate}
\item[(i)]
Das Bild von h(A) $\subseteq$ B 

\item[(ii)]
Für alle Operatoren $op^\mathfrak{B}_{h(A)}$ der Substruktur $\mathfrak{B}_{h(A)}$, muss gelten, \\
dass sie abgeschlossen bzgl. des Bildes von h(A) sind.

\end{enumerate}

\textbf{Beweis für (i):} \\
\\
Der Homomorphismus h ist als Funktion folgendermaßen definiert: \\
\\
h: A $\rightarrow$ B \\
\\
Daraus folgt sofort, dass h(A) $\subseteq$ B ist. \\
\\
\textbf{Beweis für (ii): }\\
\\
Da h ein Homomorphismus von A  nach B ist, muss gelten mit n als Stelligkeit des Operators: \\
\\
\( (*)~ \forall a_1~ \forall a_2 ... \forall a_n~ h(op^{\mathfrak{A}}(a_1,a_2,...,a_n)) 
= op^{\mathfrak{B}}(h(a_1),h(a_2),...,h(a_n)) 
= op^{\mathfrak{B}}_{h(A)}(h(a_1),h(a_2),...,h(a_n))\) \\
\\
Wären die Operatoren nicht abgeschlossen bzgl. des Bildes von h, dann würde folgendes gelten: \\
\\
$\exists a_1~ \exists a_2 ... \exists a_n~ (op^\mathfrak{B}_{h(A)} (h(a_1),h(a_2),...,h(a_n))) \notin$ h(A) \\
\\
Nach (*) gilt aber: \\
\\
$\forall a_1~ \forall a_2 ... \forall a_n~ 
(h(op^{\mathfrak{A}}(a_1,a_2,...,a_n)) = op^{\mathfrak{B}}_{h(A)}(h(a_1),h(a_2),...,h(a_n)) )$ mit\\
\\
$h(op^{\mathfrak{A}}(a_1,a_2,...,a_n)) \in h(A) \Leftrightarrow op^{\mathfrak{B}}_{h(A)}(h(a_1),h(a_2),...,h(a_n)) \in h(A)$ \\
\\
was ein Widerspruch zur Annahme darstellt.\\
Folglich müssen die Operatoren abgeschlossen sein, weshalb h(A) eine $\sigma$-abgeschlossene Menge ist. \\
\\
Da (i) und (ii) gilt, folgt nach dem Satz der VL, dass h(A) eine Substruktur 
$\mathfrak{B}_{h(A)} \subseteq \mathfrak{B}$ induziert.

\subsection*{Aufgabe 4}
\textbf{Struktur:} \\
\\
\( \mathcal{N} = (\N, <^{\N}) \\
\\
n_1 <^{\mathcal{N}} n_2 \text{~~~~~gdw. ($n_1$ mod 2 $< n_2$ mod 2) oder ($n_1 = n_2$ und $n_1 < n_2$)} \\
\\
\text{Damit $\mathcal{N}$ und $\mathcal{M}$ isomorph sind, muss es einen Isomorphismus zwischen $\{0,1\} \times \N$ und} \\
\text{$\N$ geben.} \\
\\
\textbf{Isomorphismus:} \\
\\
b: \N \rightarrow \{0,1\} \times \N \\
\\
n \rightarrow (n \text{ mod } 2, \left\lfloor \frac{n}{2} \right\rfloor) \) \\
\\
\textbf{Beweis der Richtigkeit von b:} \\
\\
b muss folgende Dinge erfüllen: 
\begin{enumerate}
\item[(i)]
b muss eine Bijektion sein: \\
b: $\N \mapsto \{0,1\} \times \N$

\item[(ii)]
Für alle n-stelligen Relationssymbole R $\in \sigma$ 
und alle $\overline{a} := a_1,a_2,...,a_n \in \N^n$: \\
wenn $\overline{a} \in \N^n$ genau, dann wenn $(b(a_1),b(a_2),...,b(a_n)) \in \{0,1\} \times \N$

\item[(iii)]
Für alle n-stelligen Funtionssymbole f $\in \sigma$ 
und alle $\overline{a} := a_1,a_2,...,a_n \in \N^n$: \\
$b(f^{\mathcal{N}}(\overline{a})) = f^{\mathcal{M}}(b(a_1),b(a_2),...,b(a_n))$

\item[(iv)]
Für alle Konstantensymbole $c \in \sigma$ gilt: \\
$b(c^{\mathcal{N}}) = c^{\mathcal{M}}$

\end{enumerate}
\textbf{Beweis für (i):} \\
\\
Damit b eine Bijektion ist, muss b und seine Inverse die Eigenschaften einer Funktion erfüllen, \\
nämlich Linkstotalität und Injektivität. \\
\\
b ist offensichtlich linkstotal.\\
b ist auch injektiv, da Folgendes gilt: \\
\\
Seien $n_1,n_2 \in \N$ mit $n_1 \neq n_2$: \\
\\
b($n_1$) = ($n_1$ mod 2, $n_1$) $\neq$ ($n_2$ mod 2, $n_2$) = b($n_2$) \\
\\
gilt wegen $n_1 \neq n_2$.\\
\\
Die Inverse von b: \\
\\
\( b^{-1}: \{0,1\} \times \N \rightarrow \N \\
\\
(k, n) \rightarrow 
\begin{cases}
2 \cdot n, & k = 0 \\
2 \cdot n + 1, & sonst
\end{cases} \) \\
\\
$b^{-1}$ ist offensichtlich linkstotal. \\
$b^{-1}$ ist auch injektiv, da Folgendes gilt: \\
\\
Seien $(k_1,n_1),(k_2,n_2) \in \{0,1\} \times \N$ mit $(k_1,n_1) \neq (k_2,n_2)$: \\
\\
\begin{tabular}{@{} lll} 
\textbf{Fall 1} & $k_1 = 0 \neq 1 = k_2$: & $b^{-1} (k_1,n_1) = 2 \cdot n_1 \neq 2 \cdot n_2 + 1 = b^{-1} (k_1,n_2)$ \\
				&&\\
\textbf{Fall 2} & $k_1 = 1 \neq 0 = k_2$: & analog zu Fall 1\\
&&\\
\textbf{Fall 3} & $n_1 \neq n_2$ und : & $b^{-1} (k_1,n_1) = 2 \cdot n_1 \neq 2 \cdot n_2 = b^{-1} (k_2,n_2)$ \\
			    & $k_1 = k_2 = 0$	   &\\
&&\\
\textbf{Fall 4} & $n_1 \neq n_2$ und : & $b^{-1} (k_1,n_1) = 2 \cdot n_1 + 1 \neq 2 \cdot n_2 + 1 = b^{-1} (k_2,n_2)$ \\
			    & $k_1 = k_2 = 1$	   &
\end{tabular} \\
\\
\\
Die Inverse $b^{-1}$ ist also injektiv. \\
b ist damit eine Bijektion. \\
\\
\textbf{Beweis für (ii):} \\
\\
2-stelliges Relationssymbol $<$: \\
\\
\( \forall a_1 \forall a_2~ ((a_1,a_2) \in <^{N} \\
\Leftrightarrow (a_1 \text{ mod } 2 < a_2 \text{ mod } 2) \vee (a_1 = a_2 \wedge a_1 < a_2) \\
\Leftrightarrow (b(a_1),b(a_2)) \in <^{M}) \) \\
\\
\textbf{Beweis für (iii):} \\
\\
Es gibt keine Funktionssymbole aus $\sigma$, also ist hier nichts zu beweisen. \\
\\
\\
\\
\textbf{Beweis für (iv):} \\
\\
Es gibt keine Konstantensymbole aus $\sigma$, also ist hier nichts zu beweisen. \\
\\
\\
b erfüllt damit i-iv und ist somit ein korrekter Isomorphismus von $\mathcal{N}$ nach $\mathcal{M}$. 
\end{document}