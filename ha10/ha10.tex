\documentclass[a4paper,10pt]{article}
\usepackage[utf8]{inputenc}
\usepackage{amsmath}
\usepackage{amsfonts}
\usepackage{amssymb}
\usepackage[german]{babel}
\setlength{\parindent}{0cm}
\usepackage{setspace}
\usepackage{mathpazo}
\usepackage{graphicx}
\usepackage{wasysym} 
\usepackage{booktabs}
\usepackage{verbatim}
\usepackage{enumerate}
\usepackage{hyperref}
\usepackage{ulem}
\usepackage{stmaryrd }
\usepackage[a4paper,
left=1.8cm, right=1.8cm,
top=2.0cm, bottom=2.0cm]{geometry}
\usepackage{tabularx}

\newcommand{\rf}{\right\rfloor}
\newcommand{\lf}{\left\lfloor}
\newcommand{\tabspace}{15cm}
\newcommand{\N}{\mathbb{N}}
\newcommand{\Z}{\mathbb{Z}}

\begin{document}
\begin{center}
\Large{Theoretische Grundlagen der Informatik 3: Hausaufgabenabgabe 10} \\
\large{Tutorium: Sebastian , Mi 14.00 - 16.00 Uhr}
\end{center}
\begin{tabbing}
Tom Nick \hspace{2cm}\= - 340528\\
Maximillian Bachl \> - 341455 \\
Marius Liwotto\> -  341051
\end{tabbing}
\subsection*{Aufgabe 1}
\begin{enumerate}[(i)]
	\item $\mathcal{A}_1 = (\mathbb{C},M^{\mathcal{A}_1})$ und $\mathcal{B}_1 = (\mathbb{R}, M^{\mathcal{B}_1})$, wobei $M$ ein 3-stelliges Relationssymbol ist und es gilt $(a,b,c) \in M^{\mathcal{A}_1} \Leftrightarrow a \cdot b = c$ und für $a,b,c \in \mathbb{C}$ und $M^{\mathcal{B}_1} = M^{\mathcal{B}_1} \cap \mathbb{R}^3$ \\
	\textbf{Die Duplikatorin gewinnt das 1-Runden Spiel}
		\begin{enumerate}[1. \text{Zug:}]
			\item  	
			\begin{enumerate}[\text{Fall} 1.]
				\item  	H wählt $a_1 \in \mathbb{C}$ mit $M(a_1,a_1,a_1)$ \\
					D wählt $b_1 \in \mathbb{R}$ mit $M(b_1,b_1,b_1)$
				\item  	H wählt $a_1 \in \mathbb{C}$ mit $\lnot(M(a_1,a_1,a_1))$ \\
					D wählt beliebiges $b_1 \in \mathbb{R}$
			\end{enumerate}
		\end{enumerate}
	\textbf{Der Herausforderer gewinnt das 2-Runden Spiel}
		\begin{enumerate}[1. \text{Zug:}]
			\item  	H wählt $a_1 \in \mathbb{C}$ mit $M(a_1,a_1,a_1)$ \\
				D wählt $b_1 \in \mathbb{R}$ mit $M(b_1,b_1,b_1)$ sonst verliert sie sofort.
			\item 	H wählt $a_2 \in \mathbb{C}$ mit $M(a_2,a_2,a_1) \land a_1 \neq a_2$ \\
			Dann gilt $M^{\mathcal{A}_1}(a_1,a_1,a_1), M^{\mathcal{A}_1}(a_2,a_2,a_1)$ \\
			Da $M^{\mathcal{B}_1}(b_1,b_1,b_1)$ gelten muss, muss $b_1$ gleich 1 oder 0 sein. Doch besitzt $\mathbb{R}$ kein $b_2$, dass
			$M^{\mathcal{A}_1}(b_2,b_2,a_1)$ erfüllt, weder für 0 noch für 1.
		\end{enumerate}
	Aus dem Spiel folgt die Formel: $\exists a \exists b (M(a,a,a) \land M(b,b,a) \land a \neq b )$ 
	\item  	$\mathcal{A}_1 = (\mathbb{C},M^{\mathcal{A}_1})$ und $\mathcal{B}_1 = (\mathbb{R}, M^{\mathcal{B}_1})$.
	\textbf{Die Duplikatorin gewinnt das 1-Runden Spiel}
		\begin{enumerate}[1. \text{Zug:}]
			\item  	
			\begin{enumerate}[\text{Fall} 1.]
				\item  	H wählt $a_1 \in \Z$ mit $R(a_1,a_1,a_1)$ \\
					D wählt $b_1 \in \Z$ mit $R(b_1,b_1,b_1) $ 
				\item  	H wählt $a_1 \in \mathbb{Z}$ mit $\lnot(R(a_1,a_1,a_1))$ \\
					D wählt beliebiges $b_1 \in \mathbb{Z}$
			\end{enumerate}
		\end{enumerate}
	\textbf{Der Herausforderer gewinnt das 2-Runden Spiel}
		\begin{enumerate}[1. \text{Zug:}]
			\item  	H wählt $a_1 \in \mathbb{Z}$ mit $R(a_1,a_1,a_1)$ \\
				D wählt $b_1 \in \Z$ mit $R(b_1,b_1,b_1) $, sonst verliert sie sofort.
			\item  	H wählt $a_2 \in \mathbb{Z}$ mit $R(a_2,a_2,a_2) \land a_1 \neq a_2$ \\
			Dann gilt $R^{\mathcal{A}_1}(a_1,a_1,a_1), R^{\mathcal{A}_1}(a_2,a_2,a_2)$. $R^{\mathcal{B}_1}(b_1,b_1,b_1)$ gilt zwar auch, aber da $a_1 \neq a_2 \Rightarrow b_1 \neq b_2$ gelten muss, jedoch nur die $0$ diese Bedingung erfüllt, gewinnt H das 2-Runden Spiel.
			
		\end{enumerate}
	Aus dem Spiel folgt die Formel: $\exists a \exists b(R(a,a,a) \land R(b,b,b) \land b \neq a)$

\end{enumerate}
\subsection*{Aufgabe 2}

\subsection*{Aufgabe 3}

\subsection*{Aufgabe 4}

\end{document}



