\documentclass[a4paper,10pt]{article}
\usepackage[utf8]{inputenc}
\usepackage{amsmath}
\usepackage{amsfonts}
\usepackage{amssymb}
\usepackage[german]{babel}
\setlength{\parindent}{0cm}
\usepackage{setspace}
\usepackage{mathpazo}
\usepackage{graphicx}
\usepackage{wasysym} 
\usepackage{booktabs}
\usepackage{verbatim}
\usepackage{enumerate}
\usepackage{hyperref}
\usepackage{ulem}
\usepackage{stmaryrd }
\usepackage[a4paper,
left=1.8cm, right=1.8cm,
top=2.0cm, bottom=2.0cm]{geometry}
\usepackage{tabularx}

\newcommand{\rf}{\right\rfloor}
\newcommand{\lf}{\left\lfloor}
\newcommand{\tabspace}{15cm}
\newcommand{\N}{\mathbb{N}}
\newcommand{\Z}{\mathbb{Z}}

\begin{document}
\begin{center}
\Large{Theoretische Grundlagen der Informatik 3: Hausaufgabenabgabe 10} \\
\large{Tutorium: Sebastian , Mi 14.00 - 16.00 Uhr}
\end{center}
\begin{tabbing}
Tom Nick \hspace{2cm}\= - 340528\\
Maximillian Bachl \> - 341455 \\
Marius Liwotto\> -  341051
\end{tabbing}
\subsection*{Aufgabe 1}
\begin{enumerate}[(i)]
	\item $\mathcal{A}_1 = (\mathbb{C},M^{\mathcal{A}_1})$ und $\mathcal{B}_1 = (\mathbb{R}, M^{\mathcal{B}_1})$, wobei $M$ ein 3-stelliges Relationssymbol ist und es gilt $(a,b,c) \in M^{\mathcal{A}_1} \Leftrightarrow a \cdot b = c$ und für $a,b,c \in \mathbb{C}$ und $M^{\mathcal{B}_1} = M^{\mathcal{B}_1} \cap \mathbb{R}^3$ \\
	Zuerst ein paar Formeln: 
	\begin{align*}
		\phi_1(x) &:= \forall y.M(x,y,y) \\
		\phi_{-1}(x) &:= \exists a \exists b. M(a, a, x) \land M(x,x,b) \land \phi_1(b) \\
		\phi_{i}(x) &:= \exists a.M(x, x, a) \land \phi_{-1}(a)
	\end{align*}
	\textbf{Die Duplikatorin gewinnt das 2-Runden Spiel}
		\begin{enumerate}[1. \text{Zug:}]
			\item  	
			\begin{enumerate}[\text{Fall} 1.]
				\item  	H wählt $a_1 \in \mathbb{C}$ mit $\phi_{-1}(a_1)) \lor \phi_1(a_1) \lor \phi_i(a_1)$ \\
					D wählt beliebiges $b_1 \in \mathbb{R}$
				\item  	H wählt $a_1 \in \mathbb{C}$ mit $\lnot(\phi_{-1}(a_1)) \lor \phi_1(a_1) \lor \phi_i(a_1))$ \\
					D wählt beliebiges $b_1 \in \mathbb{R}$
			\end{enumerate}
			\item 
			\begin{enumerate}[\text{Fall} 1.]
				\item 	H wählt $a_2 \in \mathbb{C}$ mit $M^{\mathcal{A}_1}(a_1, a_1, a_2)$ \\
					D wählt $b_1 \in \mathbb{R}$ mit $M^{\mathcal{B}_1}(b_1,b_1,b_2)$ 
				\item  	H wählt $a_2 \in \mathbb{C}$ mit $\phi_i(a_1) \lor \phi_1(a_1) \lor \phi_{-1}(a_1) \land \lnot M^{\mathcal{A}_1}(a_1,a_1,a_2) \land a_1 \neq a_2$  
				\item  	H wählt $a_2 \in \mathbb{C}$ mit $a_1 = a_2$ \\
					D wählt $b_2 \in \mathbb{R} $ mit $b_1 = b_2$
				\item  	H wählt $a_2 \in  \mathbb{C}$ mit $\lnot(\phi_i(a_1) \lor \phi_1(a_1) \lor \phi_{-1}(a_1)) \land \lnot M^{\mathcal{A}_1}(a_1,a_1,a_2) \land a_1 \neq a_2$ \\
					D wählt $b_2 \in \mathbb{R}$ beliebig
			\end{enumerate}
		\end{enumerate}
	\textbf{Der Herausforderer gewinnt das 3-Runden Spiel}
		\begin{enumerate}[1. \text{Zug:}]
			\item  	H wählt $a_1 \in \mathbb{C}$ mit $\phi_{i}(a_1))$ \\
				D wählt $b_1 \in \mathbb{R}$ beliebig
			\item 	H wählt $a_2 \in \mathbb{C}$ mit $\phi_{-1}(a_2)$ \\
				D wählt $b_1 \in \mathbb{R}$ mit $M^{\mathcal{B}_1}(b_1,b_1,b_2)$, sonst verliert sie sofort.
			\item  	H wählt $a_3 \in \mathbb{C}$ mit $\phi_{1}(a_3)$ \\
			Dann gilt $M^{\mathcal{A}_1}(a_1,a_1,a_2), M^{\mathcal{A}_1}(a_2,a_2,a_3), M^{\mathcal{A}_1}(a_3,a_3,a_3)$ \\
			Da $M^{\mathcal{B}_1}(b_3,b_3,b_3)$ gelten muss kann $b_3$ nur $0$ oder $1$ sein. Da aber auch $M^{\mathcal{B}_1}(b_2,b_2,b_3)$ gilt, muss $b_3 = 1$ sein, $b_2$ kann nur $-1$ sein, da $b_3 = 1$. Doch gibt es kein $b_1 \in  \mathbb{R}$ mit $M^{\mathcal{B}_1}(b_1,b_1,b_2)$.
			
		\end{enumerate}
\end{enumerate}
\subsection*{Aufgabe 2}

\subsection*{Aufgabe 3}

\subsection*{Aufgabe 4}

\end{document}



