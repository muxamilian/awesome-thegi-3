\documentclass[11pt]{amsart}
\usepackage[utf8x]{inputenc}
\usepackage{amsmath}
\usepackage{amsfonts}
\usepackage{amssymb}
\usepackage{fullpage}
\usepackage[german]{babel}
\setlength{\parindent}{0cm}
\usepackage{setspace}
\usepackage{mathpazo}
\usepackage{graphicx}
\usepackage{wasysym} 
\usepackage{booktabs}
\usepackage{verbatim}
\usepackage{pst-all}
\usepackage{enumerate}
\usepackage{pstricks}
\usepackage[a4paper,
left=1.8cm, right=1.8cm,
top=2.0cm, bottom=2.0cm]{geometry}
\newcommand{\N}{\mathbb{N}}
\newcommand{\A}{\mathcal{A}}
\newcommand{\ts}{\textsf}

\usepackage{tabularx}

\newcolumntype{L}[1]{>{\raggedright\arraybackslash}p{#1}}
\newcolumntype{C}[1]{>{\centering\arraybackslash}p{#1}}
\newcolumntype{R}[1]{>{\raggedleft\arraybackslash}p{#1}}


\author{}

\begin{document}

\begin{center}
\Large{Theoretische Grundlagen der Informatik 3: Hausaufgabenabgabe 1} \\
\large{Tutorium: , Mi 14.00 - 16.00 Uhr}
\end{center}
\begin{tabbing}
Tom Nick \hspace{2cm}\= - 340528\\
Maximillian Bachl \> - 341455 \\
Marius Liwotto\> -  
\end{tabbing}

\section{Aufgabe}
\textbf{T} - Tobi kommt $\mid$
\textbf{C} - Christoph kommt $\mid$
\textbf{S} - Sebastian kommt $\mid$
\textbf{V} - Viktor kommt $\mid$
\textbf{F} - Friederike kommt
\\

\begin{align*}
(T \to C \land S) \land
(C \lor V) \land
(S \to \lnot F) \land
(\lnot V) \land
(\lnot T \to \lnot C) &\leftrightarrow \\
(\lnot T \lor (C \land S)) \land
(C \lor V) \land
(\lnot S \lor \lnot F) \land
(\lnot V) \land
(T \lor \lnot C) &\leftrightarrow \\
(\lnot T \lor C) \land (\lnot T \lor S) \land
C \land \lnot V \land
(\lnot S \lor \lnot F) \land
(T \lor \lnot C) &\leftrightarrow \\
C \land (\lnot T \lor S) \land
\lnot V \land
(\lnot S \lor \lnot F) \land
(T \lor \lnot C) &\leftrightarrow \\
C \land (\lnot T \lor S) \land
\lnot V \land
(\lnot S \lor \lnot F) \land
T &\leftrightarrow \\
C \land S \land
\lnot V \land
\lnot F \land
T
\end{align*}

\section{Aufgabe}

\begin{enumerate}[(i)]
\item Ist nicht erfüllbar
\begin{align*}
\lnot(X \to (Y \to X)) &\leftrightarrow \\
\lnot(\lnot X \lor (\lnot Y \lor X)) &\leftrightarrow \\
\lnot\top &\leftrightarrow \\
\bot
\end{align*}

\item Ist erfüllbar
\begin{align*}
(X \land (Y \to \lnot X)) \to Y &\leftrightarrow \\
\lnot(X \land (\lnot Y \lor \lnot X)) \lor Y &\leftrightarrow \\
\lnot(X \land \lnot Y) \lor Y &\leftrightarrow \\
\lnot X \lor Y \lor Y &\leftrightarrow \\
\lnot X \lor Y
\end{align*}

\item Ist erfüllbar
\begin{align*}
(\lnot X \to (X \land Y)) \to (Y \to X) &\leftrightarrow \\
\lnot (X \lor (X \land Y)) \lor \lnot(Y \lor X) &\leftrightarrow\\
\lnot ((X \lor (X \land Y)) \land (Y \lor X)) &\leftrightarrow\\
\lnot (X \land (Y \lor X)) &\leftrightarrow\\
\lnot (X \land (Y \lor X)) &\leftrightarrow\\
\lnot X
\end{align*}

\item Ist erfüllbar

\begin{align*}
(X \lor Y) \to (X \land Y) &\leftrightarrow\\
\lnot(X \lor Y) \lor (X \land Y) &\leftrightarrow\\
(\lnot X \land \lnot Y) \lor (X \land Y) &\leftrightarrow\\
(X \leftrightarrow Y)
\end{align*}

\item Ist eine Tautologie

\begin{align*}
(X \land Y ) \to (X \lor Y) &\leftrightarrow \\
\lnot (X \land Y ) \lor (X \lor Y) &\leftrightarrow\\
\lnot X \lor \lnot Y \lor (X \lor Y) &\leftrightarrow\\
\lnot X \lor \lnot Y \lor (X \lor Y) &\leftrightarrow\\
\lnot X \lor \lnot Y \lor X \lor Y &\leftrightarrow\\
\top
\end{align*}
\end{enumerate}

\section{Aufgabe}
$$
\mathsf{
\phi_i(a_{n-1},...,a_0,b_{n-1},...,b_0) = concat(a_{n-1},...,a_0) + concat(b_{n-1},...,b_0))_i}
$$
\section{Aufgabe}
% Irgendwas mit der Disjunktive Normalform würde ich probieren. Die war in der Vorlesung und dürfen wir auch benutzen...
\end{document}








