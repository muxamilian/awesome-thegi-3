\documentclass[11pt]{amsart}
\usepackage{geometry}                % See geometry.pdf to learn the layout options. There are lots.
\geometry{a4paper}                   % ... or a4paper or a5paper or ... 
%\geometry{landscape}                % Activate for for rotated page geometry
%\usepackage[parfill]{parskip}    % Activate to begin paragraphs with an empty line rather than an indent
\usepackage{graphicx}
\usepackage{amssymb}
\usepackage{epstopdf}
\usepackage[ngerman]{babel}
\DeclareGraphicsRule{.tif}{png}{.png}{`convert #1 `dirname #1`/`basename #1 .tif`.png}

\title{Hausaufgabe 1}
\author{Tom Nick, Marius Liwotto, Maximilian Bachl (341455)}
%\date{}                                           % Activate to display a given date or no date

\begin{document}
\maketitle

\section{Aufgabe}
T -- Tobi kommt
C -- Christoph kommt
S -- Sebastian kommt
V -- Viktor kommt
F -- Friederike kommt
\\

$
(T \to C \land S) \land
(C \lor V) \land
(S \to \lnot F) \land
(\lnot V) \land
(\lnot T \to \lnot C) \leftrightarrow
$

$
(\lnot T \lor (C \land S)) \land
(C \lor V) \land
(\lnot S \lor \lnot F) \land
(\lnot V) \land
(T \lor \lnot C) \leftrightarrow
$

$
(\lnot T \lor C) \land (\lnot T \lor S) \land
C \land \lnot V \land
(\lnot S \lor \lnot F) \land
(T \lor \lnot C) \leftrightarrow
$

$
C \land (\lnot T \lor S) \land
\lnot V \land
(\lnot S \lor \lnot F) \land
(T \lor \lnot C) \leftrightarrow
$

$
C \land (\lnot T \lor S) \land
\lnot V \land
(\lnot S \lor \lnot F) \land
T \leftrightarrow
$

$
C \land S \land
\lnot V \land
\lnot F \land
T
$

\section{Aufgabe}

Ist nicht erfüllbar

$
\lnot(X \to (Y \to X)) \leftrightarrow \lnot(\lnot X \lor (\lnot Y \lor X)) \leftrightarrow \lnot\top \leftrightarrow \bot
$

Ist erfüllbar

$
(X \land (Y \to \lnot X)) \to Y \leftrightarrow
$

$
\lnot(X \land (\lnot Y \lor \lnot X)) \lor Y \leftrightarrow
$

$
\lnot(X \land \lnot Y) \lor Y \leftrightarrow
$

$
\lnot X \lor Y \lor Y \leftrightarrow
$

$
\lnot X \lor Y
$

Ist erfüllbar

$
(\lnot X \to (X \land Y)) \to (Y \to X) \leftrightarrow
$

$
\lnot (X \lor (X \land Y)) \lor \lnot(Y \lor X) \leftrightarrow
$

$
\lnot ((X \lor (X \land Y)) \land (Y \lor X)) \leftrightarrow
$

$
\lnot (X \land (Y \lor X)) \leftrightarrow
$

$
\lnot (X \land (Y \lor X)) \leftrightarrow
$

$
\lnot X
$

Ist erfüllbar

$
(X \lor Y) \to (X \land Y) \leftrightarrow
$

$
\lnot(X \lor Y) \lor (X \land Y) \leftrightarrow
$

$
(\lnot X \land \lnot Y) \lor (X \land Y) \leftrightarrow
$

$
(X \leftrightarrow Y)
$

Ist eine Tautologie

$
(X \land Y ) \to (X \lor Y) \leftrightarrow
$

$
\lnot (X \land Y ) \lor (X \lor Y) \leftrightarrow
$

$
\lnot X \lor \lnot Y \lor (X \lor Y) \leftrightarrow
$

$
\lnot X \lor \lnot Y \lor (X \lor Y) \leftrightarrow
$

$
\lnot X \lor \lnot Y \lor X \lor Y \leftrightarrow
$

$
\top
$

\section{Aufgabe}

$
% Das Ergebnis dann noch in Top/Bottom konvertieren mit so ner großen LaTeX-Klammer ;)
\phi_i(a_{n-1},...,a_0,b_{n-1},...,b_0) = concat(a_{n-1},...,a_0) + concat(b_{n-1},...,b_0))_i
$

\section{Aufgabe}
% Irgendwas mit der Disjunktive Normalform würde ich probieren. Die war in der Vorlesung und dürfen wir auch benutzen...
\end{document}








