\documentclass[a4paper,10pt]{article}
\usepackage[utf8]{inputenc}
\usepackage{amsmath}
\usepackage{amsfonts}
\usepackage{amssymb}
\usepackage[german]{babel}
\setlength{\parindent}{0cm}
\usepackage{setspace}
\usepackage{mathpazo}
\usepackage{graphicx}
\usepackage{wasysym} 
\usepackage{booktabs}
\usepackage{verbatim}
\usepackage{enumerate}
\usepackage{hyperref}
\usepackage{ulem}
\usepackage{stmaryrd }
\usepackage[a4paper,
left=1.8cm, right=1.8cm,
top=2.0cm, bottom=2.0cm]{geometry}
\usepackage{tabularx}
\usepackage{tikz}
\usetikzlibrary{trees,petri,decorations,arrows,automata,shapes,shadows,positioning,plotmarks}

\newcommand{\rf}{\right\rfloor}
\newcommand{\lf}{\left\lfloor}
\newcommand{\tabspace}{15cm}
\newcommand{\N}{\mathbb{N}}
\newcommand{\Z}{\mathbb{Z}}

\begin{document}
\begin{center}
\Large{Theoretische Grundlagen der Informatik 3: Hausaufgabenabgabe 11} \\
\large{Tutorium: Sebastian , Mi 14.00 - 16.00 Uhr}
\end{center}
\begin{tabbing}
Tom Nick \hspace{2cm}\= - 340528\\
Maximillian Bachl \> - 341455 \\
Marius Liwotto\> -  341051
\end{tabbing}
\subsection*{Aufgabe 1.}
\begin{enumerate}[(i)]
	\item
	\begin{enumerate}[a)]
	Das untenstehende Sequenzkalkül ist korrekt:
		\item \[ (\exists \Rightarrow) \frac{\Phi, \psi(c) \Rightarrow \Delta}{\Phi, \exists x \psi(x)\Rightarrow \Delta} c \text{ kommt nicht in $\Phi,\delta,\psi(x)$ vor}\]
		\textbf{Beweis:} 
		Sei $J = (\mathcal{A},\beta)$ ein $\tau$-Interpretation die $\Phi$ und für mindestens ein $x$, $\psi(x)$ erfüllt. Also:
		\begin{align*}
			J &\vDash \Phi \\
			J &\vDash \exists x \psi(x)
		\end{align*}
		Sei $a := \llbracket c \rrbracket^{J}$. Also gilt $J \vDash \psi[a]$, daraus folgt offensichtlich, dass $J \vDash \psi(c)$ gilt. Nach Vorraussetzung gibt es also ein $\varphi \in \Delta$, sodass $J \vDash \varphi$.
	\end{enumerate}
\end{enumerate}

	
\end{document}