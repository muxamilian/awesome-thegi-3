\documentclass[a4paper,10pt]{article}
\usepackage[utf8]{inputenc}
\usepackage{amsmath}
\usepackage{amsfonts}
\usepackage{amssymb}
\usepackage[german]{babel}
\setlength{\parindent}{0cm}
\usepackage{setspace}
\usepackage{mathpazo}
\usepackage{graphicx}
\usepackage{wasysym} 
\usepackage{booktabs}
\usepackage{verbatim}
\usepackage{enumerate}
\usepackage{hyperref}
\usepackage{ulem}
\usepackage{bussproofs}
\usepackage{stmaryrd }
\usepackage[a4paper,
left=1.8cm, right=1.8cm,
top=2.0cm, bottom=2.0cm]{geometry}
\usepackage{tabularx}
\usepackage{tikz}
\usetikzlibrary{trees,petri,decorations,arrows,automata,shapes,shadows,positioning,plotmarks}

\newcommand{\rf}{\right\rfloor}
\newcommand{\lf}{\left\lfloor}
\newcommand{\tabspace}{15cm}
\newcommand{\N}{\mathbb{N}}
\newcommand{\Z}{\mathbb{Z}}

\begin{document}
\begin{center}
\Large{Theoretische Grundlagen der Informatik 3: Hausaufgabenabgabe 11} \\
\large{Tutorium: Sebastian , Mi 14.00 - 16.00 Uhr}
\end{center}
\begin{tabbing}
Tom Nick \hspace{2cm}\= - 340528\\
Maximillian Bachl \> - 341455 \\
Marius Liwotto\> -  341051
\end{tabbing}
\subsection*{Aufgabe 1}
\begin{enumerate}[(i)]
	\item
	\begin{enumerate}[a)]
	
		\item \textbf{Behauptung: } Die untenstehende Sequenzkalkülregel ist korrekt:
		\[ (\exists \Rightarrow) \frac{\Phi, \psi(c) \Rightarrow \Delta}{\Phi, \exists x \psi(x)\Rightarrow \Delta} c \text{ kommt nicht in $\Phi,\delta,\psi(x)$ vor}\]
		\textbf{Beweis:} 
		Sei $\tau$ die Signatur die alle Relations-, Funktions- und Konstantensymbole enthält, die in $\Phi, \Delta, \psi(x)$ vorkommen, aber nicht $c$.
		Wir nehmen an, dass $\Phi \cup \{ \exists x \psi(x)\}$ erfüllbar ist, sonst sind wir fertig. \\
		Sei also $J = (\mathcal{A}, \beta)$ eine $\tau$-Interpretation mit $J \vDash \Phi \cup \{\exists x\psi(x) \}$. 
		Sei $a \in A$ sodass $J[x/a] \vDash \psi(x)$. Sei $J_a$ die $\tau \cup \{c\}$-Interpretation, die die Konstante $c$ mit $a$ belegt und
		sonst mit $J$ übereinstimmt. Da $J[x/a] \vDash \psi(x)$ und da $c$ nicht in $\Phi$ und $\psi(x)$ vorkommt, gilt $J_a \vDash \Phi \cup \{\psi(c)\}$.
		Nach Vorraussetzung gilt $J_a \vDash \delta$ für ein $\delta \in \Delta$. Da $c$ nicht in $\Delta$ vorkommt, gilt auch $J \vDash \delta$. Dies war zu zeigen.
		\item \textbf{Behauptung: } Die untenstehende Sequenzkalkülregel ist korrekt:
		\[  ( \Rightarrow \exists ) \frac{\Phi \Rightarrow \Delta, \psi(c)}{\Phi\Rightarrow \Delta, \exists x \psi(x)} \]
		Wir nehmen an, dass $\Phi$ erfüllbar ist, sonst sind wir fertig. Sei aber $J = (\mathcal{A}, \beta)$ eine Interpretation mit $J \vDash \Phi$. 
		Falls $J \vDash \delta$ für ein $\delta \in \Delta$, so sind wir fertig. Sonst gilt nach Vorraussetzung $J \vDash \psi(c)$. Sei $a = \llbracket c \rrbracket^J$,
		Daraus gilt $J[x/a] \vDash \psi(x)$
	\end{enumerate}
\end{enumerate}
\subsection*{Aufgabe 2}
\begin{enumerate}[(i)]
\item
$$\frac{\Phi, \psi \Rightarrow \Delta}{\Phi \Rightarrow \Delta}$$

Wir geben ein Gegenbeispiel an. Wir wählen $\Phi = \{\top\}$, $\psi = \bot$ sowie $\Delta = \{\top\}$. 

Dann gilt:

$$\frac{\{\top\}, \bot \Rightarrow \{\top\}}{\{\top\} \Rightarrow \{\top\}}$$

Somit gilt in jeder beliebigen Interpretation $\mathcal{J}$, dass die obere Sequenz ungültig ist, die untere aber nicht.
\item
$$\frac{\Phi, \lnot\forall x\varphi \Rightarrow \Delta}{\Phi \Rightarrow \Delta, \exists x \varphi}$$

Wir geben ein Gegenbeispiel an. Wir wählen $\Phi = \{\top\}$, $\varphi = (x=x)$ sowie $\Delta = \emptyset$. 

Dann gilt:

$$\frac{\{\top\}, (\lnot \forall x ~ x = x) \Rightarrow \emptyset}{\{\top\} \Rightarrow (\exists x~ x=x)}$$

Somit gibt es offensichtlich eine Interpretation, sodass die obere Sequenz gültig ist, die untere aber nicht


\end{enumerate}
\subsection*{Aufgabe 3}
Es gilt:
\begin{align*}
	\{ \forall x f(x) = x \} \Rightarrow \{ f(f(c)) = c \} \equiv \{ \forall x f(x) = x \} \cup \{ \forall x f(x) = x \}   \Rightarrow \{ f(f(c)) = c \}
\end{align*}
Mit dem Sequenzenkalkül kann man das Axiom nun beweisen:
\begin{prooftree}
	\AxiomC{$\{f(f(c)) = c \} \Rightarrow \{f(f(c)) = c \}$}
	\LeftLabel{$(\mathcal{S} \Rightarrow)$}
	\UnaryInfC{$\{f(c) = c, f(f(c)) = f(c) \} \Rightarrow \{f(f(c)) = c \}$}
	\LeftLabel{$(\forall \Rightarrow)$}
	\UnaryInfC{$\{\forall x f(x) = x, f(f(c)) = f(c) \} \Rightarrow \{f(f(c)) = c \}$}
	\LeftLabel{$(\forall \Rightarrow)$}
	\UnaryInfC{$\{ \forall x f(x) = x \} \cup \{ \forall x f(x) = x \}   \Rightarrow \{ f(f(c)) = c \}$}
\end{prooftree}
\subsection*{Aufgabe 4}
Da $\sigma = \emptyset$ gilt, dass alle Formeln $\varphi \in FO[\sigma]$ in der Auswertung in verschiedenen $\sigma$-Strukturen sich nur über den Quantorenrang von $\varphi$ unterscheiden. Da $\varphi$ endlich ist (Eingabe ist endlich), hat $\varphi$ einen Quantorenrang $qr(\varphi) = m$. Würde man nun eine $\sigma$-Struktur suchen die ein Modell von $\varphi$ ist, würde es reichen beliebige Strukturen  mit $m$ Elementen zu testen, da jede Struktur mit mehr als $m$ Elementen elementar-äquivalent zu der mit $m$ ist. Damit kann jede Berechnung der Turingmaschine die ein $\varphi$ bekommen hat, nach dem untersuchen von Strukturen mit $1-m$ Elementen abbrechen, falls keine gefunden wurde und eindeutig sagen, dass es kein Modell gibt bzw. dass es ein Modell gibt. Dies entspricht der Definition von Entscheidbarkeit.

\end{document}