\documentclass[a4paper,10pt]{article}
\usepackage[utf8]{inputenc}
\usepackage{amsmath}
\usepackage{amsfonts}
\usepackage{amssymb}
\usepackage{fullpage}
\usepackage[german]{babel}
\setlength{\parindent}{0cm}
\usepackage{setspace}
\usepackage{mathpazo}
\usepackage{graphicx}
\usepackage{wasysym} 
\usepackage{booktabs}
\usepackage{verbatim}
\usepackage{enumerate}
\usepackage{ulem}
\usepackage{ stmaryrd }
\usepackage[a4paper,
left=1.8cm, right=1.8cm,
top=2.0cm, bottom=2.0cm]{geometry}


\begin{document}

\begin{center}
\Large{Theoretische Grundlagen der Informatik 3: Hausaufgabenabgabe 2} \\
\large{Tutorium: Sebastian , Mi 14.00 - 16.00 Uhr}
\end{center}
\begin{tabbing}
Tom Nick \hspace{2cm}\= - 340528\\
Maximillian Bachl \> - 341455 \\
Marius Liwotto\> -  341051
\end{tabbing}
\subsection*{Aufgabe 1}
Beweis durch strukturelle Induktion. Seien dazu $\beta$ und $\beta'$ zwei Belegungen mit $\beta \leq \beta'$, sowie $\varphi$ eine Formeln die nur aus Variablen $X_1,...,X_n$ und $\bot,\top,\land,\lor$ besteht.
\begin{itemize}
\item[\textbf{IA}] $\varphi = X_1$ \\ 
Es gibt drei Möglichkeiten für $\beta$ und $\beta'$, sodass $\beta \leq \beta'$. 
\begin{itemize}
\item $\beta(X_1) = 0 \leq  0 = \beta'(X_1)$ \\
$\llbracket \varphi \rrbracket^{\beta} = 0 \leq 0 = \llbracket \varphi \rrbracket^{\beta'} \Rightarrow \llbracket \varphi \rrbracket^{\beta} \leq \llbracket \varphi \rrbracket^{\beta'} $
\item Die anderen Fälle sind analog.
\end{itemize}
\item[\textbf{IV}] Sei $\varphi$ eine monotone Formel, wobei diese nur aus Variablen $X_1,...,X_n$ und $\bot,\top,\land,\lor$ besteht.
\item[\textbf{IS}] Da das Anhängen von $\top$ und $\bot$ mittels $\land, \lor$ keinerlei Einfluss auf die Monotonie von Formeln hat (Da die Belegungen nur Einfluss auf Variablen hat), werden diese hier nicht betrachtet. Es ist nun zu zeigen, dass die Aussage auch für $\varphi \land X_{n+1}$ und $\varphi \lor X_{n+1}$ gilt.
\begin{itemize}
\item $\varphi \land X_{n+1}$ mit $\llbracket \varphi \rrbracket^{\beta} = 0 \stackrel{\textbf{IV}}{\leq} 0 = \llbracket \varphi \rrbracket^{\beta'}$ 
\begin{itemize}
\item $\beta(X_{n+1}) = 0 \leq 0 = \beta'(X_{n+1})$ \\
$\llbracket (\varphi \land X_{n+1}) \rrbracket^{\beta} = 0 \leq 0 = \llbracket (\varphi \land X_{n+1})  \rrbracket^{\beta'} \Rightarrow \llbracket (\varphi \land X_{n+1})  \rrbracket^{\beta} \leq \llbracket (\varphi \land X_{n+1})  \rrbracket^{\beta'} $
\item $\beta(X_{n+1}) = 0 \leq 1 = \beta'(X_{n+1})$ \\
$\llbracket (\varphi \land X_{n+1}) \rrbracket^{\beta} = 0 \leq 0 = \llbracket (\varphi \land X_{n+1})  \rrbracket^{\beta'} \Rightarrow \llbracket (\varphi \land X_{n+1})  \rrbracket^{\beta} \leq \llbracket (\varphi \land X_{n+1})  \rrbracket^{\beta'} $
\item $\beta(X_{n+1}) = 1 \leq 1 = \beta'(X_{n+1})$ \\
$\llbracket (\varphi \land X_{n+1}) \rrbracket^{\beta} = 0 \leq 0 = \llbracket (\varphi \land X_{n+1})  \rrbracket^{\beta'} \Rightarrow \llbracket (\varphi \land X_{n+1})  \rrbracket^{\beta} \leq \llbracket (\varphi \land X_{n+1})  \rrbracket^{\beta'} $
\end{itemize}
\item $\varphi \land X_{n+1}$ mit $\llbracket \varphi \rrbracket^{\beta} = 0 \stackrel{\textbf{IV}}{\leq} 1 = \llbracket \varphi \rrbracket^{\beta'}$ 
\begin{itemize}
\item $\beta(X_{n+1}) = 0 \leq 0 = \beta'(X_{n+1})$ \\
$\llbracket (\varphi \land X_{n+1}) \rrbracket^{\beta} = 0 \leq 0 = \llbracket (\varphi \land X_{n+1})  \rrbracket^{\beta'} \Rightarrow \llbracket (\varphi \land X_{n+1})  \rrbracket^{\beta} \leq \llbracket (\varphi \land X_{n+1})  \rrbracket^{\beta'} $
\item $\beta(X_{n+1}) = 0 \leq 1 = \beta'(X_{n+1})$ \\
$\llbracket (\varphi \land X_{n+1}) \rrbracket^{\beta} = 0 \leq 1 = \llbracket (\varphi \land X_{n+1})  \rrbracket^{\beta'} \Rightarrow \llbracket (\varphi \land X_{n+1})  \rrbracket^{\beta} \leq \llbracket (\varphi \land X_{n+1})  \rrbracket^{\beta'} $
\item $\beta(X_{n+1}) = 1 \leq 1 = \beta'(X_{n+1})$ \\
$\llbracket (\varphi \land X_{n+1}) \rrbracket^{\beta} = 0 \leq 1 = \llbracket (\varphi \land X_{n+1})  \rrbracket^{\beta'} \Rightarrow \llbracket (\varphi \land X_{n+1})  \rrbracket^{\beta} \leq \llbracket (\varphi \land X_{n+1})  \rrbracket^{\beta'} $
\end{itemize}
\item $\varphi \land X_{n+1}$ mit $\llbracket \varphi \rrbracket^{\beta} = 1 \stackrel{\textbf{IV}}{\leq} 1 = \llbracket \varphi \rrbracket^{\beta'}$ 
\begin{itemize}
\item $\beta(X_{n+1}) = 0 \leq 0 = \beta'(X_{n+1})$ \\
$\llbracket (\varphi \land X_{n+1}) \rrbracket^{\beta} = 1 \leq 1 = \llbracket (\varphi \land X_{n+1})  \rrbracket^{\beta'} \Rightarrow \llbracket (\varphi \land X_{n+1})  \rrbracket^{\beta} \leq \llbracket (\varphi \land X_{n+1})  \rrbracket^{\beta'} $
\item $\beta(X_{n+1}) = 0 \leq 1 = \beta'(X_{n+1})$ \\
$\llbracket (\varphi \land X_{n+1}) \rrbracket^{\beta} = 1 \leq 1 = \llbracket (\varphi \land X_{n+1})  \rrbracket^{\beta'} \Rightarrow \llbracket (\varphi \land X_{n+1})  \rrbracket^{\beta} \leq \llbracket (\varphi \land X_{n+1})  \rrbracket^{\beta'} $
\item $\beta(X_{n+1}) = 1 \leq 1 = \beta'(X_{n+1})$ \\
$\llbracket (\varphi \land X_{n+1}) \rrbracket^{\beta} = 1 \leq 1 = \llbracket (\varphi \land X_{n+1})  \rrbracket^{\beta'} \Rightarrow \llbracket (\varphi \land X_{n+1})  \rrbracket^{\beta} \leq \llbracket (\varphi \land X_{n+1})  \rrbracket^{\beta'} $
\end{itemize}
\item $\varphi \lor X_{n+1}$ mit $\llbracket \varphi \rrbracket^{\beta} = 0 \stackrel{\textbf{IV}}{\leq} 0 = \llbracket \varphi \rrbracket^{\beta'}$  \\
Analog
\item $\varphi \lor X_{n+1}$ mit $\llbracket \varphi \rrbracket^{\beta} = 0 \stackrel{\textbf{IV}}{\leq} 1 = \llbracket \varphi \rrbracket^{\beta'}$ \\
Anolog
\item $\varphi \lor X_{n+1}$ mit $\llbracket \varphi \rrbracket^{\beta} = 1 \stackrel{\textbf{IV}}{\leq} 1 = \llbracket \varphi \rrbracket^{\beta'}$  \\
Analog
\end{itemize}
\end{itemize}
\subsection*{Aufgabe 2}
\begin{enumerate}[(i)]
\item Es ist zu zeigen, dass \textsf{NAND} funktional vollständig ist. Bisher kennen wir $\{\lnot, \lor, \land\}$ als funktional vollständig. Wenn nun alle Operatoren dieser Basis durch \textsf{NAND} dargestellt werden können, ist \textsf{NAND} funktional vollständig.
\begin{itemize}
\item $\lnot\phi \equiv (\phi \textsf{ NAND } \phi)$ \\
\\
\begin{math}
\begin{array}{c|cc}
\toprule 
\phi & \lnot \phi & (\phi \textsf{ NAND } \phi) \\
\midrule
0 & 1 & 1 \\
1 & 0 & 0 \\
\bottomrule
\end{array}
\end{math}
\item $(\phi \lor \psi) \equiv ((\phi \textsf{ NAND } \phi) \textsf{NAND} (\psi \textsf{ NAND } \psi))$ \\
\\
\begin{math}
\begin{array}{cc|cc}
\toprule 
\phi & \psi & (\phi \lor  \psi) & ((\phi \textsf{ NAND } \phi) \textsf{NAND} (\psi \textsf{ NAND } \psi)) \\
\midrule
0 & 0 & 0 & 0 \\
0 & 1 & 1 & 1 \\
1 & 0 & 1 & 1 \\
1 & 1 & 1 & 1 \\
\bottomrule
\end{array}
\end{math}
\item $(\phi \lor \psi) \equiv ((\phi \textsf{ NAND } \phi) \textsf{NAND} (\psi \textsf{ NAND } \psi))$ \\
\\
\begin{math}
\begin{array}{cc|cc}
\toprule 
\phi & \psi & (\phi \land  \psi) & ((\phi \textsf{ NAND } \psi) \textsf{NAND} (\phi \textsf{ NAND } \psi)) \\
\midrule
0 & 0 & 0 & 0 \\
0 & 1 & 0 & 0 \\
1 & 0 & 0 & 0 \\
1 & 1 & 1 & 1 \\
\bottomrule
\end{array}
\end{math}
\end{itemize}
Da alle Operatoren der uns bekannten funktional vollständigen Junktorbasis durch \textsf{NAND} darstellbar ist, ist \textsf{NAND} funktional vollständig.
\item $\{\land, \lor, \rightarrow \}$ ist nicht funktional vollständig, da $\lnot X$ mit $\{\land, \lor, \rightarrow \}$  nicht darstellbar ist. Sei dazu die Belegung $\beta$ definiert als $\beta(X) = 1$.
Beweis per struktureller Induktion:
\begin{itemize}
\item[\textbf{IA}] $\phi = X$ \\
$\llbracket X \rrbracket^{\beta} = 1$
\item[\textbf{IV}] Seien $\phi_1, \phi_2$ Formeln die nur aus   $\{\land, \lor, \rightarrow \}$ bestehen, wobei gilt, dass $\llbracket \phi_1 \rrbracket^{\beta} = 1 = \llbracket \phi_2 \rrbracket^{\beta} $
\item[\textbf{IS}] 
\begin{itemize}
\item $\psi = (\phi_1 \land \phi_2)$ \\
$\llbracket  (\phi_1 \land \phi_2) \rrbracket^\beta = 1$
\item $\psi = (\phi_1 \lor \phi_2)$ \\
$\llbracket  (\phi_1 \lor\phi_2) \rrbracket^\beta = 1$
\item $\psi = (\phi_1 \rightarrow \phi_2)$ \\
$\llbracket  (\phi_1 \rightarrow \phi_2) \rrbracket^\beta = 1$
\end{itemize}
\end{itemize}
Damit ist bewiesen, dass die Junktorbasis $\{\land, \lor, \rightarrow \}$ nicht funktional vollständig ist.
\end{enumerate}

\subsection*{Aufgabe 3}

\begin{enumerate}[(i)]
\item

\item

\end{enumerate}

\subsection*{Aufgabe 4}

\end{document}