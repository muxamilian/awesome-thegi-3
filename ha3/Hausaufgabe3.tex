\documentclass[a4paper,10pt]{article}
\usepackage[utf8]{inputenc}
\usepackage{amsmath}
\usepackage{amsfonts}
\usepackage{amssymb}
\usepackage{fullpage}
\usepackage[german]{babel}
\setlength{\parindent}{0cm}
\usepackage{setspace}
\usepackage{mathpazo}
\usepackage{graphicx}
\usepackage{wasysym} 
\usepackage{booktabs}
\usepackage{verbatim}
\usepackage{enumerate}
\usepackage{ulem}
\usepackage{stmaryrd}
\usepackage[a4paper,
left=1.8cm, right=1.8cm,
top=2.0cm, bottom=2.0cm]{geometry}


\begin{document}

\begin{center}
\Large{Theoretische Grundlagen der Informatik 3: Hausaufgabenabgabe 2} \\
\large{Tutorium: Sebastian , Mi 14.00 - 16.00 Uhr}
\end{center}
\begin{tabbing}
Tom Nick \hspace{2cm}\= - 340528\\
Maximillian Bachl \> - 341455 \\
Marius Liwotto\> -  341051
\end{tabbing}
\section*{Aufgabe 1}
\subsection*{Hinrichtung}
Beweis durch strukturelle Induktion. Seien dazu $\beta$ und $\beta'$ zwei Belegungen mit $\beta \leq \beta'$.
\begin{itemize}
\item[\textbf{IA}]
$\varphi$ ist eine monotone Formel ohne einen einzigen Junktor. Dann gibt es nur folgende drei Möglichkeiten:
\begin{itemize}
\item  $\varphi = \bot$
\item  $\varphi = \top$
\item  $\varphi = X_1$
\end{itemize}
\item[\textbf{IV}]
$\varphi$ ist eine monotone Formel. Sie ist besteht nur aus $X_1,...,X_m$ sowie $\bot$, $\top$, $\land$ und $\lor$.
\item[\textbf{IS}] 
Das Anhängen von $\top$ und $\bot$ funktioniert, da es keinen Einfluss auf die Monotonie der Formeln hat (Da die Belegung nur Einfluss auf Variablen hat)
\begin{itemize}
\item $\lnot \varphi$ ist nicht monoton, da $\llbracket \varphi \rrbracket^{\beta} \le \llbracket \varphi \rrbracket^{\beta'} \rightarrow \llbracket \lnot\varphi \rrbracket^{\beta} \ge \llbracket \lnot\varphi \rrbracket^{\beta'}$. Daher kann die Formel nicht negiert werden.
\item $\varphi \land \lnot X_k$ ist ebenfalls nicht monoton, da wenn $\beta(X_k) = 0 \land\beta'(X_k) = 1$ ist, $\llbracket \varphi \land \lnot X_k \rrbracket^{\beta} \ge \llbracket \varphi \land \lnot X_k \rrbracket^{\beta'}$. 
\item $\varphi \lor \lnot X_k$ ist ebenfalls nicht monoton, da wenn $\beta(X_k) = 0 \land\beta'(X_k) = 1$ ist, $\llbracket \varphi \lor \lnot X_k \rrbracket^{\beta} \ge \llbracket \varphi \lor \lnot X_k \rrbracket^{\beta'}$. 
\end{itemize}
Somit kann an $\varphi$ nur einer der in der Angabe aufgezählten Bausteine erweitert werden, damit die Monotonie nicht verletzt wird.
\end{itemize}
\subsection*{Rückrichtung}
Beweis durch strukturelle Induktion. Seien dazu $\beta$ und $\beta'$ zwei Belegungen mit $\beta \leq \beta'$, sowie $\varphi$ eine Formeln die nur aus Variablen $X_1,...,X_n$ und $\bot,\top,\land,\lor$ besteht.
\begin{itemize}
\item[\textbf{IA}] $\varphi = X_1$ \\ 
Es gibt drei Möglichkeiten für $\beta$ und $\beta'$, sodass $\beta \leq \beta'$.
\begin{itemize}
\item $\beta(X_1) = 0 \leq  0 = \beta'(X_1)$ \\
$\llbracket \varphi \rrbracket^{\beta} = 0 \leq 0 = \llbracket \varphi \rrbracket^{\beta'} \Rightarrow \llbracket \varphi \rrbracket^{\beta} \leq \llbracket \varphi \rrbracket^{\beta'} $
\item Die anderen Fälle sind analog.
\end{itemize}
\item[\textbf{IV}] Sei $\varphi$ eine monotone Formel, wobei diese nur aus Variablen $X_1,...,X_n$ und $\bot,\top,\land,\lor$ besteht.
\item[\textbf{IS}] Da das Anhängen von $\top$ und $\bot$ mittels $\land, \lor$ keinerlei Einfluss auf die Monotonie von Formeln hat (Da die Belegung nur Einfluss auf Variablen hat), werden diese hier nicht betrachtet. Es ist nun zu zeigen, dass die Aussage auch für $\varphi \land X_{n+1}$ und $\varphi \lor X_{n+1}$ gilt.
\begin{itemize}
\item $\varphi \land X_{n+1}$ mit $\llbracket \varphi \rrbracket^{\beta} = 0 \stackrel{\textbf{IV}}{\leq} 0 = \llbracket \varphi \rrbracket^{\beta'}$ 
\begin{itemize}
\item $\beta(X_{n+1}) = 0 \leq 0 = \beta'(X_{n+1})$ \\
$\llbracket (\varphi \land X_{n+1}) \rrbracket^{\beta} = 0 \leq 0 = \llbracket (\varphi \land X_{n+1})  \rrbracket^{\beta'} \Rightarrow \llbracket (\varphi \land X_{n+1})  \rrbracket^{\beta} \leq \llbracket (\varphi \land X_{n+1})  \rrbracket^{\beta'} $
\item $\beta(X_{n+1}) = 0 \leq 1 = \beta'(X_{n+1})$ \\
$\llbracket (\varphi \land X_{n+1}) \rrbracket^{\beta} = 0 \leq 0 = \llbracket (\varphi \land X_{n+1})  \rrbracket^{\beta'} \Rightarrow \llbracket (\varphi \land X_{n+1})  \rrbracket^{\beta} \leq \llbracket (\varphi \land X_{n+1})  \rrbracket^{\beta'} $
\item $\beta(X_{n+1}) = 1 \leq 1 = \beta'(X_{n+1})$ \\
$\llbracket (\varphi \land X_{n+1}) \rrbracket^{\beta} = 0 \leq 0 = \llbracket (\varphi \land X_{n+1})  \rrbracket^{\beta'} \Rightarrow \llbracket (\varphi \land X_{n+1})  \rrbracket^{\beta} \leq \llbracket (\varphi \land X_{n+1})  \rrbracket^{\beta'} $
\end{itemize}
\item $\varphi \land X_{n+1}$ mit $\llbracket \varphi \rrbracket^{\beta} = 0 \stackrel{\textbf{IV}}{\leq} 1 = \llbracket \varphi \rrbracket^{\beta'}$ 
\begin{itemize}
\item $\beta(X_{n+1}) = 0 \leq 0 = \beta'(X_{n+1})$ \\
$\llbracket (\varphi \land X_{n+1}) \rrbracket^{\beta} = 0 \leq 0 = \llbracket (\varphi \land X_{n+1})  \rrbracket^{\beta'} \Rightarrow \llbracket (\varphi \land X_{n+1})  \rrbracket^{\beta} \leq \llbracket (\varphi \land X_{n+1})  \rrbracket^{\beta'} $
\item $\beta(X_{n+1}) = 0 \leq 1 = \beta'(X_{n+1})$ \\
$\llbracket (\varphi \land X_{n+1}) \rrbracket^{\beta} = 0 \leq 1 = \llbracket (\varphi \land X_{n+1})  \rrbracket^{\beta'} \Rightarrow \llbracket (\varphi \land X_{n+1})  \rrbracket^{\beta} \leq \llbracket (\varphi \land X_{n+1})  \rrbracket^{\beta'} $
\item $\beta(X_{n+1}) = 1 \leq 1 = \beta'(X_{n+1})$ \\
$\llbracket (\varphi \land X_{n+1}) \rrbracket^{\beta} = 0 \leq 1 = \llbracket (\varphi \land X_{n+1})  \rrbracket^{\beta'} \Rightarrow \llbracket (\varphi \land X_{n+1})  \rrbracket^{\beta} \leq \llbracket (\varphi \land X_{n+1})  \rrbracket^{\beta'} $
\end{itemize}
\item $\varphi \land X_{n+1}$ mit $\llbracket \varphi \rrbracket^{\beta} = 1 \stackrel{\textbf{IV}}{\leq} 1 = \llbracket \varphi \rrbracket^{\beta'}$ 
\begin{itemize}
\item $\beta(X_{n+1}) = 0 \leq 0 = \beta'(X_{n+1})$ \\
$\llbracket (\varphi \land X_{n+1}) \rrbracket^{\beta} = 1 \leq 1 = \llbracket (\varphi \land X_{n+1})  \rrbracket^{\beta'} \Rightarrow \llbracket (\varphi \land X_{n+1})  \rrbracket^{\beta} \leq \llbracket (\varphi \land X_{n+1})  \rrbracket^{\beta'} $
\item $\beta(X_{n+1}) = 0 \leq 1 = \beta'(X_{n+1})$ \\
$\llbracket (\varphi \land X_{n+1}) \rrbracket^{\beta} = 1 \leq 1 = \llbracket (\varphi \land X_{n+1})  \rrbracket^{\beta'} \Rightarrow \llbracket (\varphi \land X_{n+1})  \rrbracket^{\beta} \leq \llbracket (\varphi \land X_{n+1})  \rrbracket^{\beta'} $
\item $\beta(X_{n+1}) = 1 \leq 1 = \beta'(X_{n+1})$ \\
$\llbracket (\varphi \land X_{n+1}) \rrbracket^{\beta} = 1 \leq 1 = \llbracket (\varphi \land X_{n+1})  \rrbracket^{\beta'} \Rightarrow \llbracket (\varphi \land X_{n+1})  \rrbracket^{\beta} \leq \llbracket (\varphi \land X_{n+1})  \rrbracket^{\beta'} $
\end{itemize}
\item $\varphi \lor X_{n+1}$ mit $\llbracket \varphi \rrbracket^{\beta} = 0 \stackrel{\textbf{IV}}{\leq} 0 = \llbracket \varphi \rrbracket^{\beta'}$  \\
Analog
\item $\varphi \lor X_{n+1}$ mit $\llbracket \varphi \rrbracket^{\beta} = 0 \stackrel{\textbf{IV}}{\leq} 1 = \llbracket \varphi \rrbracket^{\beta'}$ \\
Anolog
\item $\varphi \lor X_{n+1}$ mit $\llbracket \varphi \rrbracket^{\beta} = 1 \stackrel{\textbf{IV}}{\leq} 1 = \llbracket \varphi \rrbracket^{\beta'}$  \\
Analog
\end{itemize}
\end{itemize}
\section*{Aufgabe 2}
\begin{enumerate}[(i)]
\item Es ist zu zeigen, dass \textsf{NAND} funktional vollständig ist. Bisher kennen wir $\{\lnot, \lor, \land\}$ als funktional vollständig. Wenn nun alle Operatoren dieser Basis durch \textsf{NAND} dargestellt werden können, ist \textsf{NAND} funktional vollständig.
\begin{itemize}
\item $\lnot\phi \equiv (\phi \textsf{ NAND } \phi)$ \\
\\
\begin{math}
\begin{array}{c|cc}
\toprule 
\phi & \lnot \phi & (\phi \textsf{ NAND } \phi) \\
\midrule
0 & 1 & 1 \\
1 & 0 & 0 \\
\bottomrule
\end{array}
\end{math}
\\
\item $(\phi \land \psi) \equiv ((\phi \textsf{ NAND } \psi) \textsf{NAND} (\phi \textsf{ NAND } \psi))$\\
\\
\begin{math}
\begin{array}{cc|cc}
\toprule 
\phi & \psi & (\phi \land  \psi) & ((\phi \textsf{ NAND } \psi) \textsf{NAND} (\phi \textsf{ NAND } \psi))  \\
\midrule
0 & 0 & 0 & 0 \\
0 & 1 & 0 & 0 \\
1 & 0 & 0 & 0 \\
1 & 1 & 1 & 1 \\
\bottomrule
\end{array}
\end{math}
\\
\item $(\phi \lor \psi) \equiv (\phi \textsf{ NAND } \phi) \textsf{NAND} (\psi \textsf{ NAND } \psi))$\\
\\
\begin{math}
\begin{array}{cc|cc}
\toprule 
\phi & \psi & (\phi \lor  \psi) & ((\phi \textsf{ NAND } \phi) \textsf{NAND} (\psi \textsf{ NAND } \psi)) \\
\midrule
0 & 0 & 0 & 0 \\
0 & 1 & 1 & 1 \\
1 & 0 & 1 & 1 \\
1 & 1 & 1 & 1 \\
\bottomrule
\end{array}
\end{math}
\end{itemize}
Da alle Operatoren der uns bekannten funktional vollständigen Junktorbasis durch \textsf{NAND} darstellbar ist, ist \textsf{NAND} funktional vollständig.
\item $\{\land, \lor, \rightarrow \}$ ist nicht funktional vollständig, da $\lnot X$ mit $\{\land, \lor, \rightarrow \}$  nicht darstellbar ist. Sei dazu die Belegung $\beta$ definiert als $\beta(X) = 1$.
Beweis per struktureller Induktion:
\begin{itemize}
\item[\textbf{IA}] $\phi = X$ \\
$\llbracket X \rrbracket^{\beta} = 1$
\item[\textbf{IV}] Seien $\phi_1, \phi_2$ Formeln die nur aus   $\{\land, \lor, \rightarrow \}$ bestehen, wobei gilt, dass $\llbracket \phi_1 \rrbracket^{\beta} = 1 = \llbracket \phi_2 \rrbracket^{\beta} $
\item[\textbf{IS}] 
\begin{itemize}
\item $\psi = (\phi_1 \land \phi_2)$ \\
$\llbracket  (\phi_1 \land \phi_2) \rrbracket^\beta = 1$
\item $\psi = (\phi_1 \lor \phi_2)$ \\
$\llbracket  (\phi_1 \lor\phi_2) \rrbracket^\beta = 1$
\item $\psi = (\phi_1 \rightarrow \phi_2)$ \\
$\llbracket  (\phi_1 \rightarrow \phi_2) \rrbracket^\beta = 1$
\end{itemize}
\end{itemize}
Damit ist bewiesen, dass die Junktorbasis $\{\land, \lor, \rightarrow \}$ nicht funktional vollständig ist.
\end{enumerate}

\section*{Aufgabe 3}

\begin{enumerate}[(i)]
\item Sei $\varphi_n$ induktiv definiert als:
\begin{align*}
\varphi_1(X_1) &= X_1 \\
\varphi_{n+1}(X_1,...,X_{n+1}) &= \varphi_n(X_1,...,X_n) \oplus X_{n+1}
\end{align*}
Wie im Tutorium 2 bewiesen wurde, wechselt $\varphi$ immer den Wahrheitswert, wenn sich in der Belegung eine Zahl ändert. Damit gibt es genau $2^{n-1}$ Belegungen zu denen $\varphi$ zu wahr auswertet. Somit hätte die kanonische KNF $2^{n-1}$ Klauseln. Diese lassen sich jedoch \textbf{nicht} zusammenfassen, da sich in jeder Klausel immer mindestens 2 Variablen unterscheiden (wobei $X$ und $\lnot X$ als unterschiedlich gelten) . Dies entsteht wie schon erwähnt dadurch, dass sich der Wahrheitswert ändert, sobald eine Variable ihren Wert ändert. Somit müssen sich immer mindestens 2 Variablen jeder Belegung unterscheiden damit die KNF zu \textsf{true} auswertet. Wollte man die Aussage beweisen, müsste man zeigen, dass sich Klauseln mit mindestens 2 unterschiedlichen Variablen nicht zusammenfassen lassen. Siehe Satz von Quine-McClusky.

 Dadurch hat die minimale KNF $2^{n-1}$ Klauseln mit jeweils $n$ Variablen in jeder Klausel. Das ist offensichtlich exponentiell.

\item Die Formel, sowie die Begründung von 3i kann übernommen wurden.
\end{enumerate}

\section*{Aufgabe 4}
\subsection*{Rückrichtung}
Annahme: Für jedes $n \ge 0$ existiert eine Parkettierung.

Wir definieren für jeden Platz am Parkett die folgende Variable:
\begin{align*} X_{i,j} &:= (P(i,j)_1 = P(i-1,j)_3) \land (P(i,j)_2 = P(i,j+1)_4) \\ &\land (P(i,j)_3 = P(i+1,j)_1) \land (P(i,j)_4 = P(i,j-1)_2)\end{align*}
wobei der Vergleich weggelassen wird, falls $P(k,l)$ nicht definiert ist. Sei nun $\Phi$ definiert als: 
$$\Phi = \{X_{i,j} \mid i,j \in \mathbb{Z} \}$$
Zu zeigen $\Phi$ ist erfüllt. Sei $\mathbb{Z}_n \times \mathbb{Z}_n \subseteq \mathbb{Z} \times \mathbb{Z}$. Nun definiere $\Phi_0$ als: $$\Phi_0 = \{\textsf{proper-surrounding}(i,j) \mid i,j \in \mathbb{Z}_n \}$$ 
Offensichtlich gilt: $\Phi_0 \subseteq \Phi$. Die Parkettierung definiert eine erfüllbare Belegung von $\Phi_0$. Nach dem Kompaktheitssatz existiert auch eine erfüllbare Belegung von $\Phi$. 
\subsection*{Hinrichtung}
Die Hinrichtung ist trivial.


\end{document}