\documentclass[a4paper,10pt]{article}
\usepackage[utf8]{inputenc}
\usepackage{amsmath}
\usepackage{amsfonts}
\usepackage{amssymb}
\usepackage{fullpage}
\usepackage[german]{babel}
\setlength{\parindent}{0cm}
\usepackage{setspace}
\usepackage{mathpazo}
\usepackage{graphicx}
\usepackage{wasysym} 
\usepackage{booktabs}
\usepackage{verbatim}
\usepackage{enumerate}
\usepackage{ulem}
\usepackage{ stmaryrd }
\usepackage[a4paper,
left=1.8cm, right=1.8cm,
top=2.0cm, bottom=2.0cm]{geometry}


\begin{document}

\begin{center}
\Large{Theoretische Grundlagen der Informatik 3: Hausaufgabenabgabe 2} \\
\large{Tutorium: Sebastian , Mi 14.00 - 16.00 Uhr}
\end{center}
\begin{tabbing}
Tom Nick \hspace{2cm}\= - 340528\\
Maximillian Bachl \> - 341455 \\
Marius Liwotto\> -  341051
\end{tabbing}
\subsection*{Aufgabe 1}

\subsection*{Aufgabe 2}

\begin{enumerate}[(i)]
\item

\item

\end{enumerate}

\subsection*{Aufgabe 3}

\begin{enumerate}[(i)]
\item

\item

\end{enumerate}

\subsection*{Aufgabe 4}

\end{document}