\documentclass[a4paper,10pt]{article}
\usepackage[utf8]{inputenc}
\usepackage{amsmath}
\usepackage{amsfonts}
\usepackage{amssymb}
\usepackage[german]{babel}
\setlength{\parindent}{0cm}
\usepackage{setspace}
\usepackage{mathpazo}
\usepackage{graphicx}
\usepackage{wasysym} 
\usepackage{booktabs}
\usepackage{verbatim}
\usepackage{enumerate}
\usepackage{hyperref}
\usepackage{ulem}
\usepackage{stmaryrd }
\usepackage[a4paper,
left=1.8cm, right=1.8cm,
top=2.0cm, bottom=2.0cm]{geometry}
\usepackage{tabularx}

\newcommand{\rf}{\right\rfloor}
\newcommand{\lf}{\left\lfloor}
\newcommand{\tabspace}{15cm}
\newcommand{\N}{\mathbb{N}}
\newcommand{\Z}{\mathbb{Z}}

\begin{document}
\begin{center}
\Large{Theoretische Grundlagen der Informatik 3: Hausaufgabenabgabe 11} \\
\large{Tutorium: Sebastian , Mi 14.00 - 16.00 Uhr}
\end{center}
\begin{tabbing}
Tom Nick \hspace{2cm}\= - 340528\\
Maximillian Bachl \> - 341455 \\
Marius Liwotto\> -  341051
\end{tabbing}
\subsection*{Aufgabe 1}
\begin{enumerate}[(i)]
\item
Sei $n$ die Anzahl an Elementen im Universum von $G$. Ein $H$ muss für einen Isomorphismus auf jeden Fall die gleiche Anzahl an Elementen haben, wie $G$, nämlich $n$.

Wir führen die Variable $E_{i,j}$ für jede Kante $E^G(i,j)$ ein, wobei $1\le i,j \le n$. Es muss gelten $E_{i,j} \equiv E^G(i,j)$.

Wir konstruieren folgende Formel:
\begin{align*}
\varphi &:= \exists y_1 ... \exists y_n~ \left(\left(y_1 \neq y_2 \land ... \land  y_1 \neq y_n\right) \land ... \land \left(y_{n-1} \neq y_n\right)\right) \\
&\land \left(\left( E^H(y_1,y_2) \leftrightarrow E_{1,2} \land ... \land E^H(y_1,y_n) \leftrightarrow E_{1,n} \right) \land \left( E^H(y_{n-1},y_n) \leftrightarrow E_{n-1,n} \right) \right)
\end{align*}

Der Satz stellt sicher, dass alle $x_1$ bis $x_n$ ungleich gewählt sind und sie genau dann in Relation zueinander stehen, wenn sie dies auch im Graphen $G$ taten.
\item
Es muss eine Menge von Sätzen $\Phi$ oder ein Satz $\xi$ gefunden werden, sodass $\mathcal C = \textsf{Mod}(\Phi)$ oder $\mathcal C = \textsf{Mod}(\xi)$.

Wir definieren für jeden Graphen $G_i$ die folgende Formel:
\begin{align*}
\psi_i &:= \bigvee_{G' \subset G_i} \varphi_{G'} \text{ , wobei $\varphi_{G'}$ die Formel aus (i) für den Untergraph $G'$ ist.}
\end{align*}
$\psi_i$ sagt also aus, ob $H$ isomorph zu einem Teilgraphen von $G_i$ ist.

Ferner definieren wir folgende Formel:
\begin{align*}
\xi &:= \bigvee_{i \in \{1,...,k\}} \psi_i
\end{align*}
Diese Formel verodert die vorhin definierten $\psi_i$. Sie sag also aus, ob H zu einem Subgraphen eines der Graphen $G_1,...,G_k$ isomorph ist.

Da wir hiermit ein endliches Axiomensystem -- nämlich $\xi$ -- aufgestellt haben, ist gewiss, dass $\mathcal C$ endlich axiomatisierbar ist.

$\xi$ ist ein endliches Axiomensystem, da alle $\psi_i$ und auch $\varphi$ aus (i) für endliche Graphen trivialerweise stets endlich sind. 
\end{enumerate}
\end{document}