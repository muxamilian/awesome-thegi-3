\documentclass[a4paper,10pt]{article}
\usepackage[utf8]{inputenc}
\usepackage{amsmath}
\usepackage{amsfonts}
\usepackage{amssymb}
\usepackage[german]{babel}
\setlength{\parindent}{0cm}
\usepackage{setspace}
\usepackage{mathpazo}
\usepackage{graphicx}
\usepackage{wasysym} 
\usepackage{booktabs}
\usepackage{verbatim}
\usepackage{enumerate}
\usepackage{hyperref}
\usepackage{ulem}
\usepackage{stmaryrd }
\usepackage[a4paper,
left=1.8cm, right=1.8cm,
top=2.0cm, bottom=2.0cm]{geometry}
\usepackage{tabularx}

\newcommand{\rf}{\right\rfloor}
\newcommand{\lf}{\left\lfloor}
\newcommand{\tabspace}{15cm}
\newcommand{\N}{\mathbb{N}}
\newcommand{\Z}{\mathbb{Z}}

\begin{document}
\begin{center}
\Large{Theoretische Grundlagen der Informatik 3: Hausaufgabenabgabe 11} \\
\large{Tutorium: Sebastian , Mi 14.00 - 16.00 Uhr}
\end{center}
\begin{tabbing}
Tom Nick \hspace{2cm}\= - 340528\\
Maximillian Bachl \> - 341455 \\
Marius Liwotto\> -  341051
\end{tabbing}
\subsection*{Aufgabe 1}
\begin{enumerate}[(i)]
	\item Sei der Satz $\varphi$ definiert als:
	\[ \varphi = \bigwedge_{(a,b) \in E(G)} (a,b) \]
	\item 
\end{enumerate}

\subsection*{Aufgabe 2}
Es wurde in den Präzensübungen gezeigt, dass die Duplikatorin das EF Spiel zwischen $(\mathbb{Q},<)$ und $(\mathbb{R},<)$ immer gewinnt, d.h. diese Strukturen sind Elementar äquivalent.
Somit kann es keine Menge $\Phi$ an $FO[\sigma]$ geben, sodass Mod($\Phi$) genau die Klasse aller zu $(\mathbb{Q},<)$ isomorphen Mengen ist, da $(\mathbb{R},<)$ nicht isomorph ist.
\subsection*{Aufgabe 3}
\begin{enumerate}[(i)]
	\item Es wird im folgenden widerlegt, dass $T$ eine vollständige Theorie ist.
	\item 
\end{enumerate}

\subsection*{Aufgabe 4}
Die Struktur der unendlichen $\sigma$-Strukturen ist axiomatisierbar mit:
\begin{align*}
	\Phi = \{\varphi_n | n \in \N \} \text{ wobei } \varphi_n = \exists x_1 ... \exists x_n. \bigwedge_{1  \le i \le n} \bigwedge_{1 \le j \le, i \neq j} x_i \neq x_j
\end{align*}
Falls es ein endliches $\Phi$ geben sollte heisst das man könnte die Konjunktion über $\Phi$ bilden:
\[\varphi = \bigwedge_{\psi \in \Phi} \psi \]
Da $\varphi$ endlich ist kann man den Quantorenrang bestimmen: $m = qr(\varphi)$. \\
\\
Wähle zwei $\sigma$-Strukturen $(\mathcal{A} = (A,<),\mathcal{B} = (B,<))$, wobei die Menge $A$ die grösse $2^{m+1}$ hat und $B$ unendlich ist. Würden wir nun eine EF-Spiel auf diesen Strukturen spielen, würde die Duplikatorin das $m$-Runden Spiel gewinnen (siehe Satz aus der Vorlesung), daraus folgt die $m$-Äquivalenz zwischen diesen Strukturen, d.h. f.a. $\varphi \in FO[\sigma] $ mit $qr(\varphi) = m$ gilt $\mathcal{A} \vDash \varphi \Leftrightarrow \mathcal{B} \vDash \varphi$. Somit kann es kein endliches Axiomsystem geben, dass die Menge der undendlichen Mengen axiomatisiert.


\end{document}