\documentclass[a4paper,10pt]{article}
\usepackage[utf8]{inputenc}
\usepackage{amsmath}
\usepackage{amsfonts}
\usepackage{amssymb}
\usepackage[german]{babel}
\setlength{\parindent}{0cm}
\usepackage{setspace}
\usepackage{mathpazo}
\usepackage{graphicx}
\usepackage{wasysym} 
\usepackage{booktabs}
\usepackage{verbatim}
\usepackage{enumerate}
\usepackage{hyperref}
\usepackage{ulem}
\usepackage{stmaryrd }
\usepackage[a4paper,
left=1.8cm, right=1.8cm,
top=2.0cm, bottom=2.0cm]{geometry}
\usepackage{tabularx}

\newcommand{\rf}{\right\rfloor}
\newcommand{\lf}{\left\lfloor}
\newcommand{\tabspace}{15cm}
\newcommand{\N}{\mathbb{N}}
\newcommand{\Z}{\mathbb{Z}}

\begin{document}
\begin{center}
\Large{Theoretische Grundlagen der Informatik 3: Hausaufgabenabgabe 11} \\
\large{Tutorium: Sebastian , Mi 14.00 - 16.00 Uhr}
\end{center}
\begin{tabbing}
Tom Nick \hspace{2cm}\= - 340528\\
Maximillian Bachl \> - 341455 \\
Marius Liwotto\> -  341051
\end{tabbing}
\subsection*{Aufgabe 1}
\begin{enumerate}[(i)]
\item
Sei $n$ die Anzahl an Elementen im Universum von $G$. Ein $H$ muss für einen Isomorphismus auf jeden Fall die gleiche Anzahl an Elementen haben, wie $G$, nämlich $n$.

Wir führen die Variable $E_{i,j}$ für jede Kante $E^G(i,j)$ ein, wobei $1\le i,j \le n$. Es muss gelten $E_{i,j} \equiv E^G(i,j)$.

Wir konstruieren folgende Formel: 
\begin{align*}
\varphi &:= \exists y_1 ... \exists y_n~ \left(\left(y_1 \neq y_2 \land ... \land  y_1 \neq y_n\right) \land ... \land \left(y_{n-1} \neq y_n\right)\right) \\
&\land \left(\left( E^H(y_1,y_2) \leftrightarrow E_{1,2} \land ... \land E^H(y_1,y_n) \leftrightarrow E_{1,n} \right) \land \left( E^H(y_{n-1},y_n) \leftrightarrow E_{n-1,n} \right) \right)
\end{align*}

Der Satz stellt sicher, dass alle $x_1$ bis $x_n$ ungleich gewählt sind und sie genau dann in Relation zueinander stehen, wenn sie dies auch im Graphen $G$ taten.
\item
Es muss eine Menge von Sätzen $\Phi$ oder ein Satz $\xi$ gefunden werden, sodass $\mathcal C = \textsf{Mod}(\Phi)$ oder $\mathcal C = \textsf{Mod}(\xi)$.

Wir definieren für jeden Graphen $G_i$ die folgende Formel:
\begin{align*}
\psi_i &:= \bigvee_{G' \subset G_i} \varphi_{G'} \text{ , wobei $\varphi_{G'}$ die Formel aus (i) für den Untergraph $G'$ ist.}
\end{align*}
$\psi_i$ sagt also aus, ob $H$ isomorph zu einem Teilgraphen von $G_i$ ist.

Ferner definieren wir folgende Formel:
\begin{align*}
\xi &:= \bigvee_{i \in \{1,...,k\}} \psi_i
\end{align*}
Diese Formel verodert die vorhin definierten $\psi_i$. Sie sag also aus, ob H zu einem Subgraphen eines der Graphen $G_1,...,G_k$ isomorph ist.

Da wir hiermit ein endliches Axiomensystem -- nämlich $\xi$ -- aufgestellt haben, ist gewiss, dass $\mathcal C$ endlich axiomatisierbar ist.

$\xi$ ist ein endliches Axiomensystem, da alle $\psi_i$ und auch $\varphi$ aus (i) für endliche Graphen trivialerweise stets endlich sind. 

\section*{Aufgabe 2}
Es wurde in den Präsenzübungen gezeigt, dass die Duplikatorin das E.F.-Spiel zwischen $(\mathbb{Q},<)$ und $(\mathbb{R},<)$ immer gewinnt, d.h. diese Strukturen sind elementar äquivalent.
Somit kann es keine Menge $\Phi$ an $FO[\sigma]$ geben, sodass Mod($\Phi$) genau die Klasse aller zu $(\mathbb{Q},<)$ isomorphen Mengen ist, da $(\mathbb{R},<)$ nicht isomorph ist.
\section*{Aufgabe 3}
\begin{enumerate}[(i)]
\item Es wird im Folgenden widerlegt, dass $T$ eine vollständige Theorie ist.

Das Gegenbeispiel sei hierbei der folgende Satz:
\begin{align*}
\varphi &:= \exists x \forall y~ x < y
\end{align*}
Dieser Satz besagt, ob es ein kleinstes Element in der Relation gibt. \\

Für $\mathcal A := \{\mathbb{N},<\}$ -- welches eine lineare Ordnung darstellt -- gilt dieser Satz. Die Zahl $0$ ist hierbei das Element, welches $\varphi$ erfüllt.

Für $\mathcal B := \{\mathbb{Z},<\}$ -- welches ebenfalls eine lineare Ordnung darstellt -- gilt dieser Satz jedoch nicht. Wählt man für $y = x-1$ findet man immer ein noch kleineres Element.\\

Somit kann die Theorie nicht vollständig sein, da es einen Satz gibt, der manchmal gilt, manchmal aber auch nicht. 
\item Im folgenden zeigen wir, dass $\textsf{Th}(\mathcal{A})$ für eine Struktur $\mathcal{A}$ eine vollständige Theorie ist.

Angenommen es gäbe einen Satz $\psi$, sodass nicht gilt: $\psi \in T$ oder $\lnot \psi \in T$.
Das würde bedeuten, dass es einen Satz gibt, von dem nicht festgestellt werden kann, ob er in $\mathcal{A}$ gilt oder nicht... TO BE CONTINUED
\\ \textbf{Ich glaube, dass das vollständige Theorien sind, bin mir aber nicht ganz sicher -- Max}
\end{enumerate}

\section*{Aufgabe 4}
Die Struktur der unendlichen $\sigma$-Strukturen ist axiomatisierbar mit:
\begin{align*}
	\Phi = \{\varphi_n | n \in \N \} \text{ wobei } \varphi_n = \exists x_1 ... \exists x_n. \bigwedge_{1  \le i \le n} \bigwedge_{1 \le j \le, i \neq j} x_i \neq x_j
\end{align*}
Falls es ein endliches $\Phi$ geben sollte, heißt das man könnte die Konjunktion über $\Phi$ bilden:
\[\varphi = \bigwedge_{\psi \in \Phi} \psi \]
Da $\varphi$ endlich ist kann man den Quantorenrang bestimmen: $m = qr(\varphi)$. \\
\\
Wähle zwei $\sigma$-Strukturen $(\mathcal{A} = (A,<),\mathcal{B} = (B,<))$, wobei die Menge $A$ die grösse $2^{m+1}$ hat und $B$ unendlich ist. Würden wir nun eine EF-Spiel auf diesen Strukturen spielen, würde die Duplikatorin das $m$-Runden Spiel gewinnen (siehe Satz aus der Vorlesung), daraus folgt die $m$-Äquivalenz zwischen diesen Strukturen, d.h. f.a. $\varphi \in FO[\sigma] $ mit $qr(\varphi) = m$ gilt $\mathcal{A} \vDash \varphi \Leftrightarrow \mathcal{B} \vDash \varphi$. Somit kann es kein endliches Axiomensystem geben, dass die Menge der undendlichen Mengen axiomatisiert.

\end{enumerate}
\end{document}