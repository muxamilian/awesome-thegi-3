\documentclass{article}
\usepackage[ngerman]{babel}
\usepackage[utf8]{inputenc}
\usepackage[margin=2cm]{geometry}
\usepackage{amssymb} 
\usepackage{amsmath}
\usepackage{ulem}
\usepackage{mathabx}

\begin{document}
	\begin{Large}
	\section*{TheGi 3 HA 2}	
	\subsection*{Aufgabe 1}

	\( N_k = \{X_{ik}~|~(v_i,v_k) \in E \wedge v_i,v_k \in V \} \)
	
	\begin{enumerate}
	\item[(i)]	
	\( \varphi_n = \bigwedge\limits_{k = 1}^{n} (\bigvee\limits_{c = 1}^{\#(N_k)} X_{kc} \in  N_k) \)

	\item[(ii)]
	\( \phi(k) = \begin{cases}	
	(X_{ij} \wedge X_{i(j+1)}) \wedge (\neg X_{i(j+2)} \wedge ... \wedge \neg X_{i(j+m)}), & \#(N_k) \geq 2\\
	X_{ij}, & \#(N_k) = 1 \wedge X_{ij} \in N_k 	
	\end{cases} \)
	
	\ \\ \( \varphi_n = \bigwedge\limits_{k = 1}^{\#V} (\phi(k))
	\)
	
	\end{enumerate}
	
	\subsection*{Aufgabe 2}	
	
	\subsection*{Aufgabe 3}
	
	\( \varphi_1 = X \Rightarrow (Y \wedge Z) \\
	\equiv \neg X \vee (Y \wedge Z) \\
	\equiv (\neg X \vee Y) \wedge (\neg X \vee Z) \\
	\equiv (X \Rightarrow Y) \wedge (X \Rightarrow Z) \\
	\equiv \psi_1 \\
	\) 
	
	\ \\ \( \varphi_2 = (X \wedge Y \wedge Z) \Rightarrow Q \\
	\equiv \neg (X \wedge Y \wedge Z) \vee Q \\
	\equiv \neg X \vee \neg Y \vee \neg Z \vee Q \\
	\equiv (\neg X \vee (\neg Y \vee (\neg Z \vee Q))) \\
	\equiv (X \Rightarrow (Y \Rightarrow (Z \Rightarrow Q))) \\
	\equiv \psi_2 \\
	\) 
	
	\ \\ \( \varphi_3 = (X \wedge Y) \Rightarrow \neg (Z \Rightarrow X)  \\
	\equiv \neg (X \wedge Y) \vee \neg (\neg Z \vee X)  \\
	\equiv (\neg X \vee \neg Y) \vee (Z \wedge \neg X)  \\
	\equiv (\neg X \vee \neg Y \vee Z) \wedge (\neg X \vee \neg Y \vee \neg X)  \\
	\equiv (\neg X \vee \neg Y \vee Z) \wedge (\neg X \vee \neg Y)  \\
	\equiv (\neg X \vee \neg Y) \\
	\equiv (\neg X \vee \neg Y) \wedge T \\
	\equiv (\neg X \vee \neg Y) \wedge (\neg X \vee X) \\
	\equiv (\neg X \vee \neg Y) \wedge (Y \vee \neg X \vee X) \\
	\equiv (\neg X \vee \neg Y) \wedge (Y \vee X) \vee \neg X \\
	\equiv (X \Rightarrow \neg Y) \wedge (\neg Y \Rightarrow X) \vee \neg X \\
	\equiv \psi_3 \\
	\) 
	
	\ \\ \( \varphi_4 = (Y \Rightarrow Z) \Rightarrow (Y \Rightarrow X) \\
	\equiv \neg (Y \Rightarrow Z) \vee (Y \Rightarrow X) \\
	\equiv \neg (\neg Y \vee Z) \vee (\neg Y \vee X) \\
	\equiv (Y \wedge \neg Z) \vee (\neg Y \vee X) \\
	\equiv (Y \vee \neg Y \vee X) \wedge (\neg Z \vee \neg Y \vee X)\\
	\equiv (T \vee X) \wedge (\neg Z \vee \neg Y \vee X)\\
	\equiv T \wedge (\neg Z \vee \neg Y \vee X)\\
	\equiv \neg Z \vee \neg Y \vee X\\
	\equiv \psi_4 \\
	\) 
	
	\ \\ \( \varphi_5 = (X \wedge\neg Y) \vee (Y \wedge \neg X) \\
	\equiv \neg (\neg X \vee Y) \vee \neg (\neg Y \vee X) \\
	\equiv \neg (X \Rightarrow Y) \vee \neg (Y \Rightarrow X) \\
	\equiv \neg ((X \Rightarrow Y) \wedge (Y \Rightarrow X)) \\
	\equiv \neg (X \Leftrightarrow Y) \\
	\equiv \psi_5 \\
	\) 
	
	\subsection*{Aufgabe 4}
	Aus der VL wissen wir, dass zu jeder aussagenlogischen Formel eine äquivalente KNF existiert. \\
	Folglich gilt: $\varphi \equiv knf_{\varphi} = \bigwedge\limits_{i=1}^{n} \bigvee\limits_{j=1}^{m} L_{ij}$ \\
	Aus $\varphi \equiv True$ kann man folgern, dass alle Disjunktionsterme von $knf_{\varphi}$ wahr sein müssen.\\
	Des Weiteren gilt, dass $\chi$ die folgende äquivalente Form besitzt, wenn $\chi$ keine Tautologie ist:
	
	\ \\ \( \chi \equiv t_{\varphi 1} \vee ... \vee  t_{\varphi i} \vee \gamma \text{ mit } \\
	\gamma \in AL,~t_{\varphi i} \text{ ist ein Disjunktionsterm
	mit i $\in [1,u]$ und 1 $\leq$ u $\leq$ n von } \\ 
	knf_{\varphi} \text{ und } var(\gamma) ~\cap~ var(\varphi) = \emptyset \)
	
	\ \\Würde dies nicht gelten, so gäbe es eine passende Belegung $\beta$, \\
	sodass $\ldbrack \chi \rdbrack ^\beta \equiv False$, obwohl $\varphi \equiv True$. \\
	Das stünde im Widerspruch zur Aussage, dass $\varphi \Rightarrow \chi$ gilt. \\
	Ist $\chi$ eine Tautologie gilt: $t_{\varphi} = True$ \\
	Aus der Struktur von $\chi$ folgt ebenfalls, dass $t_{\varphi}$ folgende Bedingungen erfüllt:\\
	\ \\ \(\varphi \Rightarrow t_{\varphi} \wedge t_{\varphi} \Rightarrow \chi \wedge var(t_{\varphi}) \subseteq 		  	var(\varphi) \cap var(\chi) \)
	
	\ \\Daraus folgt, dass für alle Formeln $\varphi$, $\chi \in$ AL gilt mit $\varphi \Rightarrow \chi \equiv True$:
	
	\ \\ \( \exists \psi \in AL. \varphi \Rightarrow \psi \wedge 
	\psi \Rightarrow \chi \wedge var(\psi) \subseteq var(\varphi) \cap var(\chi) \)
	
	
	

	
	
	\end{Large}
\end{document}