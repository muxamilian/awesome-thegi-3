\documentclass[a4paper,10pt]{article}
\usepackage[utf8]{inputenc}
\usepackage{amsmath}
\usepackage{amsfonts}
\usepackage{amssymb}
\usepackage{fullpage}
\usepackage[german]{babel}
\setlength{\parindent}{0cm}
\usepackage{setspace}
\usepackage{mathpazo}
\usepackage{graphicx}
\usepackage{wasysym} 
\usepackage{booktabs}
\usepackage{verbatim}
\usepackage{pst-all}
\usepackage{enumerate}
\usepackage{pstricks}
\usepackage{ulem}
\usepackage{ stmaryrd }
\usepackage[a4paper,
left=1.8cm, right=1.8cm,
top=2.0cm, bottom=2.0cm]{geometry}
\newcommand{\N}{\mathbb{N}}
\newcommand{\A}{\mathcal{A}}
\newcommand{\ts}{\textsf}

\usepackage{tabularx}

\newcolumntype{L}[1]{>{\raggedright\arraybackslash}p{#1}}
\newcolumntype{C}[1]{>{\centering\arraybackslash}p{#1}}
\newcolumntype{R}[1]{>{\raggedleft\arraybackslash}p{#1}}


\begin{document}

\begin{center}
\Large{Theoretische Grundlagen der Informatik 3: Hausaufgabenabgabe 2} \\
\large{Tutorium: Sebastian , Mi 14.00 - 16.00 Uhr}
\end{center}
\begin{tabbing}
Tom Nick \hspace{2cm}\= - 340528\\
Maximillian Bachl \> - 341455 \\
Marius Liwotto\> -  341051
\end{tabbing}
	\subsection*{Aufgabe 1}

	\begin{enumerate}
	\item[(i)]
%	$E'' := \lbrace (v_1, v_2) | \lbrace v_1, v_2 \rbrace \in E \rbrace$\\
%	$G' := (V, t(s(r(E'')))$ mit $t(s(r(E)))$ = reflexiver, symmetrischer und transitiver Abschluss von $E$.
%	
%	\ \\Die Menge aller Pfade von $v_n$ nach $v_m$ mit $v_n, v_m \in V$ in G':\\	
%	\( P_{v_n,v_m} := \{p~|~\text{wenn p ein Pfad von $v_n$ zu $v_m$ ist mit $v_n, v_m \in V$} \}  \)
%	
%	\ \\Ein Pfad $p \in P_{v_n,v_m}$ von $v_n$ nach $v_m$ ist eine Menge aus Tupeln $(v_i,v_j)$ mit:\\
%	\( \forall (v_i,v_j) \in p.~\exists (v_j,v_l) \in p \vee (v_j,v_m).~\top \)
%	% Diese Formel ergibt für mich irgendwie keinen Sinn -- Max
%	
%	\ \\ \( \phi(P_i) := \begin{cases}
%	X_{ij} \wedge \phi(P_i\setminus \{(v_i,v_j)\}), & \#(P_i) \geq 2 \wedge (v_i,v_j) \in P_i \\
%	X_{ij},  & \#(P_i) = 1 \wedge (v_i,v_j) \in P_i
%	\end{cases} \)
%	
%	\ \\Jeder Pfad $P_i \in P_{v_n,v_m}$ wird mit $\phi(P_i)$ zu einem Konjunktionsterm aus $X_{ij}$ \\
%	umgebaut.
%		
%	\ \\ \( \varphi_n = \bigwedge\limits_{l = 1}^{n} \bigwedge\limits_{k = 1}^{n} 
%	(\bigvee\limits_{c = 1}^{\#(P_{v_l,v_k})} \phi(P_c) \text{ mit }P_c \in P_{v_l,v_k}) \)
%		
%	\ \\$\varphi_n$ wird gdw. wahr, wenn für jeden Knoten $v_n$ gilt, dass er zu jedem anderen Knoten $v_m$ \\
%	einen Pfad besitzt, der auch in G enthalten ist. \\
%	Daraus folgt: $\llbracket \varphi_n \rrbracket^{\beta_G} = 1 \Leftrightarrow$ G ist zusammenhängend
Ein Graph $G = (V,E)$ ist zusammenhängend, falls von einem beliebigen Knoten $v_n \in V$ ein Pfad zu jedem anderen Knoten $v_m \in V$ existiert. 

Man definiert $E' = \bigcup\limits_{n \in \mathbb{N}\setminus{\{0\}}}  \{(x,y) | \{ x,y \} \in E \}^n$. $E'$ ist also der reflexive, symmetrische und transitive Abschluss von E.

Man kann sehr einfach feststellen ob ein Knoten eine Verbindung zu jedem anderen Knoten hat, dazu bilde man den reflexiven, transitiven und symmetrischen Abschluss von $E$ und bezeichne ihn $E''$. Weiterhin definiere  $X'_{ij}$ und erweitere den Zielbereich von $\beta_G$.
$$\llbracket X'_{ij} \rrbracket = 1 \Leftrightarrow \{v_i,v_j\} \in E''$$
Wenn nun alle Knoten untereinander verbunden sind (wenn $E''$ als Kantenmenge benutzt wird) dann ist der Graph $G$ zusammenhängend.
$$\varphi_n = \bigwedge_{i=1}^n\bigwedge_{j=i+1}^n X'_{ij}$$


	\item[(ii)]
%	\( N_k = \{X_{kj}~|~\{v_k,v_j\} \in E\} \)
%	
%	\ \\ \( \phi(k) = \begin{cases}	
%	(X_{ij} \wedge X_{i(j+1)}) \wedge (\neg X_{i(j+2)} \wedge ... \wedge \neg X_{i(j+m)}), & \#(N_k) \geq 2\\
%	X_{ij}, & \#(N_k) = 1 \wedge X_{ij} \in N_k 	
%	\end{cases} \)
%	
%	\ \\ \( \varphi_n = \bigwedge\limits_{k = 1}^{\#V} (\phi(k)) 
%	\wedge \left( \bigwedge\limits_{l = 1}^{n} \bigwedge\limits_{k = 1}^{n} 
%	(\bigvee\limits_{c = 1}^{\#(P_{v_l,v_k})} \phi(P_c) \text{ mit }P_c \in P_{v_l,v_k})\right)  \)
%	
%	\ \\Wenn ein Graph einen Hamilton-Kreis besitzt, so muss man ihn einmal komplett durchlaufen können, \\
%	sodass man jeden Knoten genau einmal besucht und wieder beim Anfangsknoten landet. Folglich muss der Graph ein 	       	Kreis sein bzw., man kann die Knoten so verschieben, dass ein Kreis entsteht, 
%	der äquivalent zum Ausgangsgraph ist. \\
%	Diese Kreisstruktur bedeutet, dass jeder Knoten genau 2 Nachbarn hat. \\
%	Die Funktion $\phi(k)$ konstruiert für jeden Knoten $v_i$ einen Konjunktionsterm so, dass wenn $v_i$  mehr als \\
%	2 Nachbarn hat, der Term zu \bot auswertet.\\
%	$\varphi_n$ prüft, ob $\phi$ für jeden Knoten $v_i \in V$ gilt und ob der Graph abgeschlossen \\
%	ist mit der Formel aus (i).
Wenn ein Graph einen Hamilton Kreis besizt, dann gibt es von einem beliebigen Knoten aus einen Pfad der Länge $n$, wobei $n$ die Anzahl der Knoten ist der wieder zu diesem führt. Zuerst erweitere die Definition von $X_{ij}$:
$$X''_{ij} \rightarrow \begin{cases}X_{ij}, & i < j \\ X_{ji}, & i > j \\ 0, & i = j \end{cases} $$
Sei weiterhin die Menge $P_V$ die Menge aller möglichen Permutationen der Reihenfolge der Knoten $V$, diese Folgen sind dargestellt als n-Tupel. Die Schreibweise $t[x]$ bei Tupeln soll das $x$-te Element des Tupels $t$ sein.  
$$\varphi_n =  \bigvee_{p \in P_V}\left(\left(\bigwedge_{k=1}^{n-1} X''_{p[k],p[k+1]} \right) \land X''_{p[n],p[1]} \right)$$

	\end{enumerate}
	
	\subsection*{Aufgabe 2}	
	\begin{enumerate}[(i)]
	\item 
	\begin{align*}
	\varphi \mathcal{S} &= (((X \land Y) \lor Z) \leftrightarrow (((X \lor \neg Y) \leftrightarrow Y))\mathcal{S} \\
	&= (((X \land Y) \lor Z)\mathcal{S} \leftrightarrow (((X \lor \neg Y)\mathcal{S} \leftrightarrow Y) ) \\
	&= (((X \land Y)\mathcal{S}  \lor Z\mathcal{S} ) \leftrightarrow (((X \lor \neg Y)\mathcal{S}  \leftrightarrow Y\mathcal{S} ) ) \\
	&= (((X\mathcal{S} \land Y\mathcal{S})  \lor Z ) \leftrightarrow (((X\mathcal{S} \lor \neg Y\mathcal{S})  \leftrightarrow (Y \leftrightarrow (Z \rightarrow (Y \land Z)) ) ) \\
	&= ((((Z \lor U) \land (Y \leftrightarrow (Z \rightarrow (Y \land Z)))  \lor Z ) \leftrightarrow (((Z \lor U) \lor \neg (Y \leftrightarrow (Z \rightarrow (Y \land Z)))  \leftrightarrow (Y \leftrightarrow (Z \rightarrow (Y \land Z)))) \\	\end{align*}
	\item
	\begin{align*}
	\beta\mathcal{S}(X) &= \llbracket \mathcal{S}(X)\rrbracket^\beta \\
	\beta\mathcal{S}(Y) &= \llbracket \mathcal{S}(Y)\rrbracket^\beta \\
	\beta\mathcal{S}(U) &= \beta(U)
	\end{align*}
	\begin{itemize}
	\item $\beta\mathcal{S}$ ist passend für $\varphi$, da $\{X, Y\} = \textsf{var}(\varphi) \subseteq \textsf{Dom}(\beta\mathcal{S}) = \{X,Y,U \}$ \\
	$\beta$ ist passend für $\varphi\mathcal{S}$, da $\{X, Y, U\} = \textsf{var}(\varphi\mathcal{S}) \subseteq \textsf{Dom}(\beta) = \{X,Y,U \}$ 
	\item Da $\beta$ eine zu $\varphi\mathcal{S}$ passende Belegung ist und  $\beta\mathcal{S}$ zu $\varphi$ passt, kann Lemma 2.27-A aus der Vorlesung angewendet werden, welches dort auch allgemein bewiesen wird.
	%Dies ist im Substitutionslemma bewiesen, was ja eine vollständige Verifikation ist, dass es auch mit der  hier angegebenen Formel, Substitution und Belegung gilt.
	
	\end{itemize}
		
	\end{enumerate}
	\subsection*{Aufgabe 3}
	\begin{align*}
	\varphi_1 \equiv X \Rightarrow (Y \wedge Z) 
	&\equiv \neg X \vee (Y \wedge Z) \\
	&\equiv (\neg X \vee Y) \wedge (\neg X \vee Z) \\
	&\equiv (X \Rightarrow Y) \wedge (X \Rightarrow Z) \\
	&\equiv \psi_1 \\
	\end{align*}
	\begin{align*}
	\varphi_2 \equiv (X \wedge Y \wedge Z) \Rightarrow Q 
	&\equiv \neg (X \wedge Y \wedge Z) \vee Q \\
	&\equiv \neg X \vee \neg Y \vee \neg Z \vee Q \\
	&\equiv (\neg X \vee (\neg Y \vee (\neg Z \vee Q))) \\
	&\equiv (X \Rightarrow (Y \Rightarrow (Z \Rightarrow Q))) \\
	&\equiv \psi_2 
	\end{align*}	 \
	
	\begin{align*}
	\varphi_3 \equiv (X \wedge Y) \Rightarrow \neg (Z \Rightarrow X) 
	&\equiv \neg (X \wedge Y) \vee \neg (\neg Z \vee X)  \\
	&\equiv (\neg X \vee \neg Y) \vee (Z \wedge \neg X)  \\
	&\equiv (\neg X \vee \neg Y \vee Z) \wedge (\neg X \vee \neg Y \vee \neg X)  \\
	&\equiv (\neg X \vee \neg Y \vee Z) \wedge (\neg X \vee \neg Y)  \\
	&\equiv (\neg X \vee \neg Y) \\
	&\equiv (\neg X \vee \neg Y) \wedge \top \\
	&\equiv (\neg X \vee \neg Y) \wedge (\neg X \vee X) \\
	&\equiv (\neg X \vee \neg Y) \wedge (Y \vee \neg X \vee X) \\
	&\equiv (\neg X \vee \neg Y) \wedge (Y \vee X) \vee \neg X \\
	&\equiv (X \Rightarrow \neg Y) \wedge (\neg Y \Rightarrow X) \vee \neg X \\
	&\equiv \psi_3 
	\end{align*}
	
	\begin{align*}
	\varphi_4 \equiv (Y \Rightarrow Z) \Rightarrow (Y \Rightarrow X) 
	&\equiv \neg (Y \Rightarrow Z) \vee (Y \Rightarrow X) \\
	&\equiv \neg (\neg Y \vee Z) \vee (\neg Y \vee X) \\
	&\equiv (Y \wedge \neg Z) \vee (\neg Y \vee X) \\
	&\equiv (Y \vee \neg Y \vee X) \wedge (\neg Z \vee \neg Y \vee X)\\
	&\equiv (T \vee X) \wedge (\neg Z \vee \neg Y \vee X)\\
	&\equiv T \wedge (\neg Z \vee \neg Y \vee X)\\
	&\equiv \neg Z \vee \neg Y \vee X\\
	&\equiv \psi_4 
	\end{align*}
	
	\begin{align*}
	\varphi_5 = (X \wedge\neg Y) \vee (Y \wedge \neg X) 
	&\equiv \neg (\neg X \vee Y) \vee \neg (\neg Y \vee X) \\
	&\equiv \neg (X \Rightarrow Y) \vee \neg (Y \Rightarrow X) \\
	&\equiv \neg ((X \Rightarrow Y) \wedge (Y \Rightarrow X)) \\
	&\equiv \neg (X \Leftrightarrow Y) \\
	&\equiv \psi_5 
	\end{align*}
	
	\subsection*{Aufgabe 4}
	Aus der VL wissen wir, dass zu jeder aussagenlogischen Formel eine äquivalente KNF existiert. \\
	Folglich gilt: $\varphi \equiv knf_{\varphi} = \bigwedge\limits_{i=1}^{n} \bigvee\limits_{j=1}^{m} L_{ij}$ \\
	Aus $\varphi \equiv \top$ kann man folgern, dass alle Disjunktionsterme von $knf_{\varphi}$ wahr sein müssen.\\
	Des Weiteren gilt, dass $\chi$ die folgende äquivalente Form besitzt:
	
	\ \\ \( \chi \equiv t_{\varphi} \vee \gamma \text{ mit } \\
	\gamma \in AL,~t_{\varphi} \text{ ist ein Disjunktionsterm von } knf_{\varphi} \) 	
	
	\ \\Würde dies nicht gelten, so gäbe es eine passende Belegung $\beta$, \\
	sodass $\llbracket \chi \rrbracket ^\beta \equiv \bot$, obwohl $\varphi \equiv \top$. \\
	Das stünde im Widerspruch zur Aussage, dass $\varphi \Rightarrow \chi$ gilt. \\
	Aus der Struktur von $\chi$ folgt ebenfalls, dass $t_{\varphi}$ folgende Bedingungen erfüllt:\\
	\ \\ \(\varphi \Rightarrow t_{\varphi} \wedge t_{\varphi} \Rightarrow \chi \wedge var(t_{\varphi}) \subseteq 		  	var(\varphi) \cap var(\chi) \)
	
	\ \\Daraus folgt, dass für alle Formeln $\varphi$, $\chi \in$ AL gilt mit $\varphi \Rightarrow \chi \equiv \top$:
	
	\ \\ \( \exists \psi \in AL. \varphi \Rightarrow \psi \wedge 
	\psi \Rightarrow \chi \wedge var(\psi) \subseteq var(\varphi) \cap var(\chi) \)
\end{document}